\documentclass{myreport}

\usepackage{longtable}

% % Change figure (table, section) numbering (e.g., from 'Figure 1' to 'Figure S1')
%  \renewcommand{\thefigure}{S\arabic{figure}}
%  \renewcommand{\thetable}{S\arabic{table}}
%  \renewcommand{\thesection}{S\arabic{section}}
%  \renewcommand{\theequation}{S\arabic{equation}}

\newcommand{\coo}{CO$_2$}

\begin{document}
\pagestyle{headings}

% Document must include
% ---------------------
% 

%% Title
\title{P-model evaluation}

\maketitle

%\tableofcontents

%%%%%%%%%%%%%%%%%%%%%%%%%%%%%%%%%%%%%%%%%%%%%%%%%%%%%%%%
%\section{Introduction}

% should clearly state what the difference between this version and the Wang Han et al. 2017 formulation is
\section{Introduction}

\section{Theory}

The P-model centers around a prediction for the ratio of leaf-internal to ambient \coo ($c_i : c_a = \chi$) governed by the trade-off between the costs arising from the maintenance of carboxylation capacity ($V_{\mathrm{cmax}}$) and transpiration ($E$). 
It is founded on the standard model for C3 plant photosynthesis (Section \ref{sec:farquhar}), the coordination hypothesis (Section \ref{sec:coordination}) and the least-cost hypothesis (Section \ref{sec:least-cost}).

\textit{The following derivation and text is adopted and modified from Wang Han et al., 2017.}\\

\subsection{The standard model of C3 plant photosynthesis}
\label{sec:farquhar}
Following the Farquhar model for photosynthesis of C3 plants, instantaneous assimilation rates $A$ are limited either by the capacity of Rubisco for carboxylation of RuBP or the electron transport rate for the regeneration of RuBP. 
The Rubisco-limited photosynthetic rate $A_C$ is given by:
\begin{equation}
\label{eq:rubiscolimited}
    A_C = V_{\mathrm{cmax}} \; \frac{\chi\;c_a-\Gamma^{\ast}}{\chi\;c_a + K}
\end{equation}
where $V_{\mathrm{cmax}}$ is the Rubisco activity, $c_a$ is the ambient partial pressure of CO$_2$, $\chi$ is the ratio of leaf-internal to ambient CO$_2$ partial pressures, $\Gamma^{\ast}$ is the \coo\ compensation point in the absence of mitochondrial respiration, and $K$ is the effective Michaelis-Menten coefficient of Rubisco for carboxylation. 
Both $\Gamma^{\ast}$ and $K$ are influenced by the partial pressure of oxygen. 
$V_{\mathrm{cmax}}$, $\Gamma^{\ast}$ and $K$ are temperature-dependent, following Arrhenius kinetics.

The electron-transport limited photosynthetic rate $A_J$ is given by:
\begin{equation}
\label{eq:lightlimited}
    A_J = \phi_0 \; I\; \frac{\chi \; c_a - \Gamma^{\ast}}{\chi\;c_a + 2\Gamma^{\ast}}
\end{equation}
for low PPFD (I) where $\phi_0$ is the intrinsic quantum efficiency of photosynthesis. 
With increasing PPFD, $I$ must be substituted with a saturating function of the electron-transport capacity $J_{\mathrm{max}}$; various empirical functions have been used. 
The actual photosynthetic rate is then given by:
\begin{equation}
    A = \min(A_C, A_J)
\end{equation}

\subsection{The coordination hypothesis}
\label{sec:coordination}
Light use efficiency (LUE) models provide a powerful method and estimate assimilation rates as a linear function of absorbed light over a given time interval. 
But the connection between the Farquhar and LUE models is not obvious. 
Equation \ref{eq:lightlimited} predicts that electron-transport limited photosynthesis is proportional to absorbed PPFD but only applies at relatively low PPFD, and in any case, Equation \ref{eq:rubiscolimited} for Rubisco-limited photosynthesis is expected to apply at high PPFD. 
The conundrum is this: how can GPP (the time-integral of photosynthesis) be proportional to PPFD, which is the basis of the LUE model, if the response of photosynthesis to PPFD saturates, as it should according to the Farquhar model? This question has surfaced occasionally in the literature, but not been fully resolved. 
Medlyn10 reviewed some alternative explanations.
 
One of the explanations discussed by Medlyn10 invokes the co-ordination (or co-limitation) hypothesis, which states that $V_{\mathrm{cmax}}$ of leaves at any level in the canopy acclimates spatially and temporally to the prevailing daytime incident PPFD in such a way as to be neither in excess (entailing additional, futile maintenance respiration), nor less than required for full exploitation of the available light. 
In other words, under typical daytime condition when most photosynthesis takes place, the following is valid:
\begin{equation}
\label{eq:coordination}
    A_J \approx A_C
\end{equation}
This hypothesis also requires that $J_{\mathrm{max}}$ maintain a ratio to $V_{\mathrm{cmax}}$ such that strong limitation by $J_{\mathrm{max}}$ is avoided. 
Evidence for the co-ordination hypothesis was presented by Haxeltine and Prentice11, and Dewar12, who noted that it can explain many otherwise unexplained responses of C3 plants to environmental changes: including changes in leaf C:N ratios along environmental gradients, and the widely observed reduction of $V_{\mathrm{cmax}}$ under experimentally increased atmospheric CO2. More recently Maire et al.13 showed very good agreement between typical daytime values of $A_J$ and $A_C$ as calculated under the prevailing growth conditions for 31 species (293 data points) based on published studies. 
The co-ordination hypothesis allows a simple approximation by which equation S2 is applied to predict GPP over time scales for which acclimation of $V_{\mathrm{cmax}}$ is possible, with $I$ now representing daily, rather than instantaneous, PPFD.

\subsection{The least-cost hypothesis}
\label{sec:least-cost}
Missing from the Farquhar model is an equation to predict χ, which constrains both the Rubisco- and electron-transport limited rates of carbon fixation and therefore appears in both equations S1 and S2. 
$\chi$ at any moment must be consistent both with the rate of carbon fixation and with the rate of diffusion of \coo\ through the stomata. 
Although the mechanism of stomatal control is still an active research topic, there is abundant evidence that $\chi$ is closely regulated to remain within a narrow range. 
All current Earth System Models include a 'closure' that predicts either stomatal conductance or $\chi$. 
The most commonly used closures are the one-parameter Ball-Berry equation14 and the two-parameter Leuning equation15 (or equivalently, the ‘Jacobs closure’16). 
Both are empirical, and incomplete in the sense that they allow $\chi$ to react only to relative humidity (Ball-Berry) or VPD (Leuning/Jacobs). 
Although superficially similar, these equations make substantially different predictions. 
For example, Ball-Berry allows $\chi$ to approach unity as VPD tends to zero, whereas Leuning/Jacobs caps $\chi$ at a maximum value. 
Both equations are usually implemented with different parameter values for different PFTs, but with no strong basis for the distinctions.

The least-cost hypothesis by Prentice et al. (2014), first proposed by Wright et al. (2003), states that $\chi$ should minimize the combined cost (per unit of assimilation) of maintaining the capacities for carboxylation and transpiration. 
If $a$ and $b$ are dimensionless cost factors (maintenance respiration per unit assimilation) for the maximum rates of water transport ($E$) and carbon fixation ($V_{\mathrm{cmax}}$) respectively, the optimality criterion is:
\begin{equation}
\label{eq:leastcost}
    a \; \frac{\partial (E/A)}{\partial \chi} = -b \; \frac{\partial (V_{\mathrm{cmax}}/A)}{\partial \chi}
\end{equation}

\subsubsection{Predicting $\chi$}
The following section provides a derivation of optimal $\chi$ using Fick's Law, above stated hypotheses and the Farquhar model for C3 photosynthesis. 
Transpiration $E$ and assimilation $A$ are coupled through stomatal conductance ($g_s$).
According to the Fick's Law of diffusion:
\begin{align}
\label{eq:fick}
    E &= 1.6 \; g_s \; D \\
    A &= g_s \; c_a \; (1-\chi)
\end{align}
Therefore,
\begin{equation}
    E/A = \frac{1.6 \; D}{c_a\;(1-\chi)}
\end{equation}
The derivative term on the left-hand-side of Eq.\label{eq:leastcost} can thus be written as
\begin{equation}
\label{eq:partial1}
    \frac{\partial (E/A)}{\partial \chi} = \frac{1.6\;D}{c_a\;(1-\chi)^2}\;.
\end{equation}
Using Equation \ref{eq:rubiscolimited} and the simplification $\Gamma^{\ast}=0$, the derivative term on the right-hand-side of Eq.\label{eq:leastcost} can be written as
\begin{equation}
\label{eq:partial2}
    \frac{\partial (V_{\mathrm{cmax}}/A)}{\partial \chi} = - \frac{K}{c_a\;\chi^2}
\end{equation}
Using equations \ref{eq:partial1} and \ref{eq:partial2}, Eq. \ref{eq:leastcost} can be written as
\begin{equation}
    a\;\frac{1.6\;D}{c_a\;(1-\chi)^2} = b\;\frac{K}{c_a\;\chi^2}
\end{equation}
and solved for $\chi$:
\begin{align}
    \chi &= \frac{\xi}{\xi + \sqrt{D}} \\ 
    \xi &= \sqrt{\frac{b\;K}{1.6\;a}}
\end{align}
The exact solution, without the simplification $\Gamma^{\ast}=0$, is 
\begin{align}
\label{eq:chi_exact}
    \chi &= \frac{\Gamma^{\ast}}{c_a} + \left(1- \frac{\Gamma^{\ast}}{c_a}\right)\;\frac{\xi}{\xi + \sqrt{D}}\\
    \xi &= \sqrt{\frac{b(K+\Gamma^{\ast})}{1.6\;a}}
\end{align}
This can also be written as
\begin{equation}
\label{eq:ci}
    c_i = \frac{\Gamma^{\ast}\sqrt{D}+ \xi\;c_a}{\xi + \sqrt{D}} \\ 
\end{equation}
\clearpage

\subsection{The light use efficiency model}
With this prediction for $c_i$, acclimated at a time scale on the order of weeks, the assimilation rate, Eq. \ref{eq:lightlimited} can be used in the sense of a light use efficiency model, whereby the total assimilation is proportional to the total absorbed PPFD over a given time interval:
\begin{equation}
\label{eq:lue}
        A_J = \phi_0 \; I_{\mathrm{abs}}\;\underbrace{\frac{c_i - \Gamma^{\ast}}{c_i + 2\Gamma^{\ast}}}_{m}
\end{equation}
Using Eq. \ref{eq:ci} and $\beta=b/a$, $m$ can be written as
\begin{equation}
    m = \frac{c_a - \Gamma^{\ast}}{c_a + 2 \Gamma^{\ast} + 3 \Gamma^{\ast} \sqrt{\frac{1.6 \eta^{\ast} D }{\beta\;(K+\Gamma^{\ast})}}}
\end{equation}
This provides an expression for predicting the assimilation rate from first principles as a function of temperature, moisture (vapour pressure deficit $D$), elevation and atmospheric CO$_2$ partial pressure.

\subsection{Introducing $J_{\mathrm{max}}$ limitation}
Equation \ref{eq:lue} is correct only if the response of GPP ($A$) to increasing PPFD remains linear up to the co-limitation point. 
By considering a non-rectangular hyperbola relationship between $A_J$ and $I_{\mathrm{abs}}$ (ref 26), we allow for the effect of finite $J_{\mathrm{max}}$:
\begin{equation}
\label{eq:ajlim}
    A_J = \phi_0 \; I_{\mathrm{abs}} \; m \; \underbrace{ \frac{1}{\sqrt{1+ \left( \frac{4\;\phi_0\;I_{\mathrm{abs}}}{J_{\mathrm{max}}} \right)^{2}}} }_{L}
\end{equation}
In the following, we aim for an expression of the limitation factor $L$ as a function of a $J_{\mathrm{max}}$ cost factor $c^{\ast}$. 
We define two dimensionless quantities, $a_0$ and $k$:
\begin{equation}
\label{eq:a0}
    a_0 = \frac{J_{\mathrm{max}}}{4\;\phi_0\;I_{\mathrm{abs}}}
\end{equation}
\begin{equation}
    k = \frac{J_{\mathrm{max}}}{4\;V_{\mathrm{cmax}}}\;.
\end{equation}
The model equations for $A_J$ and $A_C$ can then be re-written as:
\begin{equation}
\label{eq:ajlim2}
    A_J = \phi_0 \; I_{\mathrm{abs}} \; \; \frac{c_i - \Gamma^{\ast}}{c_i + 2\Gamma^{\ast}} \frac{1}{\sqrt{1+a_0^{-2}}}
\end{equation}
and
\begin{equation}
    A_C = \frac{J_{\mathrm{max}}}{4\;k} \cdot \frac{c_i - \Gamma^{\ast}}{c_i + K}
\end{equation}
Under the assumption of the coordination hypothesis, we set again $A_J = A_C$ and solve for $k$ to define the ratio of $J_{\mathrm{max}}$ to $V_{\mathrm{cmax}}$:
\begin{equation}
\label{eq:k}
    k = \frac{c_i + 2\Gamma^{\ast}}{c_i + K} \; \sqrt{a_0^2 + 1}
\end{equation}
Solving Eq. \ref{eq:k} for $a_0$ and substituting this into Eq. \ref{eq:ajlim} we can express the $J_{\mathrm{max}}$ limitation factor $L$ as:
\begin{equation}
\label{eq:ajlim3}
    L = \frac{1}{\sqrt{1 - \left( \frac{c_i+2\Gamma^{\ast}}{k(c_i+K)} \right)^{2}}}
\end{equation}

To obtain an estimate of the optimum value of $J_{\mathrm{max}}$ we assume that (a) there is a cost associated with $J_{\mathrm{max}}$ that is equal to the product of $J_{\mathrm{max}}$ and a constant $c^{\ast}$, and (b) that the value of $J_{\mathrm{max}}$ maximizes the benefit ($A_J$) minus the cost. 
This maximum is obtained when
\begin{equation}
\label{eq:jmaxpartial}
    \frac{\partial A_J}{\partial a_0} = c \; \frac{\partial J_{\mathrm{max}}}{\partial a_0}\;.
\end{equation}
Using Eq. \ref{eq:ajlim2}, the left-hand-side of Eq. \ref{eq:jmaxpartial} is 
\begin{equation}
    \frac{\partial A_J}{\partial a_0} = \phi_0 \; I_{\mathrm{abs}} \; \; \frac{c_i - \Gamma^{\ast}}{c_i + 2\Gamma^{\ast}} \left( a_0^{-2} + 1 \right) ^{-2/3} a_0^{-3}
\end{equation}
Using Eq. \ref{eq:a0}, the right-hand-side of Eq. \ref{eq:jmaxpartial} is 
\begin{equation}
    \frac{\partial J_{\mathrm{max}}}{\partial a_0} = 4 \; \phi_0 \; I_{\mathrm{abs}}\;.
\end{equation}
Now, we can solve Eq. \ref{eq:jmaxpartial} for $a_0^2$:
\begin{equation}
\label{eq:a0}
a_0^2 = \frac{(c_i - \Gamma^{\ast})^{2/3}}{c^{2/3} \; (c_i+2\Gamma^{\ast})^{2/3}}-1
\end{equation}
and use this in Eq. \ref{eq:k} to solve for $k^2$:
\begin{equation}
\label{eq:k2}
k^2 = (c_i+ \Gamma^{\ast})^{4/3} \; (c_i - \Gamma^{\ast})^{2/3} \; (c_i + K)^{-2} \; c^{-2/3}
\end{equation}
and for $c$:
\begin{equation}
\label{eq:c}
    c = \frac{(c_i+2\Gamma^{\ast})^2\;(c_i-\Gamma^{\ast})}{k^3\;(c_i+K)^3}
\end{equation}
Eq. \ref{eq:k2} can be plugged into Eq. \ref{eq:ajlim3} to express $L$ as
\begin{equation}
    L = \sqrt{1-c^{2/3} \; \left( \frac{c_i+2\Gamma^{\ast}}{c_i-\Gamma^{\ast}}\right)^{2/3}  }
\end{equation}
Using the prediction of $c_i$ from Eq. \ref{eq:ci}, this can be written as
\begin{equation}
    L =  \sqrt{1 - \left( \frac{c}{m} \right)^{2/3} }
\end{equation}
The revised LUE model thus becomes
\begin{equation}
\label{eq:ajlim4}
    A_J = \phi_0 \; I_{\mathrm{abs}} \; m \; \sqrt{1 - \left( \frac{c}{m} \right)^{2/3} }
\end{equation}
Taking typical values of $k$ = 0.4726 and $\chi$ = 0.820, we estimate (using Eq. \ref{eq:c}) c = 0.41.

% \subsection{Corollary of the $\chi$ prediction}
% \subsubsection{Stomatal conductance}
% Stomatal conductance $g_s$ follows from the prediction of $\chi$ given by Eq. \ref{eq:chi_exact} and $g_s = A / ( c_a\;(1-\chi) )$ (from Eq. \ref{eq:fick}). Stomatal contuctance can thus be written as
% \begin{equation}
%     g_s = \left( 1 + \frac{\xi}{\sqrt{D}} \right) \frac{A}{c_a}
% \end{equation}
% This is equivalent to the form derived by Medlyn et al., 2011, apart from the $g_0$ parameter that is missing here.

% $g_s$ also follows from the predictions of $A$ and $ci$, using Eq. \ref{eq:fick}. The stomatal conductance to water vapour (not CO$_2$) is:
% \begin{equation}
% g_s^W = \frac{1.6 \; p\; A}{c_a - c_i}
% \end{equation}
% With $g_s^W$ commonly expressed in units of mol H$_2$O m$^{-2}$ s$^{-1}$, $g_s$ in P-model being the stomatal conductance to CO$_2$, and $c_i$ (and $c_a$) defined as CO$_2$ partial pressure in units of Pa, multiplication with atmospheric pressure $p$ (Pa) and the factor 1.6 to convert stomatal conductance to CO$_2$ into stomatal conductance to H$_2$O are required.

% Note that in the P-model output, $c_i$ is given in ppm.

% \subsubsection{Intrinsic water use efficiency}
% The intrinsic water use efficiency (iWUE) is defined as the ratio of assimilation over stomatal conductance (to water). With Eq. \ref{eq:fick} this is
% \begin{equation}
%     \mathrm{iWUE} = A / g_s^W = \frac{c_a - c_i}{1.6\;p}
% \end{equation}
% With $A$ expressed in units of mol CO$_2$ m$^{-2}$ s$^{-1}$, and $g_s^W$ in units of mol H$_2$O m$^{-2}$ s$^{-1}$, iWUE is unitless. The factor 1.6 accounts for the difference in diffusivity between CO$_2$ and H$_2$O. In all above calculations, $c_a$ and $c_i$ are expressed as partial pressure of CO$_2$ in units of Pa. Division by the atmospheric pressure $p$ converts $(c_a - c_i)$ to unitless.


% \subsubsection{$V_{\mathrm{cmax}}$}

% Rearranging Eq. \ref{eq:rubiscolimited} and $\gamma=\Gamma^{\ast}/c_a$ and $\kappa=K/c_a$ gives
% \begin{equation*}
%     V_{\mathrm{cmax}} = A_C\;\frac{\chi+\kappa}{\chi-\gamma}
% \end{equation*}
% With the coordination $A_J=A_C$ (Eq. \ref{eq:coordination}) and Eq. \ref{eq:lue}, $V_{\mathrm{cmax}}$ can be written as
% \begin{equation}
%     \label{eq:vcmax}
%     V_{\mathrm{cmax}} = \phi_0\;I_{\mathrm{abs}}\;\frac{\chi+\kappa}{\chi+2\gamma}
% \end{equation}
% %Normalising to standard temperature (25$^{\circ}$C) gives $V_{\mathrm{cmax25}}$ as
% %\begin{equation}
% %    V_{\mathrm{cmax25}} = V_{\mathrm{cmax}} \; \exp\Bigg(\frac{\Delta H_{\mathrm{Vcmax}}}{R}\Big( \frac{1}{T_{K25}} - \frac{1}{T_K} \Big)\Bigg)
% %\end{equation}
% %with $\Delta H_{\mathrm{Vcmax}}=65330$ J mol$^{-1}$, $R = 8.3145$ J mol$^{-1}$ K$^{-1}$, $T_K$ is the ambient temperature in Kelvin, and $T_{K25}$ is 25$^{\circ}$C in Kelvin (298.15 K).
% %Wang Han (in prep.) implemented this as
% Normalising to standard temperature (25$^{\circ}$C) gives $V_{\mathrm{cmax25}}$ as
% \begin{equation}
% \label{eq:vcmaxsens}
%     V_{\mathrm{cmax25}} = V_{\mathrm{cmax}}\; f_V^{-1}  = V_{\mathrm{cmax}} \; \Bigg( \exp\Big( \frac{H_a(T_K-T_{K25})}{T_KT_{K25}R}\;\frac{1+\exp( \frac{T_{K25}\Delta S-H_d}{RT_{K25}} )}{1+\exp( \frac{T_K\Delta S - H_d}{RT_K} )} \Big) \Bigg)^{-1}
% \end{equation}
% with $H_a$ being the activation energy (71513 J mol$^{-1}$), $R$ is the universal gas constant (8.3145 J mol$^{-1}$ K$^{-1}$), $T_K$ is the ambient temperature in Kelvin, and $T_{K25}$ is 25$^{\circ}$C in Kelvin (298.15 K). $H_d$ is the deactivation energy (200000 J mol$^{-1}$), and $\Delta S$ is an entropy term (J mol$^{-1}$ K$^{-1}$) calculated using a linear relationship with $T$ from Kattge and Knorr (2007), with a slope of 1.07 J mol$^{-1}$ K$^{-2}$ and intercept of 668.39 J mol$^{-1}$ K$^{-1}$. 
% %Wang Han (in prep.) implemented this as
% %\begin{equation}
% %    V_{\mathrm{cmax25}} = V_{\mathrm{cmax}} \; \exp\Bigg( \frac{H_a(T-T_{\text{ref}})}{T_{\text{ref}}TR}\;\frac{1+\exp( \frac{T_{\text{ref}}\Delta S - H_d}{RT_{\text{ref}}}  )}{1+\exp( \frac{T\Delta S-H_d}{RT} )} \Bigg)
% %\end{equation}

% \subsubsection{Dark respiration $R_{\mathrm{d}}$}

% Dark respiration at standard temperature $R_{\mathrm{d25}}$ is calculated as being proportional to $V_{\mathrm{cmax25}}$:
% \begin{equation}
% \label{eq:rd25}
%     R_{\mathrm{d25}} = b_0 \; V_{\mathrm{cmax25}}
% \end{equation}
% where $b_0 = 0.014$. 
% Dark respiration follows a different temperature sensitivity. Following Heskel et al. (2016):
% \begin{equation}
% \label{eq:rdsens}
%     R_{\mathrm{d}} =  R_{\mathrm{d25}}\; f_R  = R_{\mathrm{d25}} \exp \Big(  0.1012(T_{K25}-T_K) - 0.0005(T_{K25}^2-T_K^2) \Big) 
% \end{equation}
% By combining Equations \ref{eq:vcmaxsens}, \ref{eq:rd25}, and \ref{eq:rdsens}, $R_d$ at growth temperature $T$ can directly be calculated from $V_{\mathrm{cmax}}$
% \begin{equation}
%     R_d = b_0 \frac{f_R}{f_V}\;V_{\mathrm{cmax}}
% \end{equation}

%Dark respiration at ambient temperature follows a sensitivity with a different activation energy than $\Delta H_{\mathrm{Vcmax}}$:
%\begin{equation}
%    R_{\mathrm{d}} = R_{\mathrm{d25}} \Bigg( \exp\Bigg(\frac{\Delta H_{\mathrm{Rd}}}{R}\Big( \frac{1}{T_{K25}} - \frac{1}{T_K} \Big)\Bigg) \Bigg)^{-1}
%\end{equation}

\subsection{Temperature dependence of quantum yield efficiency}

\subsection{Soil moisture stress}

\section{Implementation, forcing data description}

\subsection{Soil moisture model}

The soil water balance is solved following the SPLASH model and accounting only for liquid water. WATCH-WFDEI $P_{\text{rain}}$ and $P_{\text{snow}}$ are summed and converted from kg m$^{-2}$ s$^{-1}$ to mm d$^{-1}$ by multiplication with $(60 \cdot 60 \cdot 24)$ s d$^{-1}$. 

% I'm a bit confused. In the SPLASH documentation, we used the term 'soil holding capacity' to refer to what one may rather call the field capacity and expressed it in units of mm. What you termed 'water holding capacity, WHC' may be refered to as 'porosity' instead. We may rename terms in Eq. 77 of the SPLASH documentation (field capacity instead of soil moisture capacity) and write this as follows:

To obtain the total soil water holding capacity (WHC, in mm), we use data on soil depth ($d$, in m) and porosity ($\phi$, in $m_{H_{2}O}^{3} \cdot m_{Soil}^{-3}$) and calculate $W_m$ as
\begin{equation}
    W_m = \phi \cdot d \cdot 10^{-3}
\end{equation}
Runoff is generated when ...


Where, the water holding capacity, defined as volumetric proportion, was estimated, following Hillel (1982), as the difference between field capacity and wilting point.
\begin{equation}
\text{WHC}=W_{\text{FC}}-W_{\text{PWP}}
\end{equation}

Field capacity and wilting point, were obtained from texture and organic matter content data, through pedotransfer functions, described in Saxton \& Rawls (2006) as follows:
\begin{equation}
W_{\text{FC}}= W_{\text{FC}}_{0}+(1.283\cdot W_{\text{FC}}_{0}^{2}-0.374\cdot W_{\text{FC}}_{0}-0.015)
\end{equation}
Where,
\begin{align}
W_{\text{FC}}_{0} &=-0.251f_{\text{sand}}+195f_{\text{clay}}+0.011f_{\text{OM}}\\                            &+0.006(f_{\text{sand}}\cdot f_{\text{OM}})\\
                  &-0.027(f_{\text{clay}}\cdot f_{\text{OM}})\\
                  &+0.452 (f_{\text{sand}}\cdot f_{\text{clay}})\\
                  +0.299
\end{align}
And,
\begin{equation}
W_{\text{PWP}}=W_{\text{PWP}}_{0}+(0.14\cdot W_{\text{PWP}}_{0}-0.02)
\end{equation}
Where,
\begin{align}
W_{\text{PWP}}_{0} &=0.024 f_{\text{sand}} + 0.487 f_{\text{clay}} + 0.006 f_{\text{OM}} \\
                  &+0.005 ( f_{\text{sand}}\cdot f_{\text{OM}} )\\
                  &-0.013( f_{\text{clay}}\cdot f_{\text{OM}} )\\
                  &+0.068( f_{\text{sand}} \cdot f_{\text{clay}} )\\
                  &+0.031
\end{align}
Where, $W_{\text{FC}}_{0}$ and $W_{\text{PWP}}_{0}$ are the field capacity and wilting point first solutions respectively. And, $f_{\text{sand}}$, $f_{\text{clay}}$, $f_{\text{OM}}$ are the sand, clay and organic matter contents.
Contents of sand, clay, organic matter and soil depth data were acquired from the ISRIC-SoilGrids web portal (ftp://ftp.soilgrids.org/data/aggregated/10km/), and resampled to 0.5$^{\circ}$ to match the meteorological data resolution.



\subsection{Greenness forcing data: EVI, FPAR, FPARitpl}

\subsection{FLUXNET 2015 data, unit conversions}
\subsubsection{PPFD}
The daily photosynthetically active photon flux density (PPFD) is calculated from shortwave downwelling radiation as 
\begin{equation}
    \text{PPFD} = 2.04 \cdot R_{\text{SW}} 
\end{equation}
$R_{\text{SW}}$ is provided by WATCH-WFDEI in units of W m$^{-2}$. The factor 2.04 (units: $\mu$mol J$^{-1}$) is adopted from SPLASH (Meek et al., 1984, see Davis et al., 2017). PPFD is converted to units of mol m$^{-2}$ d$^{-1}$ by multiplication with $(10^{-6} \cdot 60 \cdot 60 \cdot 24)$.

\section{Evaluation data description}

\subsection{GPP target data uncertainty: DT, NT, Ty}

\subsection{Site selection}

\subsection{Data filtering}


\section{Simulation protocol}

Multiple calibration setups, multiple evaluations

\section{Evaluation results}

We evaluate the performance of the model in capturing multiple aspects of variability (mean across multi-day periods, spatial (across sites), inter-annual, seasonal, and daily anomalies from the mean seasonal cycle). In addition, we investigate how the greenness forcing data affects results, and how uncertainty in different methods to derive "observed" GPP affects the model evaluation.

\begin{table}
\centering
\begin{tabular}{lllllll}
  \hline
  Setup & xdaily & spatial & annual & seasonal & daily\_var \\ 
  \hline
  ORG & 0.65 & 0.58 & 0.57 & 0.67 & 0.23 \\ 
  BRC & 0.68 & 0.61 & 0.60 & 0.71 & 0.23 \\ 
  FULL & 0.69 & 0.63 & 0.66 & 0.70 & 0.25 \\ 
  NULL & 0.65 & 0.59 & 0.55 & 0.69 & 0.22 \\ 
  \end{tabular}
\caption{$R^2$ of different models and for different components of variability} 
\label{tab:rsq}
\end{table}

\begin{table}
\centering
\begin{tabular}{lllllll}
  \hline
  Setup & xdaily & spatial & annual & seasonal & daily\_var \\ 
  \hline
  ORG & 2.29 & 483 & 461 & 2.00 & 1.69 \\ 
  BRC & 2.18 & 458 & 440 & 1.88 & 1.69 \\ 
  FULL & 2.14 & 445 & 407 & 1.91 & 1.70 \\ 
  NULL & 2.28 & 476 & 478 & 1.94 & 1.66 \\ 
  \end{tabular}
\caption{RMSE of different models and for different components of variability} 
\end{table}


\subsection{X-daily values}

Performance of the model is improved from ORG ($R^2=0.65$, RMSE$=2.29$), to BRC ($R^2=0.68$, RMSE$=2.18$), and to FULL ($R^2=0.69$, RMSE$=2.14$). Model versions BRC and FULL (slightly) outperform the NULL model ($R^2=0.65$, RMSE$=2.28$).

\begin{figure}[!ht]
    \centering
    \includegraphics[width=0.7\textwidth]{fig/modobs_xdaily_FULL.pdf}
    \includegraphics[width=0.7\textwidth]{fig/modobs_xdaily_NULL.pdf}
    \caption{Correlation of observed vs. modelled GPP values of all sites pooled, mean over 10-day periods.}
    \label{fig:modobs_xdaily}
\end{figure}

\clearpage

\subsection{Spatial/annual}

Both the temperature-dependence of $\varphi_0$ (simulation BRC, $R^2=0.61$, RMSE$=458$) and the soil moisture stress function (simulation FULL, $R^2=0.63$, RMSE$=445$) improve the spatial correlation compared to the original (ORG, $R^2=0.58$, RMSE$=483$). Models BRC and FULL outperform the NULL model ($R^2=0.59$, RMSE$=476$). The spatial (means by site) explains most of the variability in annual values (black lines in Fig. \ref{fig:modobs_spatialannual}), while inter-annual variability is poorly captured by all models (see Tabs. \ref{tab:rsq} and \ref{tab:rmse}).

\begin{figure}[!ht]
    \centering
    \includegraphics[width=0.7\textwidth]{fig/modobs_spatial_annual_FULL.pdf}
    \includegraphics[width=0.7\textwidth]{fig/modobs_spatial_annual_NULL.pdf}
    \caption{}
    \label{fig:modobs_spatialannual}
\end{figure}

\clearpage

\subsection{Seasonality}
 
 We show only figures for climate zones, where more than four sites' data was available.
 
 by climate zone, FULL/ORG/BRC, NULL
 
\begin{table}
\centering
\begin{tabular}{lllrl}
  \hline
  Code & Hemisphere & N & Description \\ 
  \hline
  Cfb & north & 32 & Warm temperate fully humid with warm summer \\ 
  Dfc & north & 25 & Snow fully humid cool summer \\ 
  Csa & north & 20 & Warm temperate with dry \\ 
  Dfb & north & 18 & Snow fully humid warm summer \\ 
  Cfa & north & 12 & Warm temperate fully humid with hot summer \\ 
  ET & north & 8 & Polar tundra \\ 
  Aw & south & 7 & Equatorial savannah with dry winter \\ 
  BSk & north & 5 & Arid Steppe cold \\ 
  Cfb & south & 5 & Warm temperate fully humid with warm summer \\ 
   Csb & north & 5 & Warm temperate with dry \\ 
   Dfa & north & 4 & Snow with fully humid hot summer \\ 
   BSh & south & 3 & Arid Steppe hot \\ 
   Bwk & south & 3 &  \\ 
   Bsh & north & 2 &  \\ 
   BSk & south & 2 & Arid Steppe cold \\ 
   BWh & north & 2 & Arid desert hot \\ 
   Csb & south & 2 & Warm temperate with dry \\ 
   Dwb & north & 2 & Snow dry winter warm summer \\ 
   Af & north & 1 & Equatorial rainforest \\ 
   Am & south & 1 & Equatorial monsoon \\ 
   Bsh & south & 1 &  \\ 
   Bwh & south & 1 &  \\ 
   BWh & south & 1 & Arid desert hot \\ 
   Cfa & south & 1 & Warm temperate fully humid with hot summer \\ 
   Csa & south & 1 & Warm temperate with dry \\ 
   Cwb & north & 1 & Warm temperate with dry winter and warm summer \\ 
   \hline
  \end{tabular}
\caption{Description of Koeppen-Geiger climate zones and number of sites per climate zone } 
\end{table}

 \begin{figure}[!ht]
    \centering
\includegraphics[width=0.4\textwidth]{fig/meandoy_byzone_Awsouth_all.pdf}
\includegraphics[width=0.4\textwidth]{fig/meandoy_byzone_BSknorth_all.pdf}\\
\includegraphics[width=0.4\textwidth]{fig/meandoy_byzone_Cfanorth_all.pdf}
\includegraphics[width=0.4\textwidth]{fig/meandoy_byzone_Cfbnorth_all.pdf}\\
\includegraphics[width=0.4\textwidth]{fig/meandoy_byzone_Cfbsouth_all.pdf}
\includegraphics[width=0.4\textwidth]{fig/meandoy_byzone_Csanorth_all.pdf}
    \caption{}
    \label{fig:modobs_spatialannual}
\end{figure}

 \begin{figure}[!ht]
    \centering
\includegraphics[width=0.4\textwidth]{fig/meandoy_byzone_Csbnorth_all.pdf}
\includegraphics[width=0.4\textwidth]{fig/meandoy_byzone_Dfbnorth_all.pdf}\\
\includegraphics[width=0.4\textwidth]{fig/meandoy_byzone_Dfcnorth_all.pdf}
\includegraphics[width=0.4\textwidth]{fig/meandoy_byzone_ETnorth_all.pdf}
    \caption{}
    \label{fig:modobs_spatialannual}
\end{figure}



\subsection{Daily anomalies}
scatter and histogram: FULL, NULL

\subsection{Greenness data}
: EVI, FPAR, FPARitp
spatial/annual
Seasonality for subset of zones (boreal, temperate, arid)

\subsection{GPP target data}
Seasonality

\subsection{Overview statistics}
R2 table, RMSE table

\subsection{Functional relationships}


\subsection{Drought response}


% \subsubsection{VPD}

% Vapour pressure deficit ($D$) is calculated from vapour pressure (CRU) or specific humidity (WATCH-WFDEI) input data. In general, $D$ is the difference between actual and saturation vapour pressure:
% \begin{equation}
%     D = e_a - e_s
% \end{equation}
% We calculate saturation vapour pressure ($e_s$, in Pa) following Allen et al. (2005) as a function of temperature  as
% \begin{equation}
% e_s = 611 \; \exp \left( {\frac{17.27\;T_C}{T_C+237.3}} \right)
% \end{equation}
% $T_C$ is the temperature expressed in units of degrees Celsius. Note that Allen et al. (2005) use 6.108 instead of 6.11. The Python implementation uses daily maximum and daily minimum temperature, which is equivalent to above formulation with $T=(T_{\text{min}}+T_{\text{max}})/2$. $e_a$ is provided by CRU as an input dataset. WATCH-WFDEI provides specific humidity data ($q$), which can be converted to the mass mixing ratio of water vapor to dry air ($w$) (dimensionless) by
% \begin{equation}
%     w = \frac{q}{1-q}
% \end{equation}
% and finally to actual vapour pressure by
% \begin{equation}
%     e_a = P \frac{w R_v}{R_d + w R_v}
% \end{equation}
% where $P$ is the atmospheric pressure (Pa), $R_v$ is the specific gas constant for water vapour and $R_d$ is the specific gas constant for dry air. The specific gas constants are calculated from the universal gas constant $R$ and the molecular mass $M$ as $R_{\text{specific}}=R/M$. ($R$ is 8.3143 J mol$^{-1}$ K$^{-1}$. The molecular mass of dry air is $M_d=$ 28.963 g mol$^{-1}$ and the molecular mass of water vapor is $M_v=$ 18.02 g mol$^{-1}$.) Atmospheric pressure is assumed to be at standard conditions (101325 Pa) corrected for local elevation (barometric formula adopted from SPLASH, Eq. 20 in Davis et al., 2017).




% \begin{table}
% \centering
% \begin{tabular}{ p{1.2cm} p{2.5cm} p{2.5cm} p{4cm} l }
% \multicolumn{5}{l}{\textbf{ WATCH-WFDEI }} \\
% \hline
% \textbf{symbol} & \textbf{variable name} & \textbf{file name} & \textbf{variable description} & \textbf{units} \\
% \hline
% $T$ & \texttt{Tair} & \texttt{Tair\_WFDEI} & 2 m instantaneous air temperature & K \\
% $P_{\text{rain}}$    & \texttt{Rainf}  & \texttt{Rainf\_WFDEI\_CRU} & Rainfall rate, bias corrected with CRU TS3.101 data (TS3.21 for 2010-2012) and gauge ``catch corrected'' (average over previous 3 hrs) & kg m$^{-2}$ s$^{-1}$ \\  
% $P_{\text{snow}}$    & \texttt{Snowf} & \texttt{Snowf\_WFDEI\_CRU} & Snowfall rate, bias corrected with CRU TS3.101 data (TS3.21 for 2010-2012) and gauge ``catch corrected'' (average over previous 3 hrs)  & kg m$^{-2}$ s$^{-1}$ \\  
% $R_{\text{SW}}$    & \texttt{SWdown} & \texttt{SWdown\_WFDEI} & Short-wave downwards surface radiation flux (average over previous 3 hours) & W m$^{-2}$ \\  
% $q$    & \texttt{Qair} & \texttt{Qair\_WFDEI} & 2 m instantaneous specific humidity & kg kg$^{-1}$ \\  
% $p$    & \texttt{PSurf} & \texttt{PSurf\_WFDEI} & Instantaneous surface pressure & Pa \\ 
% \hline
% \end{tabular}
% \caption{Variables used from WATCH-WFDEI meteorological data.}
% \label{tab:meteovars}
% \end{table}


\subsection{Green vegetation cover}


\section{Evaluation against FLUXNET data}

Several performance metrics are calculated for different features of GPP variability. The performance metrics are:
\begin{itemize}
    \item R$^2$
    \item RMSE
    \item slope (of regression observed over modelled)
    \item bias
\end{itemize}

The features of variability in GPP, for which model-observation agreement is calculated, are:
\begin{itemize}
    \item mean annual values (giving "spatial" correlation)
    \item annual anomalies from mean across years
    \item daily values, absolute
    \item mean across X-day periods, absolute
    \item mean seasonal cycle (mean by day of year)
    \item daily anomalies from mean seasonal cycle
\end{itemize}

\clearpage

\subsection{Spatial correlation}

\begin{figure}[!ht]
    \centering
    \includegraphics[width=0.8\textwidth]{fig/modobs_spatial.pdf}
    \caption{Correlation of observed vs. modelled mean annual GPP by site.}
    \label{fig:modobs_spatial}
\end{figure}

\clearpage


\subsection{Annual GPP anomalies}

\begin{figure}[!ht]
    \centering
    \includegraphics[width=0.8\textwidth]{fig/modobs_anomalies_annual.pdf}
    \caption{Correlation of observed vs. modelled annual GPP anomalies from mean across multiple years.}
    \label{fig:modobs_iav}
\end{figure}

\clearpage


\subsection{Combined spatial/annual}

\begin{figure}[!ht]
    \centering
    \includegraphics[width=0.8\textwidth]{fig/modobs_spatial_annual.pdf}
    \caption{Spatial and annual correlation of observed over modelled annual GPP by site.}
    \label{fig:modobs_spatialannual}
\end{figure}

\clearpage


\section{Sites and data processing}

\begin{longtable}{lrrlllrl}
  \hline
 Site & Lon. & Lat. & Period & Veg. & Clim. & N & Reference \\ 
  \hline
 AR-SLu & -66.46 & -33.46 & 2009-2011 & MF & Bwk & 430 & \cite{AR-SLu} \\ 
 AR-Vir & -56.19 & -28.24 & 2009-2012 & ENF & Csb & 585 & \cite{AR-Vir} \\ 
 AT-Neu & 11.32 & 47.12 & 2002-2012 & GRA & Dfc & 3172 & \cite{AT-Neu} \\ 
 AU-Ade & 131.12 & -13.08 & 2007-2009 & WSA & Aw & 528 & \cite{AU-Ade} \\ 
 AU-ASM & 133.25 & -22.28 & 2010-2013 & ENF & BSh & 863 & \cite{AU-ASM} \\ 
 AU-Cpr & 140.59 & -34.00 & 2010-2014 & SAV & BSk & 1346 & \cite{AU-Cpr} \\ 
 AU-Cum & 150.72 & -33.61 & 2012-2014 & EBF & Cfa & 727 & \cite{AU-Cum} \\ 
 AU-DaP & 131.32 & -14.06 & 2007-2013 & GRA & Aw & 1377 & \cite{AU-DaP} \\ 
 AU-DaS & 131.39 & -14.16 & 2008-2014 & SAV & Aw & 2148 & \cite{AU-DaS} \\ 
 AU-Dry & 132.37 & -15.26 & 2008-2014 & SAV & Aw & 1579 & \cite{AU-Dry} \\ 
 AU-Emr & 148.47 & -23.86 & 2011-2013 & GRA & Bwk & 674 & \cite{AU-Emr} \\ 
 AU-Fog & 131.31 & -12.55 & 2006-2008 & WET & Aw & 866 & \cite{AU-Fog} \\ 
 AU-Gin & 115.71 & -31.38 & 2011-2014 & WSA & Cwb & 935 & \cite{AU-Gin} \\ 
 AU-GWW & 120.65 & -30.19 & 2013-2014 & SAV & Bwk & 646 & \cite{AU-GWW} \\ 
 AU-How & 131.15 & -12.49 & 2001-2014 & WSA & Aw &  & \cite{AU-How} \\ 
 AU-Lox & 140.66 & -34.47 & 2008-2009 & DBF & Bsh & 271 & \cite{AU-Lox} \\ 
 AU-RDF & 132.48 & -14.56 & 2011-2013 & WSA & Bwh & 424 & \cite{AU-RDF} \\ 
 AU-Rig & 145.58 & -36.65 & 2011-2014 & GRA & Cfb & 1116 & \cite{AU-Rig} \\ 
 AU-Rob & 145.63 & -17.12 & 2014-2014 & EBF & Csb & 314 & \cite{AU-Rob} \\ 
 AU-Stp & 133.35 & -17.15 & 2008-2014 & GRA & BSh & 1314 & \cite{AU-Stp} \\ 
 AU-TTE & 133.64 & -22.29 & 2012-2013 & OSH & BWh &  89 & \cite{AU-TTE} \\ 
 AU-Tum & 148.15 & -35.66 & 2001-2014 & EBF & Cfb & 4176 & \cite{AU-Tum} \\ 
 AU-Wac & 145.19 & -37.43 & 2005-2008 & EBF & Cfb & 968 & \cite{AU-Wac} \\ 
 AU-Whr & 145.03 & -36.67 & 2011-2014 & EBF & Cfb & 1062 & \cite{AU-Whr} \\ 
 AU-Wom & 144.09 & -37.42 & 2010-2012 & EBF & Cfb & 897 & \cite{AU-Wom} \\ 
 AU-Ync & 146.29 & -34.99 & 2012-2014 & GRA & BSk & 336 & \cite{AU-Ync} \\ 
 BE-Bra & 4.52 & 51.31 & 1996-2014 & MF & Cfb & 4458 & \cite{BE-Bra} \\ 
 BE-Lon & 4.75 & 50.55 & 2004-2014 & CRO & Cfb & 2834 & \cite{BE-Lon} \\ 
 BE-Vie & 6.00 & 50.31 & 1996-2014 & MF & Cfb & 5556 & \cite{BE-Vie} \\ 
 BR-Sa3 & -54.97 & -3.02 & 2000-2004 & EBF & Am & 1122 & \cite{BR-Sa3} \\ 
 CA-Man & -98.48 & 55.88 & 1994-2008 & ENF & Dfc & 2435 & \cite{CA-Man} \\ 
 CA-NS1 & -98.48 & 55.88 & 2001-2005 & ENF & Dfc & 763 & \cite{CA-NS1} \\ 
 CA-NS2 & -98.52 & 55.91 & 2001-2005 & ENF & Dfc & 859 & \cite{CA-NS2} \\ 
 CA-NS3 & -98.38 & 55.91 & 2001-2005 & ENF & Dfc & 1062 & \cite{CA-NS3} \\ 
 CA-NS4 & -98.38 & 55.91 & 2002-2005 & ENF & Dfc & 601 & \cite{CA-NS4} \\ 
 CA-NS5 & -98.48 & 55.86 & 2001-2005 & ENF & Dfc & 903 & \cite{CA-NS5} \\ 
 CA-NS6 & -98.96 & 55.92 & 2001-2005 & OSH & Dfc & 904 & \cite{CA-NS6} \\ 
 CA-NS7 & -99.95 & 56.64 & 2002-2005 & OSH & Dfc & 693 & \cite{CA-NS7} \\ 
 CA-Qfo & -74.34 & 49.69 & 2003-2010 & ENF & Dfc & 1795 & \cite{CA-Qfo} \\ 
 CA-SF1 & -105.82 & 54.48 & 2003-2006 & ENF & Dfc & 513 & \cite{CA-SF1} \\ 
 CA-SF2 & -105.88 & 54.25 & 2001-2005 & ENF & Dfc & 664 & \cite{CA-SF2} \\ 
 CA-SF3 & -106.01 & 54.09 & 2001-2006 & OSH & Dfc & 634 & \cite{CA-SF3} \\ 
 CH-Cha & 8.41 & 47.21 & 2005-2014 & GRA & Cfb & 2872 & \cite{CH-Cha} \\ 
 CH-Dav & 9.86 & 46.82 & 1997-2014 & ENF & ET & 5293 & \cite{CH-Dav} \\ 
 CH-Fru & 8.54 & 47.12 & 2005-2014 & GRA & Cfb & 2527 & \cite{CH-Fru} \\ 
 CH-Lae & 8.37 & 47.48 & 2004-2014 & MF & Cfb & 3144 & \cite{CH-Lae} \\ 
 CH-Oe1 & 7.73 & 47.29 & 2002-2008 & GRA & Cfb & 2083 & \cite{CH-Oe1} \\ 
 CH-Oe2 & 7.73 & 47.29 & 2004-2014 & CRO & Cfb & 2990 & \cite{CH-Oe2} \\ 
 CN-Cha & 128.10 & 42.40 & 2003-2005 & MF & Dwb & 804 & \cite{CN-Cha} \\ 
 CN-Cng & 123.51 & 44.59 & 2007-2010 & GRA & Bsh & 1026 & \cite{CN-Cng} \\ 
 CN-Dan & 91.07 & 30.50 & 2004-2005 & GRA & ET & 619 & \cite{CN-Dan} \\ 
 CN-Din & 112.54 & 23.17 & 2003-2005 & EBF & Cfa & 894 & \cite{CN-Din} \\ 
 CN-Du2 & 116.28 & 42.05 & 2006-2008 & GRA & Dwb & 520 & \cite{CN-Du2} \\ 
 CN-Ha2 & 101.33 & 37.61 & 2003-2005 & WET & ET & 886 & \cite{CN-Ha2} \\ 
 CN-HaM & 101.18 & 37.37 & 2002-2004 & GRA &  & 664 & \cite{CN-HaM} \\ 
 CN-Qia & 115.06 & 26.74 & 2003-2005 & ENF & Cfa & 987 & \cite{CN-Qia} \\ 
 CN-Sw2 & 111.90 & 41.79 & 2010-2012 & GRA & Bsh & 202 & \cite{CN-Sw2} \\ 
 CZ-BK1 & 18.54 & 49.50 & 2004-2008 & ENF & Dfb & 1051 & \cite{CZ-BK1} \\ 
 CZ-BK2 & 18.54 & 49.49 & 2004-2006 & GRA & Dfb & 156 & \cite{CZ-BK2} \\ 
 CZ-wet & 14.77 & 49.02 & 2006-2014 & WET & Cfb & 2592 & \cite{CZ-wet} \\ 
 DE-Akm & 13.68 & 53.87 & 2009-2014 & WET & Cfb &  & \cite{DE-Akm} \\ 
 DE-Geb & 10.91 & 51.10 & 2001-2014 & CRO & Cfb & 3454 & \cite{DE-Geb} \\ 
 DE-Gri & 13.51 & 50.95 & 2004-2014 & GRA & Cfb & 3349 & \cite{DE-Gri} \\ 
 DE-Hai & 10.45 & 51.08 & 2000-2012 & DBF & Cfb & 3436 & \cite{DE-Hai} \\ 
 DE-Kli & 13.52 & 50.89 & 2004-2014 & CRO & Cfb &  & \cite{DE-Kli} \\ 
 DE-Lkb & 13.30 & 49.10 & 2009-2013 & ENF & Cfb & 846 & \cite{DE-Lkb} \\ 
 DE-Obe & 13.72 & 50.78 & 2008-2014 & ENF & Cfb & 1992 & \cite{DE-Obe} \\ 
 DE-RuR & 6.30 & 50.62 & 2011-2014 & GRA & Cfb & 1178 & \cite{DE-RuR} \\ 
 DE-RuS & 6.45 & 50.87 & 2011-2014 & CRO & Cfb &  & \cite{DE-RuS} \\ 
 DE-Seh & 6.45 & 50.87 & 2007-2010 & CRO & Cfb & 1026 & \cite{DE-Seh} \\ 
 DE-SfN & 11.33 & 47.81 & 2012-2014 & WET & Cfb & 737 & \cite{DE-SfN} \\ 
 DE-Spw & 14.03 & 51.89 & 2010-2014 & WET & Cfb & 1323 & \cite{DE-Spw} \\ 
 DE-Tha & 13.57 & 50.96 & 1996-2014 & ENF & Cfb & 5964 & \cite{DE-Tha} \\ 
 DK-Fou & 9.59 & 56.48 & 2005-2005 & CRO & Cfb & 212 & \cite{DK-Fou} \\ 
 DK-NuF & -51.39 & 64.13 & 2008-2014 & WET & ET & 869 & \cite{DK-NuF} \\ 
 DK-Sor & 11.64 & 55.49 & 1996-2014 & DBF & Cfb & 5514 & \cite{DK-Sor} \\ 
 DK-ZaF & -20.55 & 74.48 & 2008-2011 & WET & ET & 328 & \cite{DK-ZaF} \\ 
 DK-ZaH & -20.55 & 74.47 & 2000-2014 & GRA & ET & 1648 & \cite{DK-ZaH} \\ 
 ES-LgS & -2.97 & 37.10 & 2007-2009 & OSH & Cwc & 766 & \cite{ES-LgS} \\ 
 ES-Ln2 & -3.48 & 36.97 & 2009-2009 & OSH & Cwc &  61 & \cite{ES-Ln2} \\ 
 FI-Hyy & 24.30 & 61.85 & 1996-2014 & ENF & Dfc & 5204 & \cite{FI-Hyy} \\ 
 FI-Jok & 23.51 & 60.90 & 2000-2003 & CRO & Dfc & 595 & \cite{FI-Jok} \\ 
 FI-Lom & 24.21 & 68.00 & 2007-2009 & WET & Dfc & 430 & \cite{FI-Lom} \\ 
 FI-Sod & 26.64 & 67.36 & 2001-2014 & ENF & Dfc & 2634 & \cite{FI-Sod} \\ 
 FR-Fon & 2.78 & 48.48 & 2005-2014 & DBF & Cfb & 2821 & \cite{FR-Fon} \\ 
 FR-Gri & 1.95 & 48.84 & 2004-2013 & CRO & Cfb &  & \cite{FR-Gri} \\ 
 FR-LBr & -0.77 & 44.72 & 1996-2008 & ENF & Cfb & 3527 & \cite{FR-LBr} \\ 
 FR-Pue & 3.60 & 43.74 & 2000-2014 & EBF & Cwb & 4650 & \cite{FR-Pue} \\ 
 GF-Guy & -52.92 & 5.28 & 2004-2014 & EBF & Am & 3583 & \cite{GF-Guy} \\ 
 IT-BCi & 14.96 & 40.52 & 2004-2014 & CRO & Cwb &  & \cite{IT-BCi} \\ 
 IT-CA1 & 12.03 & 42.38 & 2011-2014 & DBF & Cwb & 991 & \cite{IT-CA1} \\ 
 IT-CA2 & 12.03 & 42.38 & 2011-2014 & CRO & Cwb & 959 & \cite{IT-CA2} \\ 
 IT-CA3 & 12.02 & 42.38 & 2011-2014 & DBF & Cwb & 813 & \cite{IT-CA3} \\ 
 IT-Col & 13.59 & 41.85 & 1996-2014 & DBF & Cfa & 3276 & \cite{IT-Col} \\ 
 IT-Cp2 & 12.36 & 41.70 & 2012-2014 & EBF & Cwb & 756 & \cite{IT-Cp2} \\ 
 IT-Cpz & 12.38 & 41.71 & 1997-2009 & EBF & Cwb & 2536 & \cite{IT-Cpz} \\ 
 IT-Isp & 8.63 & 45.81 & 2013-2014 & DBF & Cfb & 557 & \cite{IT-Isp} \\ 
 IT-La2 & 11.29 & 45.95 & 2000-2002 & ENF & Cfb & 470 & \cite{IT-La2} \\ 
 IT-Lav & 11.28 & 45.96 & 2003-2014 & ENF & Cfb & 3799 & \cite{IT-Lav} \\ 
 IT-MBo & 11.05 & 46.01 & 2003-2013 & GRA & Dfb & 3203 & \cite{IT-MBo} \\ 
 IT-Noe & 8.15 & 40.61 & 2004-2014 & CSH & Cwb & 3013 & \cite{IT-Noe} \\ 
 IT-PT1 & 9.06 & 45.20 & 2002-2004 & DBF & Cfa & 813 & \cite{IT-PT1} \\ 
 IT-Ren & 11.43 & 46.59 & 1998-2013 & ENF & Dfc & 3180 & \cite{IT-Ren} \\ 
 IT-Ro1 & 11.93 & 42.41 & 2000-2008 & DBF & Cwb &  & \cite{IT-Ro1} \\ 
 IT-Ro2 & 11.92 & 42.39 & 2002-2012 & DBF & Cwb & 2641 & \cite{IT-Ro2} \\ 
 IT-SR2 & 10.29 & 43.73 & 2013-2014 & ENF & Cwb & 658 & \cite{IT-SR2} \\ 
 IT-SRo & 10.28 & 43.73 & 1999-2012 & ENF & Cwb & 4021 & \cite{IT-SRo} \\ 
 IT-Tor & 7.58 & 45.84 & 2008-2014 & GRA & Dfc & 1351 & \cite{IT-Tor} \\ 
 JP-MBF & 142.32 & 44.39 & 2003-2005 & DBF & Dfb & 459 & \cite{JP-MBF} \\ 
 JP-SMF & 137.08 & 35.26 & 2002-2006 & MF & Cfa & 1272 & \cite{JP-SMF} \\ 
 NL-Hor & 5.07 & 52.24 & 2004-2011 & GRA & Cfb & 2113 & \cite{NL-Hor} \\ 
 NL-Loo & 5.74 & 52.17 & 1996-2013 & ENF & Cfb & 5417 & \cite{NL-Loo} \\ 
 NO-Adv & 15.92 & 78.19 & 2011-2014 & WET & ET &  90 & \cite{NO-Adv} \\ 
 NO-Blv & 11.83 & 78.92 & 2008-2009 & SNO & ET &  60 & \cite{NO-Blv} \\ 
 RU-Che & 161.34 & 68.61 & 2002-2005 & WET & Dfc & 277 & \cite{RU-Che} \\ 
 RU-Cok & 147.49 & 70.83 & 2003-2014 & OSH & Dfc & 962 & \cite{RU-Cok} \\ 
 RU-Fyo & 32.92 & 56.46 & 1998-2014 & ENF & Dfb & 4397 & \cite{RU-Fyo} \\ 
 RU-Ha1 & 90.00 & 54.73 & 2002-2004 & GRA & Dfc & 516 & \cite{RU-Ha1} \\ 
 SD-Dem & 30.48 & 13.28 & 2005-2009 & SAV & BWh & 735 & \cite{SD-Dem} \\ 
 SN-Dhr & -15.43 & 15.40 & 2010-2013 & SAV & BWh & 631 & \cite{SN-Dhr} \\ 
 US-AR1 & -99.42 & 36.43 & 2009-2012 & GRA & Cfa & 925 & \cite{US-AR1} \\ 
 US-AR2 & -99.60 & 36.64 & 2009-2012 & GRA & Cfa & 810 & \cite{US-AR2} \\ 
 US-ARb & -98.04 & 35.55 & 2005-2006 & GRA & Cfa & 408 & \cite{US-ARb} \\ 
 US-ARc & -98.04 & 35.55 & 2005-2006 & GRA & Cfa & 478 & \cite{US-ARc} \\ 
 US-ARM & -97.49 & 36.61 & 2003-2012 & CRO & Cfa & 2306 & \cite{US-ARM} \\ 
 US-Blo & -120.63 & 38.90 & 1997-2007 & ENF & Cwc & 2227 & \cite{US-Blo} \\ 
 US-Cop & -109.39 & 38.09 & 2001-2007 & GRA & BSk & 933 & \cite{US-Cop} \\ 
 US-GBT & -106.24 & 41.37 & 1999-2006 & ENF & Dfc & 533 & \cite{US-GBT} \\ 
 US-GLE & -106.24 & 41.37 & 2004-2014 & ENF & Dfb & 2146 & \cite{US-GLE} \\ 
 US-Ha1 & -72.17 & 42.54 & 1991-2012 & DBF & Dfb & 4810 & \cite{US-Ha1} \\ 
 US-KS2 & -80.67 & 28.61 & 2003-2006 & CSH & Cfa & 1255 & \cite{US-KS2} \\ 
 US-Los & -89.98 & 46.08 & 2000-2014 & WET & Dfb & 2040 & \cite{US-Los} \\ 
 US-Me1 & -121.50 & 44.58 & 2004-2005 & ENF & Cwc & 260 & \cite{US-Me1} \\ 
 US-Me2 & -121.56 & 44.45 & 2002-2014 & ENF & Cwc & 3399 & \cite{US-Me2} \\ 
 US-Me6 & -121.61 & 44.32 & 2010-2014 & ENF & Cwc & 1243 & \cite{US-Me6} \\ 
 US-MMS & -86.41 & 39.32 & 1999-2014 & DBF & Cfa & 3610 & \cite{US-MMS} \\ 
 US-Myb & -121.77 & 38.05 & 2010-2014 & WET & Cwb & 1111 & \cite{US-Myb} \\ 
 US-Ne1 & -96.48 & 41.17 & 2001-2013 & CRO & Dfa &  & \cite{US-Ne1} \\ 
 US-Ne2 & -96.47 & 41.16 & 2001-2013 & CRO & Dfa &  & \cite{US-Ne2} \\ 
 US-Ne3 & -96.44 & 41.18 & 2001-2013 & CRO & Dfa &  & \cite{US-Ne3} \\ 
 US-NR1 & -105.55 & 40.03 & 1998-2014 & ENF & Dfc & 4205 & \cite{US-NR1} \\ 
 US-ORv & -83.02 & 40.02 & 2011-2011 & WET & Dfa &  & \cite{US-ORv} \\ 
 US-PFa & -90.27 & 45.95 & 1995-2014 & MF & Dfb & 4590 & \cite{US-PFa} \\ 
 US-Prr & -147.49 & 65.12 & 2010-2013 & ENF & Dfc & 515 & \cite{US-Prr} \\ 
 US-SRG & -110.83 & 31.79 & 2008-2014 & GRA & BSk & 1874 & \cite{US-SRG} \\ 
 US-SRM & -110.87 & 31.82 & 2004-2014 & WSA & BSk & 2759 & \cite{US-SRM} \\ 
 US-Syv & -89.35 & 46.24 & 2001-2014 & MF & Dfb & 1977 & \cite{US-Syv} \\ 
 US-Ton & -120.97 & 38.43 & 2001-2014 & WSA & Cwb & 3981 & \cite{US-Ton} \\ 
 US-Tw1 & -121.65 & 38.11 & 2012-2014 & WET & Cwb & 554 & \cite{US-Tw1} \\ 
 US-Tw2 & -121.64 & 38.10 & 2012-2013 & CRO & Cwb & 263 & \cite{US-Tw2} \\ 
 US-Tw3 & -121.65 & 38.12 & 2013-2014 & CRO & Cwb & 419 & \cite{US-Tw3} \\ 
 US-Tw4 & -121.64 & 38.10 & 2013-2014 & WET & Cwb & 321 & \cite{US-Tw4} \\ 
 US-Twt & -121.65 & 38.11 & 2009-2014 & CRO & Cwb & 1421 & \cite{US-Twt} \\ 
 US-UMB & -84.71 & 45.56 & 2000-2014 & DBF & Dfb & 3970 & \cite{US-UMB} \\ 
 US-UMd & -84.70 & 45.56 & 2007-2014 & DBF & Dfb & 2034 & \cite{US-UMd} \\ 
 US-Var & -120.95 & 38.41 & 2000-2014 & GRA & Cwb & 2931 & \cite{US-Var} \\ 
 US-WCr & -90.08 & 45.81 & 1999-2014 & DBF & Dfb & 2485 & \cite{US-WCr} \\ 
 US-Whs & -110.05 & 31.74 & 2007-2014 & OSH & BSk & 1452 & \cite{US-Whs} \\ 
 US-Wi0 & -91.08 & 46.62 & 2002-2002 & ENF & Dfb & 175 & \cite{US-Wi0} \\ 
 US-Wi3 & -91.10 & 46.63 & 2002-2004 & DBF & Dfb & 353 & \cite{US-Wi3} \\ 
 US-Wi4 & -91.17 & 46.74 & 2002-2005 & ENF & Dfb & 568 & \cite{US-Wi4} \\ 
 US-Wi6 & -91.30 & 46.62 & 2002-2003 & OSH & Dfb & 175 & \cite{US-Wi6} \\ 
 US-Wi9 & -91.08 & 46.62 & 2004-2005 & ENF & Dfb & 290 & \cite{US-Wi9} \\ 
 US-Wkg & -109.94 & 31.74 & 2004-2014 & GRA & BSk & 2373 & \cite{US-Wkg} \\ 
 ZA-Kru & 31.50 & -25.02 & 2000-2010 & SAV & BSh & 1890 & \cite{ZA-Kru} \\ 
 ZM-Mon & 23.25 & -15.44 & 2000-2009 & DBF & Aw & 625 & \cite{ZM-Mon} \\ 
  \hline
\caption{Sites used for evaluation. Lon. is longitude, negative values indicate west longitude; Lat. is latitude, positive values indicate north latitude; Veg. is vegetation type: deciduous broadleaf forest (DBF); evergreen broadleaf forest (EBF); evergreen needleleaf forest (ENF); grassland (GRA); mixed deciduous and evergreen needleleaf forest (MF); savanna ecosystem (SAV); shrub ecosystem (SHR); wetland (WET).} 
\end{longtable}



%% BELOW IS ORIGINAL BY DAVID (18.10.2017)

% To obtain the soil moisture capacity in mm, we used the soil depth multiplied by the water holding capacity, then, the result units were converted to mm equivalents, litres per square meter.
% \begin{equation}
%     W_{m}\left [ mm \right ]= WHC\left [ m_{H_{2}O}^{3} \cdot m_{Soil}^{-3} \right ]\cdot SD\left [ m_{Soil} \right ]\cdot 10^{-3}\left [ l \cdot m_{H_{2}O}^{-3} \right ]
% \end{equation}

% Where, the water holding capacity, defined as volumetric proportion, was estimated, following Hillel (1982), as the difference between field capacity and wilting point.
% \begin{equation}
% WHC=FC-WP
% \end{equation}

% Field capacity and wilting point, were obtained from texture and organic matter content data, through pedotransfer functions, described in Saxton & Rawls (2006) as follows:
% \begin{equation}
% FC= FC_{0}+(1.283\cdot FC_{0}^{2}-0.374\cdot FC_{0}-0.015)
% \end{equation}
% Where,
% \begin{equation}
% FC_{0}=-0.251S+195C+0.011OM+0.006(S\cdot OM)-0.027(C\cdot OM)+0.452 (S\cdot C)+0.299
% \end{equation}
% And,
% \begin{equation}
% WP=WP_{0}+(0.14\cdot WP_{0}-0.02)
% \end{equation}
% Where,
% \begin{equation}
% WP_{0}=0.024S+0.487C+0.006OM+0.005 (S\cdot OM)-0.013(C\cdot OM)+0.068(S\cdot C)+0.031
% \end{equation}
% Where, $FC_{0}$ and $WP_{0}$ are the field capacity and wilting point first solutions respectively. And, S, C, OM are the sand, clay and organic matter contents.
% Contents of sand, clay, organic matter and soil depth data were acquired from the ISRIC-SoilGrids web portal (ftp://ftp.soilgrids.org/data/aggregated/10km/), and resampled to 0.5$^{\circ}$ to match the meteorological data resolution.
% \clearpage

%%%%%%%%%%%%%%%%%%%%%%%%%%%%%%%%%%%%%%%%%%%%%%%%%%%%%%%%%%%%%%%%%%%%%%%%%%%
\addcontentsline{toc}{section}{References}
\bibliographystyle{copernicus}
\bibliography{bibliography.bib}

% \begin{appendix} 
% \section{Appendix}
% \subsection{Email, Colin Prentice, Dec 31st 2011}
% \label{app:emailcprentice}
% \small{
% Dear Renato and others,\\

% }

% \end{appendix}

\end{document}

