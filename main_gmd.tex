%% Copernicus Publications Manuscript Preparation Template for LaTeX Submissions
%% ---------------------------------
%% This template should be used for copernicus.cls
%% The class file and some style files are bundled in the Copernicus Latex Package, which can be downloaded from the different journal webpages.
%% For further assistance please contact Copernicus Publications at: production@copernicus.org
%% https://publications.copernicus.org/for_authors/manuscript_preparation.html


%% Please use the following documentclass and journal abbreviations for discussion papers and final revised papers.

%% 2-column papers and discussion papers
\documentclass[gmd, manuscript]{copernicus}

%% \usepackage commands included in the copernicus.cls:
%\usepackage[german, english]{babel}
%\usepackage{tabularx}
%\usepackage{cancel}
%\usepackage{multirow}
%\usepackage{supertabular}
%\usepackage{algorithmic}
%\usepackage{algorithm}
%\usepackage{amsthm}
%\usepackage{float}
%\usepackage{subfig}
%\usepackage{rotating}

\usepackage{longtable}

\newcommand{\coo}{CO$_2$}
\newcommand{\vcmax}{$V_{\text{cmax}}$}
\newcommand{\jmax}{$J_{\text{max}}$}
\newcommand{\rsq}{$R^2$}

\begin{document}

\title{P-model v1.0: An optimality-based light use efficiency model for simulating ecosystem gross primary production}


% \Author[affil]{given_name}{surname}

\Author[1,2]{Benjamin D.}{Stocker}
\Author[3]{Han}{Wang}
\Author[4]{Nicholas G.}{Smith}
\Author[5]{Sandy P.}{Harrison}
\Author[6, 7]{Trevor F.}{Keenan}
\Author[8]{David}{Sandoval}
\Author[8,9]{Tyler}{Davis}
\Author[8]{I. Colin}{Prentice}

\affil[1]{CREAF, Campus UAB, 08193 Bellaterra, Catalonia, Spain}
\affil[2]{Earth System Science, Stanford University, Stanford, 94305-4216, California, USA}
\affil[3]{Department of Earth System Science, Tsinghua University, Haidian, Beijing, 100084, China}
\affil[4]{Department of Biological Sciences, Texas Tech University, Box 43131 Lubbock, TX 79409, USA}
\affil[5]{Geography and Environmental Science, Reading University, Reading, RG6 6 AH, UK}
\affil[6]{Earth and Environmental Sciences Area, Lawrence Berkeley National Lab, Berkeley, CA 94709, USA}
\affil[7]{Department of Environmental Science, Policy and Management, UC Berkeley, Berkeley, CA 94720, USA}
\affil[8]{AXA Chair of Biosphere and Climate Impacts, Department of Life Sciences, Imperial College London, Silwood Park Campus, Ascot, Berkshire, SL5 7PY, UK}
\affil[9]{Center for Geospatial Analysis, The College of William \& Mary, Williamsburg, VA, 23185, USA.}

%% The [] brackets identify the author with the corresponding affiliation. 1, 2, 3, etc. should be inserted.



\runningtitle{P-model v1.0}

\runningauthor{Stocker et al.}

\correspondence{B. D. Stocker (b.stocker@creaf.uab.cat)}


\received{}
\pubdiscuss{} %% only important for two-stage journals
\revised{}
\accepted{}
\published{}

%% These dates will be inserted by Copernicus Publications during the typesetting process.


\firstpage{1}

\maketitle


\begin{abstract}
 
Terrestrial photosynthesis is the basis for vegetation growth and drives the land carbon cycle. Accurately simulating gross primary production (GPP, ecosystem-level apparent photosynthesis) is key for satellite monitoring and Earth System Model predictions under climate change. While robust models exist for describing leaf-level photosynthesis, predictions diverge due to uncertain photosynthetic traits and parameters which vary on multiple spatial and temporal scales. Here, we describe and evaluate a gross primary production (GPP, photosynthesis per unit ground area) model, the P-model, that combines the Farquhar-von Caemmerer-Berry model for C$_3$ photosynthesis with an optimality principle for the carbon assimilation-transpiration trade-off, and predicts a multi-day average light use efficiency (LUE) for any climate and C$_3$ vegetation type. The model builds on the theory developed in \citet{prentice14ecollett} and \citet{wang17natpl} and is extended to include low temperature effects on the intrinsic quantum yield and an empirical soil moisture stress factor. The model is forced with site-level data of the fraction of absorbed photosynthetically active radiation (fAPAR) and meteorological data and is evaluated against GPP estimates from a globally distributed network of ecosystem flux measurements. Although the P-model requires relatively few inputs and prescribed parameters, the \rsq\ for predicted versus observed GPP based on the full model setup is 0.73 (8-day mean, 131 sites) -- similar as comparable satellite data-driven GPP models but without predefined vegetation type-specific parameters. The \rsq\ is reduced to 0.68 when not accounting for the reduction in quantum yield at low temperatures and effects of low soil moisture on LUE. The \rsq\ for the P-model-predicted LUE is 0.32 (means by site) and 0.47 (means by vegetation type). Applying this model for global-scale simulations yields a total global GPP of 105-121 GtC yr$^{-1}$ (mean of 2001-2011), depending on the fAPAR forcing data. The P-model provides a simple but powerful method for predicting -- rather than prescribing -- light use efficiency and simulating terrestrial photosynthesis across a wide range of conditions. The model is available as an R package (\textit{rpmodel}).
\end{abstract}

\copyrightstatement{}

\introduction

Realistic, reliable and robust estimates of terrestrial photosynthesis are required to understand variations in the carbon cycle, monitor forest and cropland productivity, and predict impacts of global environmental change on ecosystem function \citep{prentice15}. Understanding how photosynthetic rates depend on temperature, humidity, solar radiation, CO$_2$ and soil moisture is at the core of this challenge. Process-based Dynamic Vegetation Models (DVMs) and Earth System Models (ESMs) in use today almost always use some form of the Farquhar-von Caemmerer-Berry (FvCB) model for C$_3$ photosynthesis \citep{farquhar80, voncaemmerer81}, in combination with stomatal conductance ($g_s$) models \citep{ball87, leuning95pce, medlyn11gcb}, that couple water and carbon fluxes at the leaf surface. 

The FvCB model describes the instantaneous saturating relationship between leaf-internal CO$_2$ concentrations ($c_i$) and assimilation ($A$), and how this relationship depends on absorbed photosynthetically active radiation (APAR). It simulates $A$ as the minimum of a light-limited and a Rubisco-limited assimilation rate, $A_J$ and $A_C$ respectively:
\begin{equation}
    A = \min(A_J, A_C)
\end{equation}

Although the FvCB model is standard for leaf-scale photosynthesis, and its environmental responses at time scales of minutes to hours, DVMs and ESMs using FvCB produce divergent results for ecosystem-level fluxes and their response to environment at longer time scales \citep{rogers17}. This is due to assumptions that have to be made about photosynthetic parameters that are not predicted by the FvCB model: stomatal conductance ($g_s$) and the maximum rates of Rubisco carboxylation (\vcmax ) and electron transport (\jmax ) for ribulose-1,5-bisphosphate (RuBP) regeneration, which together determine the relationship between $c_i$ and $A$. Common approaches for determining the values of \vcmax\ and \jmax\ in DVMs and ESMs are to prescribe fixed values per plant functional type (PFT) and attempt to simulate the distribution of PFTs in space, or to use empirical relationships between leaf N and \vcmax\ and simulate leaf N internally or prescribe it per PFT \citep{smithdukes13gcb, rogers14}. 

While the FvCB model describes a non-linear relationship between instantaneous assimilation and absorbed light, ecosystem production, integrated over weeks to months, scales proportionally with absorbed photosynthetically active radiation (APAR) \citep{monteith72, medlyn98}. This observation underlies the general light use efficiency (LUE) model which describes ecosystem-level photosynthesis (gross primary production, GPP) as the product of APAR and LUE:
\begin{equation}
\label{eq:luemodel}
\text{GPP} = \text{PAR} \cdot \text{fAPAR} \cdot \text{LUE} \;,
\end{equation}
where PAR is the incident photosynthetically active radiation and fAPAR is the fraction of PAR that is absorbed by green tissue. The LUE model is the basis for observation-driven GPP models that use fAPAR and PAR based on remote sensing data and combine this with different approaches for simulating LUE \citep{running04, Zhang2017-yr, field95rse}, and for some forest growth models \citep{landsberg97fem}. Other remote sensing data-based models \citep{jiang16rse} apply the FvCB model in combination with vegetation cover and type data and prescribed \vcmax\ for a set of PFTs.

Here, we describe a model, referred to as the \textit{P-model}, that unifies the FvCB and LUE models following the theory developed by \citet{prentice14ecollett} and \citet{wang17natpl}. The model assumes an optimality principle that balances the C cost (per unit of assimilation) of maintaining transpiration and carboxylation (\vcmax ) capacities. It thus predicts how the ratio of leaf-internal to ambient CO$_2$ ($c_i:c_a = \chi$) acclimates to the environment, given temperature ($T$), water vapour pressure deficit ($D$), atmospheric pressure ($p$) and ambient CO$_2$ concentration ($c_a$)  \citep{prentice14ecollett}. The P-model also assumes that the photosynthetic machinery tends to coordinate \vcmax\ and \jmax\ in order to operate close to the intersection of the light-limited and Rubisco-limited assimilation rates (\textit{Coordination Hypothesis}, \citet{chen93, maire12po}) under mean daytime environmental conditions. By further assuming equality in the marginal cost and benefit of \jmax , daily-to-monthly average assimilation rates can then be described as fractions of absorbed PAR, i.e. as a LUE model (Eq. \ref{eq:luemodel}) \citep{wang17natpl}.  

Thus, the P-model embodies an optimality-based theory for predicting the acclimation of leaf-level photosynthesis to its environment and for simulating LUE. In combination with prescribed PAR and remotely sensed fAPAR, it estimates GPP across diverse environmental conditions \citep{wang17natpl}. Its prediction for acclimating photosynthetic parameters reduces the number of prescribed (and temporally fixed) values and avoids the distinction of model parameterisation by vegetation types or biomes (apart from a distinction between the C$_3$ and C$_4$ photosynthetic pathways). The P-model has a further advantage over other data-driven GPP models (\citep{running04, Zhang2017-yr}, and empirically upscaled GPP estimates \citep{jung11jgr} in that it accounts for the influence of changing CO$_2$, and that it uses first principles (rather than imposed functions) to represent effects of $T$, $D$ and $p$ (Sect. \ref{sec:theory}). The theory underlying the P-model regarding the water-carbon tradeoff has been described by \citet{prentice14ecollett} and applied by \citet{keenan17natcomm} to simulate how changes in primary production have driven the terrestrial C sink over past decades; and by \citet{smith19ecollett} to explain variations in observed \vcmax . \citet{wang17natpl} complemented the theory by including effects of limited electron transport capacity (\jmax ) and predicted variations in observed $\chi$ across environmental gradients. To resolve model biases under conditions of low soil moisture, \citep{stocker19natgeo} further applied an empirical stress function to reduce LUE under dry soil conditions.

The purpose of this paper is (\textit{i}) to provide a full documentation of the model implementation and reference for open-source software (\textit{rpmodel} R package, available on CRAN); (\textit{ii}) to provide an evaluation of model-predicted LUE and GPP against GPP derived from eddy covariance flux measurements (FLUXNET 2015 Tier 1 dataset); (\textit{iii}) to apply this model for global-scale simulations and compare spatial patterns and global totals of simulated GPP with other estimates with global coverage; and (\textit{iv}) to introduce a robust and pragmatic solution to resolving model bias under dry and cold conditions. With (\textit{iv}) we do not aim at extending the theoretical basis for the P-model \citep{prentice14ecollett, wang17natpl}, but to include environmental controls in the LUE model that serve to make the model applicable as a remote sensing data-driven GPP model for a wide range of conditions and vegetation types. The evaluation focuses on different components of variability (spatial, annual, seasonal, daily anomalies) (Secs. \ref{sec:results_lue} - \ref{sec:results_gpp}). We further address uncertainties associated with the fAPAR forcing (Sect. \ref{sec:results_greenness}) and the uncertainties in the evaluation data by using GPP data derived from different flux decomposition methods (Sect. \ref{sec:results_gppdata}). The use of continuous GPP measurements, rather than experimentally disturbed measurements, makes it challenging to assess modelled GPP under extreme environmental conditions. We therefore make a further evaluation of simulated GPP during the course of soil moisture drought events (\textit{fLUE droughts}, Sect. \ref{sec:results_droughtresponse}).


\section{Theory}
\label{sec:theory}

The theory underlying the P-model has been described by \citet{wang17natpl} and the derivation of equations is given therein. It is presented here again for completeness.

\subsection{Balancing carbon and water costs}
\label{sec:watercarbon}
The P-model centres around a prediction for the optimal ratio of leaf-internal to ambient \coo\ concentration $c_i:c_a$ (termed $\chi$) that balances the costs associated with maintaining the transpiration stream and the cost of maintaining a given carboxylation capacity. The optimal balance is achieved when the two marginal costs are equal: 
\begin{equation}
\label{eq:optimality_chi}
a \; \frac{\partial (E/A)}{\partial \chi} = -b \; \frac{\partial (V_{\mathrm{cmax}}/A)}{\partial \chi}\;.
\end{equation}
Here, $a$ and $b$ are the respective unit costs. $b$ is assumed to be constant, and $a$ to scale linearly with the temperature-dependent viscosity of water $\eta(T)$, calculated following \citet{huber09}. Below, we introduce $\beta = b / a'$, with $a = \eta^\ast a'$ and $\eta^\ast = \eta(T) / \eta(25^{\circ}\text{C})$. The optimal $\chi$ solves the above equation. We use Fick's law \citep{fick1855} to express transpiration and assimilation as a function of stomatal conductance $g_s$: 
\begin{equation}
\label{eq:egs}
    E = 1.6 g_s D
\end{equation}
and 
\begin{equation}
\label{eq:ags}
    A = g_s c_a (1-\chi) \;,
\end{equation}
and use the Rubisco-limited assimilation rate from the FvCB model:
\begin{equation}
\label{eq:ac}
    A = A_C = V_{\mathrm{cmax}} \; m_C \;,
\end{equation}
with
\begin{equation}
\label{eq:mc}
   m_C = \frac{c_i - \Gamma^{\ast}}{c_i + K}\;,
\end{equation}
where $c_i$ is given by $c_a \chi$. $K$ is the effective Michaelis-Menten coefficient for Rubisco-limited assimilation (Sect. \ref{sec:kmm}), and $\Gamma^{\ast}$ is the photorespiratory compensation point in the absence of dark respiration (Sect. \ref{sec:gammastar}). The optimal $\chi$ can be derived as
\begin{equation}
\label{eq:chiopt}
\chi = \frac{\Gamma^{\ast}}{c_a} + \left(1- \frac{\Gamma^{\ast}}{c_a}\right) \frac{\xi}{\xi + \sqrt{D}}\;,
\end{equation}
with 
\begin{equation}
\label{eq:xi}
\xi = \sqrt{\frac{\beta (K+\Gamma^{\ast})}{1.6 \eta^{\ast}}}\;.
\end{equation}
(See Appendix \ref{sec:steps_chi} for intermediate steps.) Because both terms in Eq. \ref{eq:optimality_chi} are divided by $A$, the solution is independent of whether the Rubisco-limited rate $A_C$ or the light-limited rate $A_J$ (see below) are followed. With this prediction for $\chi$, we can use the \textit{Coordination Hypothesis} \citep{chen93, haxeltine96, maire12po} and the light-limited assimilation rate from the FvCB model to write
\begin{equation}
\label{eq:aj}
        A_J = \varphi_0 \; I_{\mathrm{abs}}\;m \;,
\end{equation}
with
\begin{equation}
\label{eq:m_co2limitation}
    m = \frac{c_i - \Gamma^{\ast}}{c_i + 2\Gamma^{\ast}}\;.
\end{equation}
$I_{\mathrm{abs}}$ is the amount of absorbed light and $\varphi_0$ is the intrinsic quantum yield efficiency. This equation has the form of a LUE model (Eq. \ref{eq:luemodel}) in that $A_J$ scales linearly with $I_{\mathrm{abs}}$. Using Eqs. \ref{eq:xi} and \ref{eq:chiopt}, $m$ can be expressed directly as
\begin{equation}
\label{eq:m_lue}
    m = \frac{c_a - \Gamma^{\ast}}{c_a + 2 \Gamma^{\ast} + 3 \Gamma^{\ast} \sqrt{\frac{1.6 \eta^{\ast} D }{\beta\;(K+\Gamma^{\ast})}}} \;.
\end{equation}
The unit cost ratio $\beta$ has been estimated by \citet{wang17natpl} to 240 based on global leaf $\delta^{13}$C data and a simplified version of the P-model (assuming $\Gamma^\ast = 0$ and neglecting the \jmax\ limitation). Here, we re-estimated $\beta$ to 146 based on the full version of the model using the same global leaf $\delta^{13}$C dataset. This is more strictly consistent with the model formulation implemented here. The value for $\beta$ used here is 146.0 (unitless). Eq. \ref{eq:m_lue} provides the basis for predicting \coo\ assimilation rates in the form of a LUE model (Eq. \ref{eq:luemodel}) where LUE is a function of $T$ and $p$ (both affecting $\Gamma^{\ast}$, $K$, and $\eta^\ast$; see Secs. \ref{sec:gammastar} and \ref{sec:kmm}), $D$, and $c_a$.  

The prediction of optimal $\chi$ has a number of corollaries (see Appendix \ref{sec:corollary}). An estimate for stomatal conductance ($g_s$) and the intrinsic water use efficiency (iWUE = $A/g_s$) directly follow from the optimal water-carbon balance (Eq. \ref{eq:optimality_chi}). By assuming $A_J=A_C$, we can further derive \vcmax , as well as dark respiration ($R_d$), which is a function of \vcmax\ (see Secs. \ref{sec:vcmax} and \ref{sec:rd}).

\subsection{Introducing \jmax\ limitation}
\label{sec:jmax}
Eq. \ref{eq:aj} assumes that the light response of $A$ is linear up to the coordination point. In reality, rates saturate towards high light levels because the electron transport rate $J$, necessary for the regeneration of ribulose-1,5,- bisphosphate (RuBP) tends towards a maximum \jmax . To account for this effect, Eq. \ref{eq:aj} can be modified, following the formulation by \citet{smith37}, using a non-rectangular hyperbola relationship between $A_J$ and $I_{\mathrm{abs}}$ to allow for the effect of finite $J_{\mathrm{max}}$:
\begin{equation}
\label{eq:ajlim}
    A_J = \varphi_0 \; I_{\mathrm{abs}} \; m \; \underbrace{ \frac{1}{\sqrt{1+ \left( \frac{4\;\varphi_0\;I_{\mathrm{abs}}}{J_{\mathrm{max}}} \right)^{2}}} }_{L}
\end{equation}
In this equation $A_J$ is no longer linear with respect to $I_{\mathrm{abs}}$ and thus does not have the form of a LUE model. However, \jmax\ is assumed here to acclimate on longer time scales to $I_{\mathrm{abs}}$, so that the marginal gain in assimilation $A$ per unit change in \jmax\ is equal to the unit cost ($c$) of maintaining \jmax .
\begin{equation}
\label{eq:jmaxpartial}
    \frac{\partial A}{\partial J_{\mathrm{max}}} = c 
\end{equation}
The unit cost $c$ is assumed to include the maintenance of light-harvesting complexes and various proteins involved in the electron transport chain. The cost of maintaining a given \jmax\ is thus assumed to scale linearly with \jmax\ and that this proportionality is constant ($c$ is constant). By taking the derivative of Eq. \ref{eq:ajlim} with respect to \jmax\ and re-arranging terms (see Appendix \ref{sec:steps_jmaxlim} for intermediate steps), we obtain the \jmax\ limitation factor $L$ in Eq. \ref{eq:ajlim} as:
\begin{equation}
\label{eq:factor_jmaxlim}
    L = \sqrt{ 1 - \left( \frac{c^\ast}{m} \right)^{2/3} }\;,
\end{equation}
with $c^\ast = 4c$. Note that $L$ is independent of $I_\text{abs}$. Hence, $A_J$ is again a linear function of absorbed light. The cost factor $c^\ast$ is estimated from published values of \jmax $:$\vcmax $=$ 1.88 at 25$^\circ$C. \citep{kattge07} and $\chi =$ 0.8 \citep{lloyd94} at $c^\ast = 0.41$ \citep{wang17natpl}. The revised LUE model thus becomes
\begin{equation}
\label{eq:ajlim4}
    A = \varphi_0 \; I_{\mathrm{abs}} \; m' \;,
\end{equation}
with
\begin{equation}
\label{eq:mprime}
    m' = m \; \sqrt{1 - \left( \frac{c^\ast}{m} \right)^{2/3} } \;.
\end{equation}

Wang et al (2017a) showed that this formulation of \jmax\ costs leads to a realistic dependence of the \jmax $:$\vcmax $=$ ratio on growth temperature.

As shown by \citet{smith19ecollett}, an alternative approach can be used to introduce the effects of \jmax\ limitation, replacing Eq. \ref{eq:ajlim} by the more widely used one-parameter family of saturation curves following \citet{farquhar84}. This alternative is described in Appendix \ref{sec:jmaxlim_smith} and implemented as an optional method in the R package \textit{rpmodel}.


\section{Methods}
\label{sec:methods}

\subsection{The light use efficiency model}
\label{sec:luemodel}
$A$ is commonly expressed in mol m$^{-2}$ s$^{-1}$. For further model description and evaluation, we refer to ecosystem-scale quantities in mass units of assimilated C and model GPP (g C m$^{-2}$ d$^{-1}$) following Eq. \ref{eq:luemodel}
with 
\begin{align}
    \label{eq:iabs_identification}
    \text{fAPAR} \cdot \text{PPFD} &\mathrel{\widehat{=}} I_{\text{abs}} \\
    \label{eq:lue_identification}
    \text{LUE} &\mathrel{\widehat{=}} \varphi_0(T) \; \beta(\theta) \; m' \; M_C
\end{align}
Here, $M_C$ is the molar mass of carbon (12.0107 g mol$^{-1}$) to convert from molar units to mass units, and PPFD is the photosynthetic photon flux density per square metre, integrated over a day (mol m$^{-2}$ d$^{-1}$). fAPAR is unitless and integrates across the canopy, i.e., from fluxes per unit leaf area to fluxes per unit ground area. LUE is in units of g C mol$^{-1}$. The intrinsic quantum yield parameter $\varphi_0$ is modelled as temperature-dependent, and an additional (unitless) empirical soil moisture stress factor ($\beta (\theta)$) is included for modelling LUE.

\subsubsection{Temperature dependence of the intrinsic quantum yield of photosynthesis}
\label{sec:tempstress}
The temperature dependence of the intrinsic quantum yield ($\varphi_0(T)$, mol mol$^{-1}$) is modelled following the temperature dependence of the maximum quantum yield of photosystem II in light-adapted leaves, determined by \citet{bernacchi03pce} as 
\begin{equation}
\label{eq:bernacchi03}
\varphi_0(T) = \frac{a_L b_L}{4} \; ( 0.352 + 0.022\;T - 0.00034\;T^2 )
\end{equation}
where $a_L$ is the leaf absorptance, and $b_L$ is the fraction of absorbed light that reaches photosystem II. The factor $1/4$ is introduced here as the equation given by \citet{bernacchi03pce} applies to electron transport rather than C assimilation. Here, $(a_L b_L / 4)$ is treated as a single calibratable parameter (see Section \ref{sec:calib}) and is henceforth referred to as $\widehat{c_L}\equiv a_L b_L / 4$. (All calibratable parameters are thereafter indicated by a hat over the symbol.) This temperature dependence was not accounted for in earlier P-model publications \citep{keenan17natcomm, wang17natpl}. To test the effect of this temperature dependence on simulated GPP, we conducted alternative simulations, where a constant $\widehat{\varphi_0}$ was calibrated instead (Sect. \ref{sec:protocol}). Note, that $\varphi_0$ includes the factor $a_L$ for incomplete leaf absorbtance, which is commonly quantified separately from the quantum yield efficiency. In other vegetation models, $a_L$ is commonly ascribed a value of 0.72-0.88 \citep{rogers17}. Values of $\varphi_0$ used here are accordingly lower than values for the intrinsic quantum yield reported from experimental studies \citep{long93, singsaas01}. Furthermore, within-canopy reflection and reabsorption mean that leaf-level absorptance is not equivalent to canopy-level absorptance, thus $\varphi_0$ should be regarded as canopy-scale \textit{effective} value of intrinsic quantum yield. It is treated here as a calibratable parameter, which may vary according to the fAPAR forcing data set used. 

\subsubsection{Soil moisture stress}
\label{sec:soilmstress}
$\beta(\theta)$ is an empirical soil moisture stress function. We use results by \citet{stocker18newphyt} to fit this function based on two general patterns. First, the functional form of $\beta(\theta)$ is approximated by a quadratic expression that approaches 1 for soil moisture above a certain threshold $\theta^{\ast}$ and held constant at 1 for soil moisture values above this threshold. Here $\theta$ is the plant-available soil water, expressed as a fraction of field capacity, and $\theta^{\ast}$ is set to 0.6. The general form is:
\begin{equation}
\label{eq:soilmstress}
    \beta =
\begin{cases}
    q(\theta - \theta^{\ast})^2 + 1,& \theta \leq \theta^{\ast}\\
    1,              & \theta > \theta^{\ast}
\end{cases}
\end{equation}
Second, the sensitivity of $\beta(\theta)$ to extreme soil dryness ($\theta \rightarrow 0$) is related to the mean aridity, quantified as the mean annual ratio of actual over potential evapotranspiration (AET/PET) \citep{stocker18newphyt}. The decline in $\beta(\theta)$ with drying soils is steep in dry climates and less steep in less dry climates. In equation 21, the sensitivity parameter $q$ is defined by the maximum $\beta$ reduction at low soil moisture $\beta_0\equiv\beta(\theta=\theta_0)$, leading to $q=(\beta_0-1)/(\theta^{\ast}-\theta_0)^2$. Note that $q$ has a negative value. $\beta_0$ is modelled as a linear function of the mean aridity, :
\begin{equation}
\label{eq:soilmsensitivity}
\beta_0 = \widehat{a_{\theta}} + \widehat{b_{\theta}} (\text{AET}/\text{PET})
\end{equation}
$\widehat{a_{\theta}}$ and $\widehat{b_{\theta}}$ are treated as calibratable parameters. 

Soil moisture ($\theta$), AET, and PET are simulated using the SPLASH model \citep{davis17}, which treats soil water storage as a single bucket and calculates potential evapotranspiration based on \citet{priestleytaylor72}. The only difference to the model version described by \citep{davis17} is that we account here for a variable water holding capacity calculated based on soil texture and depth data from SoilGrids \citep{Hengl2014-jm}. A detailed description of the applied empirical functions for calculating plant-available water holding capacity from texture data is given in Appendix \ref{sec:whc}.

\subsection{Simulation protocol}
\label{sec:protocol}

\subsubsection{Site-scale simulations}

We conducted multiple sets of site-scale simulations (Tab. \ref{tab:setups}) to investigate the dependence of model performance on alternative model setups (variable/fixed soil moisture and temperature effects), alternative choices of forcing data (fAPAR), and alternative observational target data for calibration (GPP based on different flux decompositions). Parameters ($\widehat{c_L}$, $\widehat{a_{\theta}}$, and $\widehat{b_{\theta}}$) were calibrated and evaluated against the appropriate observational data for each set of simulations separately. 

The setup ORG is the P-model in its original form, as described in \citet{wang17natpl}. It uses a fixed quantum efficiency of photosynthesis ($\widehat{\varphi_0}$ is calibrated, instead of $\widehat{c_L}$), and does not account for soil moisture stress ($\beta (\theta)=1$). Here, the model is forced with fAPAR data based on MODIS FPAR (MCD15A3H), linearly interpolated 4-daily values to daily values (see Section \ref{sec:greennessdata}), and is calibrated against GPP data from FLUXNET 2015 based on the nighttime partitioning method (NT) (see Section \ref{sec:datafiltering}). The simulation set BRC (`Bernacchi') is identical to ORG except that  $\widehat{\varphi_0}$ is allowed to vary with temperature following \citet{bernacchi03pce} and Eq. \ref{eq:bernacchi03}, and $\widehat{c_L}$ is calibrated. The full P-model setup (FULL) includes the soil moisture stress function described above, and $\widehat{c_L}$, $\widehat{a_{\theta}}$, and $\widehat{b_{\theta}}$ are calibrated simultaneously. 

All additional simulations account for both temperature and soil moisture effects. The simulation set FULL\textunderscore FPARspl also uses MODIS FPAR data for fAPAR, but applies a spline to get daily values. This is to evaluate alternative interpolation methods. The simulation set FULL\textunderscore EVI uses MODIS EVI (MOD13Q1), interpolated to daily from 8-daily data, to assess to which the degree the model performance depends on the fAPAR forcing data source. See Section \ref{sec:greennessdata} for more information.

All site-scale simulations were calibrated against GPP data (Sec. \ref{sec:calib}), calculated using the nighttime flux decomposition method \citep{Reichstein2005-mp}. Additional simulation sets FULL\textunderscore DT, FULL\textunderscore NTsub, and FULL\textunderscore Ty were used to investigate the dependence of model performance on the choice of observational data used for calibration. We used GPP data based on the nighttime decomposition method \citep{Reichstein2005-mp} for FULL\textunderscore NTsub, the daytime decomposition method \citep{lasslop10} for FULL\textunderscore DT, and an alternative decomposition method, previously used in \citet{wang17natpl}, for FULL\textunderscore Ty. The Ty method estimates a constant monthly background respiration rate fitted to match net ecosystem exchange fluxes of CO$_2$ from measurements assuming a linear or saturating dependence of GPP on PPFD. Calibration and evaluation of FULL\textunderscore DT, FULL\textunderscore NTsub, and FULL\textunderscore Ty are done only for sites and dates where observational data is available for all three datasets (DT, NT, and Ty), hence the distinction between FULL\textunderscore NTsub and FULL. 

\subsubsection{Global simulations}

Global simulations were conducted for the setup FULL, using the respectively calibrated parameters from the site-scale simulations. All vegetation is assumed to follow the C$_3$ photosynthetic pathway and we do not distinguish between croplands and other vegetation. We conducted two simulations with alternative fAPAR forcing data. These are described in Section \ref{sec:forcingdata}.

\begin{table*}
\caption{Model setups. The standard fAPAR data is MODIS FPAR MCD15A3H, where the original data, given at 4-day intervals, is splined to daily values (`spl.'). Alternative greenness forcing data are based on MODIS EVI MOD13Q1, splined from 8-day intervals to daily, and MODIS FPAR MCD15A3H, linearly interpolated (`itpl.') from 4-day intervals to daily. Standard observational GPP data, used for model calibration and evaluation, are from FLUXNET 2015, based on the nighttime flux decomposition method (`NT' in the table, variable \texttt{GPP\textunderscore NT\textunderscore VUT\textunderscore REF} in FLUXNET 2015). Alternative GPP data used based on the daytime flux decomposition method (`DT' in the table, variable \texttt{GPP\textunderscore DT\textunderscore VUT\textunderscore REF}), and based on an alternative method \citep{wang17natpl} (`Ty' in the table). For setups ORG, BRC, FULL, FULL\textunderscore FPARspl, and FULL\textunderscore EVI, data used for the model calibration is from all dates where NT data are available. For setups FULL\textunderscore DT, FULL\textunderscore Ty, and FULL\textunderscore NTsub, calibration data are from all dates where data is available for all three methods DT, NT, and Ty. Column $\varphi_0(T)$ specifies whether the temperature dependence of intrinsic quantum yield is included. Column $\beta(\theta )$ specifies whether soil moisture stress is included. Columns $\widehat{\varphi_0}$, $\widehat{c_L}$, $\widehat{a_{\theta}}$ and  $\widehat{b_{\theta}}$ provide the calibrated parameter values in each simulation set.}
\begin{tabular}{llllllllll}
\tophline
    Setup name                 &  fAPAR data              &  GPP      &  Calibration set  &  $\varphi_0(T)$  &  $\beta(\theta )$  &  $\widehat{\varphi_0}$ &  $\widehat{c_L}$    &  $\widehat{a_{\theta}}$  &  $\widehat{b_{\theta}}$   \\
% \middlehline
%     ORG                        &  FPAR MCD15A3H, spl.     &  NT       &   NT data    &  no         &  no         &  0.0492 &  --     &  -- & --   \\
%     BRC                        &  FPAR MCD15A3H, spl.     &  NT       &   NT data    &  yes        &  no         &  --     &  0.0817 &  -- & --   \\
%     FULL                       &  FPAR MCD15A3H, spl.     &  NT       &   NT data    &  yes        &  yes        &  --     &  0.0870 &  0  & 0.685 \\
% \middlehline
%     NULL                       &  FPAR MCD15A3H, spl.     &  NT       &   NT data    &  no         &  no         &  0.2481$^\ast$   &  --     &  --  & -- \\
% \middlehline
%     FULL\textunderscore FPARspl &  FPAR MCD15A3H, itpl.   &  NT       &   NT data    &  yes        &  yes        &  --     &  0.0846 &  0  & 0.700 \\
%     FULL\textunderscore EVI     &  EVI MOD13Q1, spl.      &  NT       &   NT data    &  yes        &  yes        &  --     &  0.1293 &  0  & 0.766 \\
% \middlehline
%     FULL\textunderscore DT      &  FPAR MCD15A3H, spl.    &  DT       &   NT, DT, Ty &  yes        &  yes        &  --     &  0.0891 & 0   & 0.690 \\
%     FULL\textunderscore Ty      &  FPAR MCD15A3H, spl.    &  Ty       &   NT, DT, Ty &  yes        &  yes        &  --     &  0.0868 & 0   & 0.721 \\
%     FULL\textunderscore NTsub   &  FPAR MCD15A3H, spl.    &  NT       &   NT, DT, Ty &  yes        &  yes        &  --     &  0.0899 & 0   & 0.690 \\
% \bottomhline
\middlehline
    ORG                        &  FPAR MCD15A3H, spl.     &  NT       &   NT data    &  no         &  no         &  0.04998 &  --     &  -- & --   \\
    BRC                        &  FPAR MCD15A3H, spl.     &  NT       &   NT data    &  yes        &  no         &  --     &  0.08179 &  -- & --   \\
    FULL                       &  FPAR MCD15A3H, spl.     &  NT       &   NT data    &  yes        &  yes        &  --     &  0.08718 &  0  & 0.73300 \\
\middlehline
    NULL                       &  FPAR MCD15A3H, spl.     &  NT       &   NT data    &  no         &  no         &  0.2475$^\ast$   &  --     &  --  & -- \\
\middlehline
    FULL\textunderscore FPARitp &  FPAR MCD15A3H, itpl.   &  NT       &   NT data    &  yes        &  yes        &  --     &  0.08486 &  0.0      & 0.74704 \\
    FULL\textunderscore EVI     &  EVI MOD13Q1, spl.      &  NT       &   NT data    &  yes        &  yes        &  --     &  0.13136 &  0.01000  & 0.78419 \\
\middlehline
    FULL\textunderscore DT      &  FPAR MCD15A3H, spl.    &  DT       &   NT, DT, Ty &  yes        &  yes        &  --     &  0.08604 & 0.0      & 0.72735 \\
    FULL\textunderscore Ty      &  FPAR MCD15A3H, spl.    &  Ty       &   NT, DT, Ty &  yes        &  yes        &  --     &  0.08701 & 0.10671  & 0.68802 \\
    FULL\textunderscore NTsub   &  FPAR MCD15A3H, spl.    &  NT       &   NT, DT, Ty &  yes        &  yes        &  --     &  0.08719 & 0.0      & 0.73334 \\
\bottomhline
\end{tabular}
\belowtable{$^\ast$The value represents the fitted LUE, corresponding to $(\varphi_0 m' M_C)$ in Eq. \ref{eq:lue_identification}.} % Table Footnotes
\label{tab:setups}
\end{table*}

\subsection{Model calibration}
\label{sec:calib}
Calibration was performed only for the model parameters determining the quantum efficiency of photosynthesis ($\widehat{\varphi_0}$ or $\widehat{c_L}$, respectively) and the dependence of the sensitivity of the soil moisture stress function on average aridity (parameters $\widehat{a_{\theta}}$ and $\widehat{b_{\theta}}$). Simulated GPP was calibrated to minimise the root mean square error (RMSE) compared to observed daily GPP (Sect. \ref{sec:calibdata}). We used  Generalised Simulated Annealing from the \textit{GenSA} R package \citep{gensa} to optimise model parameters. This algorithm is particularly suited to find global minima of non-linear objective functions in situations where there can be a large number of local minima. To test the robustness of the calibration and evaluation metrics, we additionally performed out-of-sample calibrations for the FULL setup where the training set included data from all but one site. The test dataset, used to calculate \rsq\ and RMSE, contained only data from that single left-out site.

\subsection{Forcing data}
\label{sec:forcingdata}

Unstressed light use efficiency, $m'$ in Eq. \ref{eq:lue_identification}, is simulated using monthly mean values for daytime $T$ and $D$; temporally constant site-specific elevation (used to calculate atmospheric pressure, scaled from sea-level standard pressure of 101325 Pa); and annually varying observed atmospheric \coo\ \citep{MacFarlingMeure2006}, identical across sites. The choice of aggregating to monthly mean values is motivated by the time scale of Rubisco turnover, which limits the rate at which photosynthetic parameters can acclimate to changing environmental conditions \citep{mcnevin06}.

Predicted monthly LUE ($m'$) is multiplied by daily varying $I_\text{abs}$, and response functions $\varphi_0(T)$ and $\beta(\theta)$, driven by daily varying temperature and soil moisture. This choice is motivated by the known rapid response in stomatal conductance to drying soils (represented by $\beta(\theta)$), and the instantaneous temperature response of the quantum yield efficiency ($\varphi_0(T)$). Simulating GPP as the product of LUE and daily varying PPFD would not be consistent with the non-linear instantaneous response of $A$ to light (Eq. \ref{eq:aj}) given the acclimation time scale of photosynthesis (\citep{suzuki01, maire12po}. We therefore evaluate simulated GPP averaged over 8-day periods. The choice of appropriate model prediction and evaluation time scales is further discussed in Sect. \ref{sec:discussion}.  

\subsubsection{fAPAR}
\label{sec:greennessdata}

For site-scale simulations, we used three alternative datasets as model forcing for fAPAR (MODIS FPAR linearly interpolated, MODIS FPAR splined, and MODIS EVI, linearly interpolated, see Tab. \ref{tab:setups}). MODIS FPAR data are from the MCD15A3H, Collection 6 dataset \citep{modis_fpar_6}, given at a resolution of 500 m and 4 days. The  data were filtered to remove data points where clouds were present, values equal to 1.00, and outliers (more than three times the inter-quartile range). Filtered values were replaced by the mean value for the respective day-of-year. To obtain daily varying $I_\text{abs}$ (Eq. \ref{eq:iabs_identification}), two alternatives were explored. For the first (used in all setups except FULL\textunderscore FPARspl), values were linearly interpolated to each day. For setup FULL\textunderscore FPARspl, daily fAPAR values were derived using a cubic smoothing spline (function \texttt{smooth.spline()} with parameter \texttt{spar=0.01} in R \citep{Rcoreteam}). MODIS EVI data is from the MOD13Q1, collection 6 dataset \citep{modis_evi_6}, given at a resolution of 250 m and 8 days. This data were filtered based on the summary quality control flag, removing ``cloudy'' pixels. Gaps were filled and data was linearly interpolated to daily values. All fAPAR data were downloaded from Google Earth Engine using the \textit{google\textunderscore earth\textunderscore engine\textunderscore subsets} library \citep{gee_subset}. 

For global-scale simulations, we used two alternative fAPAR datasets. `MODIS FPAR' is from globally gridded MODIS FPAR data at 0.5$^{\circ}$ resolution derived at ICDC (\url{http://icdc.cen.uni-hamburg.de}), based on the MOD15A2H MODIS Terra Leaf Area Index/FPAR 8-Day L4 Global 500m SIN Grid V006 dataset \citep{modis_MOD15A2H}. For the present application, 8-daily data is aggregated (mean) to monthly data. `fAPAR3g' is based on AVHRR GIMMS FPAR3g v2 data \citep{zhu13rs}, 15 days, $1/12^{\circ}$ resolution and aggregated for the present application to 0.5$^{\circ}$ and monthly data. For all global P-model simulations, fAPAR is held constant for each day in respective months. Simulations cover years 2000-2016. Due to limited temporal coverage, January 2000 data is taken as February 2000 for simulations driven by MODIS FPAR.


\subsubsection{Meteorological data}
\label{sec:ppfd}
For site-scale simulations, the meteorological forcing data are derived from the FLUXNET 2015 Tier 1 dataset (daily), which provides data from measurements taken and processed along with the \coo\ flux measurements. The photosynthetic photon flux density PPFD (mol m$^{-2}$ d$^{-1}$) is derived from shortwave downwelling radiation as $\text{PPFD} = 60 \cdot 60 \cdot 24 \cdot 10^{-6} \; k_\text{EC} R_{\text{SW}}$, where $k_\text{EC} = 2.04\; \mu \text{mol J}^{-1}$ \citep{meek84}, and $R_{\text{SW}}$ is incoming shortwave radiation from daily FLUXNET 2015 data (variable name \texttt{SW\textunderscore IN\textunderscore F}, given in W m$^{-2}$). The factor $k_\text{EC}$ accounts for the energy content of $R_\text{SW}$ and the fraction of photosynthtically active radiation in total short-wave radiation. Daytime vapour pressure deficit (VPD, or $D$ in Sect. \ref{sec:theory}) is calculated from half-hourly FLUXNET 2015 data (variable name \texttt{VPD\textunderscore F}) by averaging over time steps with positive insolation (\texttt{SW\textunderscore IN\textunderscore F}). Daytime air temperature is taken directly from the FLUXNET 2015 dataset (variable name \texttt{T\textunderscore F\textunderscore DAY}). This is a simplification, as we are not calculating leaf temperature or VPD at the leaf surface, which are more directly relevant for photosynthesis. %Precipitation data (variable name \texttt{P\textunderscore F}) is used to force the soil moisture model.

For global-scale simulations, we use daily, 0.5$^{^\circ}$ meteorological forcing from WATCH-WFDEI \citep{Weedon2014-nv}. We use mean daily 2 m air temperature; daily snow and rainfall; shortwave downwelling radiation converted to mol photons m$^{-2}$ d$^{-1}$ by multiplication with $k_\text{EC}$; and daily 2 m specific humidity $(q_\text{air})$, converted to VPD ($D$) as described in Appendix \ref{sec:vpd}. We used daily minimum and maximum air temperatures for each month from CRU TS 4.01 data \citep{harris14} to calculate a respective VPD and use their mean as input to P-model simulations in order to reduce effects of the non-linear dependence of $D$ on $T$ ($\overline{D} = \left( D(T_\text{min}) + D(T_\text{max} \right) / 2$). All processes that depend on atmospheric pressure use Eq. \ref{eq:pz} and the 0.5$^{^\circ}$ resolution elevation map from WATCH-WFDEI \citep{Weedon2014-nv} to calculate a temporally constant atmospheric pressure for each gridcell.

\subsection{Calibration and evaluation data}

\label{sec:calibdata}

\subsubsection{Data for site-scale simulations}

We used data from 59 sites for model calibration and 129 sites for evaluation (Fig. \ref{fig:map_sites}, and Tab. \ref{tab:sites}). The number of valid daily GPP data points used for the calibration set was 135,093 and 212,618 for the evaluation set. The calibration sites were selected based on the apparent reliability of relationships between \coo\ fluxes, co-located greenness data, measured soil moisture, and meteorological variables, emerging from a previous analysis \citep{stocker18newphyt}. For the evaluation, we used all sites except those classified as croplands or wetlands, and seven sites where C$_4$ vegetation is mentioned in the site description available through the FLUXNET2015 dataset (AU-How, DE-Kli, FR-Gri, IT-BCi, US-Ne1, US-Ne2, and US-Ne3).

\begin{figure*}[t]
    \centering
    \includegraphics[width=12cm]{fig/map_sites.pdf}
    \caption{Overview of sites selected for model calibration and evaluation (green dots). All sites and additional information are listed in Tab. \ref{tab:sites}. The color key represents aridity, quantified as the ratio of precipitation over potential evapotranspiration from \citet{greve14}.}
    \label{fig:map_sites}
\end{figure*}

\label{sec:sites}

\label{sec:datafiltering}
GPP predictions by the P-model are compared to GPP estimates from the FLUXNET 2015 Tier 1 data set (downloaded on 13 November, 2016). We used GPP based on the nighttime partitioning method \citep{Reichstein2005-mp} (GPP\textunderscore NT\textunderscore VUT\textunderscore REF) and filtered out negative daily GPP values, data for which more than 50\% of the half-hourly data are gap-filled and for which the daytime and nighttime partitioning methods (GPP\textunderscore DT\textunderscore VUT\textunderscore REF and GPP\textunderscore NT\textunderscore VUT\textunderscore REF, respectively) are inconsistent, i.e., the upper and lower 2.5\% quantiles of the difference between GPP values quantified by each method. For additional simulation sets, model calibration and evaluation was performed using GPP data based on the daytime partitioning method (GPP\textunderscore DT\textunderscore VUT\textunderscore REF) \citep{lasslop10} with analogous filtering steps, and GPP data based on an alternative method that fits a constant ecosystem respiration rate as the net ecosystem exchange under conditions where PPFD tends to zero (FULL\textunderscore Ty, \citet{wang17natpl}). For all calibration and evaluation, we removed data points before the ``MODIS era'' (before 18th of February, 2000).


\subsubsection{Data for global-scale simulations}
\label{sec:global_gpp_data}
We compare the simulated spatial distribution of GPP from global-scale simulations against seven different remote sensing data-driven GPP estimates with global coverage and two sun-induced fluorescence (SiF) data products. The global GPP estimates are from the following models: MTE \citep{jung11jgr}, FLUXCOM (`RS+METEO' setup) \citep{tramontana16bg}, MODIS GPP (MOD17A2H collections 55 and 6) \citep{running04, zhao05, modis_MOD17A2H}, BESS \citep{jiang16rse}, BEPS \citep{he18grl, chen16agrformet}, and VPM \citep{zhang17scidat}. A more detailed description of these models and aggregation to a common grid of 0.5$^{\circ}$ and monthly resolution can be found in \citet{luo18}. For sun-induced fluorescence (SiF), we use data from GOME-2A and GOME-2B, based on v.2 (V27) 740 nm terrestrial chlorophyll fluorescence data from the MetOp-A and MetOp-B satellites \citep{joiner13amt, joiner16amt}. Data were aggregated to monthly and 0.5$^{\circ}$ resolution by mean, as further described in \citet{luo18}.

% \paragraph*{MTE}
% This is an empirical GPP model, based on upscaled relationships derived from eddy covariance flux measurements, using machine learning (Model Tree Ensembles, `MTE') \citep{jung11jgr}. 

% \paragraph*{FLUXCOM}
% This is an updated version of MTE following a similar machine learning method (Random Forests) and updated flux measurements from the FLUXNET 2015 dataset. We use data from their random forest method and their `RS+METEO' setup, which included meteorological predictor variables \citep{tramontana16bg}.

% \paragraph*{MODIS GPP (collection 55 and 6)}
% This is a light use efficiency model following a functional form as defined by Eq. \ref{eq:luemodel} and with prescribed biome-specific optimal LUE values defined that are reduced by stress factors for VPD and temperature \citep{running04}. We use the collection 55 and 6 datasets \citep{zhao05}, aggregated from original data to 0.5$^{\circ}$ and monthly time steps as described in \citet{luo18}. 

% \paragraph*{BESS}

% The Breathing Earth System Simulator (BESS) \citep{jiang16rse} combines remotely sensed surface properties (leaf area index, land surface temperature, canopy structure) with the FvCB photosynthesis model and models for the radiation transfer in the canopy and the surface energy balance. \vcmax\ is prescribed for different plant functional types.

% \paragraph*{BEPS}

% The Boreal ecosystem productivity simulator (BEPS) \citep{he18, chen16agrformet} is similar to BESS in that it is based on enzyme kinetics of the FvCB model and simulates radiative transfer in the canopy (sunlit and shaded leaves). It additionally accounts for soil moisture effects on stomatal conductance and therefore GPP.

% \paragraph*{VPM}

% The Vegetation Photosynthesis Model (VPM) is a LUE model \citep{zhang17scidat}, similar to MODIS GPP, but uses the Enhanced Vegetation Index (EVI from MOD09A1 C6, 500 m, 8 d) to estimate light absorption by chlorophyll specifically. VPM distinguishes makes a distinction between C$_3$ and C$_4$ vegetation and modifies LUE by an extra water-stress scalar estimated by the Land Surface Water Index \citep{Xiao2002-jn}. As MODIS GPP, VPM uses a temperature scalar to modify LUE, but does not use VPD as an input.

% \paragraph*{GOME-2A and GOME-2B}

% XXX

% Second, we use a spatially downscaled sun-induced fluorescence (SiF) data product from \citet{duveiller19essd}. SiF has been shown to correlate well with GPP \citep{frankenberg11grl, guanter14pnas} and provides an independent estimate from the spatial distribution of global GPP here. To derive mean annual values from original data given at 8-day intervals and limit data gaps, we first took the mean for each month, then calculated a mean monthly seasonality across all years (2007-2018), and finally averaged over all months in the mean seasonality.

% Third, we use `MODIS GPP', here, the globally gridded data product, derived at ICDC (\url{http://icdc.cen.uni-hamburg.de}), based on the MOD17A2H MODIS/Terra Gross Primary Productivity 8-Day L4 Global 500m SIN Grid V006 data \citep{modis_MOD17A2H}.

\subsection{Evaluation methods}
\label{sec:methods_eval}

We evaluated both simulated LUE and GPP. The P-model  (Sect. \ref{sec:theory}) predicts variations in LUE across sites (space) and months (monthly LUE $= m'$), while simulated GPP is affected by the PPFD and fAPAR data used as model forcing (Eq. \ref{eq:lue_identification} and Sect. \ref{sec:forcingdata}). Conversely, ``observed'' LUE is calculated as $\text{LUE}_\text{obs} = \text{GPP}_\text{obs} / (\text{fAPAR} \cdot \text{PPFD})$ and the evaluation is thus also affected by the PPFD and fAPAR data. The evaluation of LUE tests the added explanatory power of the P-model compared to models that rely on fixed prescribed LUE values. Evaluating GPP facilitates the comparison of the model performance to similar models of terrestrial GPP. Model performance for GPP is benchmarked against a null model (NULL), which assumes a temporally constant and spatially uniform LUE. The LUE for the NULL model is fitted to observed GPP using linearly interpolated MODIS FPAR and GPP data from the NT method, see Tab. \ref{tab:setups}. Thus, while LUE is constant, the NULL model preserves the spatial and temporal patterns in APAR ($= \text{fAPAR} \cdot \text{PPFD}$).

\subsubsection{Components of variability}
\label{sec:evalmethod_variability}
For LUE, we separately analysed spatial (mean annual values by site) and monthly means only for the FULL setup. For GPP, we analyzed spatial, annual, seasonal (mean by day-of-year), 8-daily, and the variability in daily anomalies from the mean seasonal cycle for all setups. The seasonal variability was determined for different Koeppen-Geiger climatic zones (see Tab. \ref{tab:kgclimate}). Information about the association of sites with climatic zones was extracted from \citet{falge17}. Evaluations were made only for climatic zones with at least three sites. For each component of variability, we calculated the adjusted coefficient of determination ($R^2_\text{adj}$, thereafter referred to as $R^2$), and the root mean square error (RMSE). Figures showing correlations between simulated and observed values additionally present the mean bias, the slope of the linear regression model, and the number of data points ($N$).


\begin{table*}[t]
\caption{Description of Koeppen-Geiger climate zones (based on \citet{falge17}) and number of sites for which data is available per climate zone and hemisphere. Only zones with data from more than three sites are shown.}
\begin{tabular}{llll}
\tophline
  Code & $N$ north & $N$ south & Description \\ 
\middlehline
  Aw   & -- & 5 &  Tropical savannah with dry winter \\ 
  BSk  & 2 & -- & Arid steppe cold \\ 
  BSh  & 2 & -- & Arid steppe hot \\ 
  BWh  & 2 & -- & Arid desert hot \\ 
  Cfa  & 5 & -- & Warm temperate fully humid with hot summer \\ 
  Cfb  & 17 & 2 & Warm temperate fully humid with warm summer \\ 
  Csa  & 7 & -- & Warm temperate with dry and hot summer \\ 
  Dfb  & 9 & -- & Cold fully humid warm summer \\ 
  Dfc  & 3 & -- & Cold fully humid cold summer \\ 
%   ET   & 4 & -- & Polar tundra \\ 
\bottomhline
\end{tabular}
\belowtable{} % Table Footnotes
  \label{tab:kgclimate}
\end{table*}


\subsubsection{Drought response}
\label{sec:droughtresponse}
The bias in GPP (modelled minus observed) was calculated for 20 days before and 80 days after the onset of a drought event as identified by \citet{stocker18newphyt} for 36 sites. Drought events (fLUE droughts) are periods of consecutive days where soil moisture, separated from other drivers using neural networks, reduces LUE below a given threshold. The data specifying the timing and duration of drought events was downloaded from \textit{Zenodo} \citep{flue}. We then re-arranged the data to align all drought events at all sites, normalised data to their median value during the ten days before the onset of droughts (normalisation by subtracting median), and computed quantiles per day, where `day' is defined with respect to the onset of each drought event.


\section{Results}
\label{sec:results}

\subsection{Calibration results}
\label{sec:results_calib}

The calibration of model parameters, done with data from all calibration sites simultaneously, yielded values that closely matched the means across parameter values derived from the out-of-sample calibrations (Fig. \ref{fig:oob}). This confirms the robustness of the calibration and that the model is not overfitted. Similarly for the evaluation metrics, the \rsq\ and RMSE values reported from evaluations against data from all evaluation sites pooled yielded values that closely match the means across the out-of-sample evaluations (each calculated with data from the single left-out site). This analysis also shows that the distribution of the evaluation metrics is skewed, with evaluations against a few sites indicating relatively poor performance (\rsq\ below 0.5 for ZM-Mon, AR-Vir, and FR-Pue), while the most frequent values indicate very good model performance (evaluations at 21 sites giving \rsq\ values of above 0.8). Because the out-of-sample calibrations are computationally very expensive, we performed this analysis only for one setup (FULL) and report evaluation metrics done with pooled data from all evaluation sites for the remainder of the analysis. 

\begin{figure}[!ht]
    \includegraphics[width=\textwidth]{fig/oob_grid.pdf}
    \caption{Out-of-sample calibration and evaluation results. (a-c) Distribution of parameter values from calibrations where data from one site was left out for each individual calibration. Parameters $\widehat{a_\theta}$ and $\widehat{b_\theta}$ are unitless. (d, e) Distribution of evaluation metrics calculated on data from the left-out site based on simulations with model parameters calibrated on all other sites' data. Solid red vertical lines represent the parameter values calibrated with data from all calibration pooled. These are the values reported in Tabs. \ref{tab:rsq} and \ref{tab:rmse}. Dashed red lines represent the mean across values from out-of-bag calibrations and evaluations.}
    \label{fig:oob}
\end{figure}


\subsection{GPP variability across scales}
\label{sec:results_gpp}


Tables \ref{tab:rsq} and \ref{tab:rmse} provide an overview of model performance ($R^2$ and RMSE) in simulating GPP at different scales. The ORG setup captures 68\% of the variance in observed GPP with data aggregated to 8-day means (32'870 data points). Model performance both with respect to explained variance ($R^2$) and the RMSE is improved by including the effects of temperature on quantum yield efficiency in the BRC model setup  ($R^2 = 70$\%), and by including the effects of soil moisture stress in the FULL model setup ($R^2 = 73$\%, Fig. \ref{fig:modobs_xdaily}). Both the BRC and FULL model setup outperform the NULL model.

 % This table is created by eval_pmodel.Rmd, section 'Metrics table'
\begin{table*}[t]
\caption{$R^2$ of simulated and observed GPP based on different model setups and for different components of variability.} 
\begin{tabular}{lllllll}
  \tophline
  Setup & 8-daily & Spatial & Annual & Seasonal & var(daily) & var(annual) \\ 
  \middlehline
  FULL & 0.75 & 0.69 & 0.69 & 0.73 & 0.27 & 0.09 \\ 
  BRC & 0.72 & 0.65 & 0.63 & 0.72 & 0.25 & 0.06 \\ 
  ORG & 0.70 & 0.63 & 0.60 & 0.69 & 0.24 & 0.05 \\ 
  NULL & 0.68 & 0.65 & 0.58 & 0.71 & 0.21 & 0.03 \\ 
  \middlehline
  FULL\_FPARitp & 0.73 & 0.71 & 0.69 & 0.74 & 0.24 & 0.10 \\ 
  FULL\_EVI & 0.70 & 0.58 & 0.47 & 0.71 & 0.27 & 0.15 \\ 
  \middlehline
  FULL\_DT & 0.64 & 0.67 & 0.69 & 0.64 & 0.30 & 0.10 \\ 
  FULL\_NTsub & 0.66 & 0.69 & 0.69 & 0.66 & 0.30 & 0.09 \\ 
  FULL\_Ty & 0.68 &  &  & 0.68 & 0.49 &  \\ 
  \bottomhline
  \end{tabular}
\label{tab:rsq}
\end{table*}


 % Version 2. SOMETHING WENT REALLY WRONG. MUCH HIGHER THAN BEFORE.
 % This table is created by eval_pmodel.Rmd, section 'Metrics table'
\begin{table*}[t]
\caption{Root mean square error (RMSE) of simulated and observed GPP based on different model setups and for different components of variability.} 
\begin{tabular}{lllllll}
  \tophline
  Setup & 8-daily & Spatial & Annual & Seasonal & var(daily) & var(annual) \\ 
  \middlehline
  FULL & 1.96 & 426.66 & 398.63 & 1.84 & 1.55 & 166.97 \\ 
  BRC & 2.05 & 454.78 & 438.14 & 1.89 & 1.54 & 170.54 \\ 
  ORG & 2.15 & 466.19 & 447.80 & 1.99 & 1.54 & 172.54 \\ 
  NULL & 2.19 & 465.21 & 465.99 & 1.94 & 1.58 & 173.71 \\ 
  \middlehline
  FULL\_FPARitp & 2.01 & 427.47 & 404.98 & 1.82 & 1.64 & 165.51 \\ 
  FULL\_EVI & 2.13 & 513.68 & 526.98 & 1.91 & 1.49 & 159.82 \\ 
  \middlehline
  FULL\_DT & 2.16 & 411.30 & 392.34 & 2.00 & 1.69 & 180.75 \\ 
  FULL\_NTsub & 2.15 & 426.64 & 398.60 & 1.98 & 1.70 & 166.97 \\ 
  FULL\_Ty & 1.92 &  &  & 1.79 & 1.38 &  \\ 
  \bottomhline
  \end{tabular}
\label{tab:rmse}
\end{table*}

\begin{figure*}[t]
    \includegraphics[width=12cm]{fig/modobs_xdaily.pdf}
    \caption{Correlation of observed and modelled GPP values of all sites pooled, mean over 8-day periods, for the model setup FULL (a) and NULL (b).}
    \label{fig:modobs_xdaily}
\end{figure*}
The \rsq\ for simulated GPP, aggregated to annual totals, ranges from 0.60 (ORG) to 0.69 (FULL). The NULL model achieves an \rsq\ of 0.58. Most of the explanatory power of the different models for predicting annual total GPP stems from their power in predicting between-site (``spatial'') variations (Fig. \ref{fig:modobs_spatialannual}). The \rsq\ for spatial variations ranges from 0.62 (ORG) to 0.68 (FULL), and 0.62 for the NULL model. In contrast, inter-annual  variations at a site are poorly simulated (\rsq : 0.05-0.10 for P-model setups, and 0.03 for the NULL model). Inter-annual variations are generally much smaller than between site (spatial) variations or seasonal variations. Thus, capturing them is challenging. Inter-annual GPP variations are generally better simulated at sites where the variability is high and in particular at dry sites.

\begin{figure}[!ht]
    \includegraphics[width=\textwidth]{fig/modobs_spatial_annual.pdf}
    \caption{Correlation of modelled and observed annual GPP in simulations FULL (a), NULL (b) and FULL\textunderscore EVI (c). The red line and text are based on means across years by site and represents spatial (across-site) variations. Black lines and text are based on annual values, one line for each site. Lines represent linear regressions. $R^2$ and RMSE statistics for annual values (black text) are based on pooled data from all sites. For a perfect fit between modelled and observed annual GPP values, all black lines (representing the linear regression model of annual values for a single site) would lie on the 1:1 line and have a slope of 1. Slopes that deviate substantially from 1, or even are negative, for some sites shows poor model performance in capturing inter-annual variability.}
    \label{fig:modobs_spatialannual}
\end{figure}


\subsubsection{8-day means}

The P-model version ORG captures 68\% of the variance in observed GPP with data aggregated to 8-day means (32'870 data points). Model performance both with respect to explained variance (\rsq ) and the RMSE is improved when additionally accounting for effects of temperature on the quantum yield efficiency (BRC, \rsq $= 70$\%), and when additionally factoring in the empirical soil moisture stress function (FULL, \rsq $= 73$\%, Fig. \ref{fig:modobs_xdaily}). The NULL model with temporally constant and spatially uniform LUE performed slightly better than ORG, but is outperformed by model versions BRC and FULL. All performance statistics are given in Tables \ref{tab:rsq} and \ref{tab:rmse}.


\subsubsection{Seasonal variations}
\label{sec:results_seasonal}
Seasonal variations are generally reliably simulated (\rsq : 0.697-0.72 for P-model setups, and \rsq : 0.70 for the NULL model, Fig. \ref{fig:season}). Also the NULL model captures most of the seasonal variability, especially in climate zones Dfb and Dfc, and Cfb and Cfa. This indicates that seasonal GPP variations are largely driven by seasonal changes in insolation (PPFD) and vegetation greenness (fAPAR). Accounting for temperature effects on the quantum yield efficiency reduces the overestimation of GPP in spring, except in the case of climate zone Dfb. Observed GPP increases are lagged compared to vegetation greenness, with a delay of up to 2 months at some sites. This lag is clearly visible at almost all sites in Dfb. The early season high bias is largely absent for sites in climate zone Cfb, where observed GPP starts increasing early and the simulations match the observations except at sites CZ-wet, DE-Hai, and FR-Fon, where the simulated start of season is simulated too early.

GPP is overestimated during the dry season in climate zones with a marked dry season (Aw, BSk, Csa, and Csb) in model setups that do not account for soil moisture stress (ORG, BRC, NULL). The NULL model has the largest bias. High VPD during dry periods reduces simulated LUE and leads to lower GPP values and a smaller bias in all the P-model setups (ORG, BRC, FULL). The empirical soil moisture stress function applied in setup FULL eliminates the dryness-related bias in zones Aw, Csa, and Csb and substantially reduces this bias for sites in zone BSk. Observations suggest that GPP declines to values around zero during dry periods at sites in zone BSk (mostly savannah vegetation and grasslands, see Table \ref{tab:sites}). The remaining bias in the FULL model, which includes the soil moisture stress function, is related to the fact that fAPAR remains relatively high and that the soil moisture stress function does not decline to zero.

The ORG and BRC models tend to underestimate peak season GPP more strongly compared to the FULL model. This is a direct consequence of the calibration which balances errors across all data points. Across-site average peak-season maximum GPP is accurately captured by the FULL model in most zones (Fig. \ref{fig:season}), except for an underestimation of GPP in zones Aw, Cfa, and Cfb, and an overestimation in zone Csa. Site-level evaluations suggests no clear relationship between peak-season underestimation and vegetation type in zone Cfb. The overestimation of peak-season GPP in zone Csa is caused by a high bias at sites with evergreen broadleaved vegetation (FR-Pue, IT-Cp2, IT-Cpz); sites with other vegetation types show no consistent peak season bias. 

 \begin{figure}[!ht]
\includegraphics[width=\textwidth]{fig/meandoy_byzone.pdf}
\caption{Mean seasonal cycle. Observations are given by the black line and grey band, representing the median and 33/66 \% quantiles of all data (multiple sites and years) pooled by climate zone. Coloured lines represent different model setups. The annotation above each plot specifies the climate zone (see Tab. \ref{tab:kgclimate}). Only climate zones are shown here for which data from at least five sites was available.}
    \label{fig:season}
\end{figure}

\clearpage


\subsubsection{Spatial and annual variations}

The \rsq\ for annual GPP simulated by the P-model setups ranges from 0.60 (ORG) to 0.69 (FULL). The NULL model achieves an \rsq\ of 0.58. Most of the explanatory power of the different models for predicting annual total GPP stems from their power in predicting between-site (``spatial'') variations (Fig. \ref{fig:modobs_spatialannual}). The \rsq\ for spatial variations ranges from 0.62 (ORG) to 0.68 (FULL), and 0.62 for the NULL model. In contrast, inter-annual  variations (across years within a given site) are poorly simulated (\rsq : 0.05-0.10 for P-model setups, and 0.03 for the NULL model). Interannual variations are generally much smaller than across site variations, which likely adds to the challenge of accurately capturing interannual variations. Interannual GPP variations are generally better captured at sites where the variability is high and in particular at dry sites. 


\subsection{GPP target data}
\label{sec:results_gppdata}

The different flux decomposition methods make fundamentally different assumptions regarding the sensitivity of ecosystem respiration to diurnal changes in temperature. This should lead to systematic differences in derived observational GPP values and should affect model-data disagreement.

Model predictions compare better to GPP data based on the flux decomposition method Ty \citep{wang17natpl} than for GPP data based on the DT and NT methods. For GPP  8-day means, the model achieves an \rsq\ of 0.66 when compared to GPP Ty (model setup FULL\textunderscore Ty), as opposed to 0.61 and 0.63 compared to the DT and NT methods, respectively (FULL\textunderscore DT and FULL\textunderscore NTsub, Tab. \ref{tab:rsq}, Fig. \ref{fig:modobs_10d_gppdata}). Spatial and annual correlations are not evaluated for GPP Ty due to missing data. Correlations at the 8-day, seasonal and daily time scales rely on dates for which neither the NT, DT, nor Ty method has missing values and thus contain an equal number of data points. Therefore NT evaluations, repeated here, are not identical to the ones above and are referred to as `NTsub' in Tables \ref{tab:rsq} and \ref{tab:rmse}. 
 
We found a systematic low bias of simulated GPP in the peak-season in the climatic zone Cfb (warm temperate, fully humid, warm summer). However, as shown in Fig. \ref{fig:modobs_10d_gppdata}, this bias does not seem to be affected by the choice of GPP evaluation data.

\subsection{Drought response}
\label{sec:results_droughtresponse}

The P-model setups that do not include the soil moisture stress function (ORG and BRC) systematically overestimate GPP during droughts (Fig. \ref{fig:modobs_droughtresponse}). This bias increases sharply at the onset of drought events and continues to increase throughout the drought period. The bias is strongly reduced by applying the empiricial soil moisture stress function (Eq. \ref{eq:soilmstress}) in the FULL model. A small bias remains also in the FULL model. This stems from overestimated values at a few sites (in particular AU−DaP, US-Cop, US-SRG, US-SRM, US-Var, US-Whs, US-Wkg), mostly grasslands and sites in seasonally dry climate zones (Aw, BSk, and Csa, see Fig. \ref{fig:season}), where flux measurements indicate an almost complete shut-down of photosynthetic activity during the dry season. In contrast, the fAPAR data (MODIS FPAR) suggest values substantially greater than zero at these sites during these periods. This suggests either contributions to PAR absorption by photosynthetically inactive tissue, underestimation of LUE sensitivity to dry soils at these sites, or an overestimation of the rooting zone moisture availability by SPLASH.

\begin{figure}[t]
\includegraphics[width=8.3cm]{fig/droughtresponse.pdf}
    \caption{Bias in simulated GPP during the course of drought events. Simulated GPP is from a simulation with (FULL) and without (BRC) accounting for soil moisture stress. The timing of drought events is taken from \citet{stocker18newphyt} and is identified by an apparent soil moisture-related reduction of observed light use efficiency at 36 FLUXNET sites. The bias is calculated as simulated minus observed GPP. Data from multiple drought events and sites are aligned by the date of drought onset and aggregated across all sites and events (lines for medians, shaded ranges from the 33\% and 66\% quantiles).}
    \label{fig:modobs_droughtresponse}
\end{figure}


\subsection{Uncertainty from fAPAR input data}
\label{sec:results_greenness}

XXX UPDATE XXX

Tests of the sensitivity of model performance to alternative fAPAR forcing datasets show that the difference between splined and linearly-interpolated MODIS FPAR is negligible. However, model performance is generally better using MODIS FPAR compared to simulations using MODIS EVI. Spatial variations are well captured using MODIS FPAR (Fig. \ref{fig:modobs_spatialannual}, \rsq : 0.70) compared to MODIS EVI (\rsq : 0.56). However, the \rsq\ of inter-annual variations is 0.14 for MODIS EVI and 0.09 for MODIS FPAR. In terms of biases in climate zones, the overestimation of GPP during the dry period in zone BSk is larger when using MODIS EVI than when using MODIS FPAR (Fig. \ref{fig:season_greenness}, right). The positive spring bias in simulated GPP in zone Dfb is present irrespective of the source of the fAPAR forcing (Fig. \ref{fig:season_greenness}, left), as is the peak-season bias of GPP in zones BSk, Cfb, and Csb. Differences between the EVI and FPAR-forced simulations depend on vegetation type. The EVI-forced simulation tends to be low biased in evergreen needle-leaved vegetation, and has generally lower values in all evergreen vegetation types compared to the FPAR-forced simulation. However, there is no general difference in model bias between simulations made with the two forcings in other vegetation types. 

 \begin{figure}[t]
\includegraphics[width=8.3cm]{fig/meandoy_byzone_greenness.pdf}
    \caption{Mean seasonal cycle for model setups with different greenness forcing data for two climate zones (BSk and Dfb, both northern hemisphere). Observations are given by the black line and grey band, representing the median and 33/66 \% quantiles by day-of-year (DOY) of all data (multiple sites and years) pooled by climate zone. Coloured lines represent model setups, forced with different greenness data. The annotation above each plot specifies the climate zone (see Tab. \ref{tab:kgclimate}). Climate zones shown here are illustrative examples.}
    \label{fig:season_greenness}
\end{figure}


\subsection{Uncertainty from GPP target data}
\label{sec:results_gppdata}

Model predictions compare better to GPP data based on the Ty than either the DT or NT methods. For GPP,  8-day means, the model achieves an \rsq\ of 0.69 when compared to GPP Ty (model setup FULL\textunderscore Ty), compared to 0.65 and 0.67 for the DT and NT methods, respectively (FULL\textunderscore DT and FULL\textunderscore NTsub, Tab. \ref{tab:rsq}, Fig. \ref{fig:modobs_10d_gppdata}). Variations in daily GPP anomalies are much better captured in evaluations with Ty (\rsq : 0.49) than DT or NT (\rsq : 0.30). Spatial and annual correlations were not evaluated for Ty because of missing data. The systematic low bias of simulated peak-season GPP in climatic zone Cfb is not affected by the choice of GPP evaluation data (Fig. \ref{fig:modobs_10d_gppdata}). The systematic differences in the level of model-data agreement depending on the target GPP dataset (Fig. \ref{fig:modobs_10d_gppdata}) reflect the fact that these datasets are derived using decomposition methods with different sensitivity to diurnal changes in temperature.

 \begin{figure*}[t]
\includegraphics[width=12cm]{fig/meandoy_modobs_gpp_data.pdf}
    \caption{Model performance subject to comparison with different flux decomposition methods for GPP. (a-c): Mean seasonal cycle of simulated (red) and observed GPP (black) based on different flux decomposition methods for sites in climate zone Cfb north. The grey band represents the 33/66 \% quantiles of observed GPP by day-of-year (DOY). (d-f): Correlation of observed and simulated GPP values of all sites pooled, mean over 8-day periods, all sites pooled. `Observed GPP' refers to the different flux decomposition methods: DT for the daytime method (setup FULL\textunderscore DT), NT for the nighttime method (setup FULL\textunderscore NTsub) and Ty (setup FULL\textunderscore Ty) for the method applied for data used in \citet{wang17rs}. Dotted lines in (d-f) represent the 1:1 relationship, red lines represent the fitted linear regressions.}
    \label{fig:modobs_10d_gppdata}
\end{figure*}


\subsection{LUE}
\label{sec:results_lue}
The FULL version of the P-model captures 32\% of the variability in mean annual LUE across all sites and across the full range of observed mean annual LUE values and vegetation types (Fig. \ref{fig:lue}). 47 \% of the observed LUE variation within vegetation types is captured by the model through the relationships with climate, without the need to specify parameters for specific vegetation types. 

30\% of the variability in monthly mean LUE is captured by the model, with data from all sites and years pooled (Fig. \ref{fig:lue}). The model overestimates monthly LUE values and underestimates LUE at the lowest and highest end of the LUE range respectively. The low-end overestimation is reflected by the overestimation of GPP in the spring at winter-cold sites (Sect. \ref{sec:results_seasonal}) and during soil moisture droughts (Sect. \ref{sec:results_droughtresponse}). The underestimation of high monthly values is not clearly linked to any particular vegetation type.

\subsection{Global GPP}

Simulated global total GPP is 105 GtC yr$^{-1}$ when using MODIS FPAR and 121 GtC yr$^{-1}$ when using fAPAR3g forcing data (mean over years 2001-2011, setup FULL). The spatial pattern of simulated GPP differs substantially between simulations forced by MODIS FPAR and fAPAR3g (Fig. \ref{fig:gpp_global}). This is most evident in their latitudinal distribution (Fig. \ref{fig:gpp_by_lat}). The global spatial pattern of fAPAR3g-based GPP simulated by the P-model generally matches the global distribution of the mean across other remote sensing-based GPP models and lies within the range of their estimates for the latitudinal distribution. The MODIS FPAR-forced P-model simulation suggests lower values in the tropics that differ from the fAPAR3g-based estimates by a factor of $\sim$2 around the equator. The moderate tropical GPP of the MODIS FPAR-based P-model simulation agrees well with the latitudinal distribution of sun-induced fluorescence (SiF) from GOME-2A and GOME-2B.

 \begin{figure}[!ht]
\includegraphics[width=\textwidth]{fig/maps_comparison.pdf}
\caption{Global distribution of GPP. Shown are the mean annual values, averaged over years 2000 to 2016. The GPP shown as ``mean of other models'' is the average of MTE \citep{jung11jgr}, FLUXCOM (`RS+METEO' setup) \citep{tramontana16bg}, MODIS GPP (collection 55 and 6) \citep{running04, zhao05}, BESS \citep{jiang16rse}, BEPS \citep{he18grl, chen16agrformet}, and VPM \citep{zhang17scidat}. P-model results are from simulations with the FULL setup and calibrated parameters as given in Table \ref{tab:setups}.}
    \label{fig:gpp_global}
\end{figure}


 \begin{figure*}[t]
\includegraphics[width=12cm]{fig/gpp_by_latitude.pdf}
    \caption{Latitudinal distribution of GPP and SiF. Values shown (GPP on the left y-axis, SiF on the right y-axis) are gridcell area-weighted sums along 0.5$^{\circ}$ latitudinal bands.}
    \label{fig:gpp_by_lat}  
\end{figure*}


 \begin{figure*}[t]
\includegraphics[width=12cm]{fig/modobs_lue_FULL.pdf}
    \caption{Modelled (simulations FULL) versus observed LUE. (a) Mean monthly LUE with data pooled from all sites and available years. (b) Mean annual LUE by site (small dots and color) and vegetation type (large dots and color). ee metrics are given at the top with numbers in brackets referring to the regression of data aggregated by vegetation types and non-bracketed numbers for data aggregated by sites. Dotted lines represent the 1:1 relationship, red lines represent the fitted linear regression to all data in (a) and to mean annual LUE by site in (b). The grey band in (b) represents the 95 \% confidence interval of the linear regression. Vegetation types are: closed shrubland (CSH); deciduous broadleaf forest (DBF); evergreen broadleaf forest (EBF); evergreen needleleaf forest (ENF); grassland (GRA); mixed deciduous and evergreen needleleaf forest (MF); open shrubland (OSH); savanna ecosystem (SAV); woody savanna (WSA). }
    \label{fig:lue}
\end{figure*}


% \subsubsection{VPD}

% Vapour pressure deficit ($D$) is calculated from vapour pressure (CRU) or specific humidity (WATCH-WFDEI) input data. In general, $D$ is the difference between actual and saturation vapour pressure:
% \begin{equation}
%     D = e_a - e_s
% \end{equation}
% We calculate saturation vapour pressure ($e_s$, in Pa) following Allen et al. (2005) as a function of temperature  as
% \begin{equation}
% e_s = 611 \; \exp \left( {\frac{17.27\;T_C}{T_C+237.3}} \right)
% \end{equation}
% $T_C$ is the temperature expressed in units of degrees Celsius. Note that Allen et al. (2005) use 6.108 instead of 6.11. The Python implementation uses daily maximum and daily minimum temperature, which is equivalent to above formulation with $T=(T_{\text{min}}+T_{\text{max}})/2$. $e_a$ is provided by CRU as an input dataset. WATCH-WFDEI provides specific humidity data ($q$), which can be converted to the mass mixing ratio of water vapor to dry air ($w$) (dimensionless) by
% \begin{equation}
%     w = \frac{q}{1-q}
% \end{equation}
% and finally to actual vapour pressure by
% \begin{equation}
%     e_a = P \frac{w R_v}{R_d + w R_v}
% \end{equation}
% where $P$ is the atmospheric pressure (Pa), $R_v$ is the specific gas constant for water vapour and $R_d$ is the specific gas constant for dry air. The specific gas constants are calculated from the universal gas constant $R$ and the molecular mass $M$ as $R_{\text{specific}}=R/M$. ($R$ is 8.3143 J mol$^{-1}$ K$^{-1}$. The molecular mass of dry air is $M_d=$ 28.963 g mol$^{-1}$ and the molecular mass of water vapor is $M_v=$ 18.02 g mol$^{-1}$.) Atmospheric pressure is assumed to be at standard conditions (101325 Pa) corrected for local elevation (barometric formula adopted from SPLASH, Eq. 20 in Davis et al., 2017).




% \begin{table}
% \centering
% \begin{tabular}{ p{1.2cm} p{2.5cm} p{2.5cm} p{4cm} l }
% \multicolumn{5}{l}{\textbf{ WATCH-WFDEI }} \\
% \hline
% \textbf{symbol} & \textbf{variable name} & \textbf{file name} & \textbf{variable description} & \textbf{units} \\
% \hline
% $T$ & \texttt{Tair} & \texttt{Tair\_WFDEI} & 2 m instantaneous air temperature & K \\
% $P_{\text{rain}}$    & \texttt{Rainf}  & \texttt{Rainf\_WFDEI\_CRU} & Rainfall rate, bias corrected with CRU TS3.101 data (TS3.21 for 2010-2012) and gauge ``catch corrected'' (average over previous 3 hrs) & kg m$^{-2}$ s$^{-1}$ \\  
% $P_{\text{snow}}$    & \texttt{Snowf} & \texttt{Snowf\_WFDEI\_CRU} & Snowfall rate, bias corrected with CRU TS3.101 data (TS3.21 for 2010-2012) and gauge ``catch corrected'' (average over previous 3 hrs)  & kg m$^{-2}$ s$^{-1}$ \\  
% $R_{\text{SW}}$    & \texttt{SWdown} & \texttt{SWdown\_WFDEI} & Short-wave downwards surface radiation flux (average over previous 3 hours) & W m$^{-2}$ \\  
% $q$    & \texttt{Qair} & \texttt{Qair\_WFDEI} & 2 m instantaneous specific humidity & kg kg$^{-1}$ \\  
% $p$    & \texttt{PSurf} & \texttt{PSurf\_WFDEI} & Instantaneous surface pressure & Pa \\ 
% \hline
% \end{tabular}
% \caption{Variables used from WATCH-WFDEI meteorological data.}
% \label{tab:meteovars}
% \end{table}


\section{Discussion}
\label{sec:discussion}

The performance of the P-model can be compared to results obtained from other remote-sensing driven GPP models (RS-models). In its FULL setup, the P-model achieves an \rsq\ of 0.73 and a RMSE of 2.00 g C m$^{-2}$ d$^{-1}$, in simulating 8-day mean GPP and evaluated against GPP data (NT method) from 129 sites. This can be compared to predictions from the VPM model (\rsq : 0.74, RMSE: 2.08 g C m$^{-2}$ d$^{-1}$, 113 sites, 8-daily, \citet{Zhang2017-yr}), or BESS (\rsq : 0.67, RMSE: 2.58 g C m$^{-2}$ d$^{-1}$, 113 sites, 8-daily, \citet{jiang16rse}). The performance of the P-model in simulating \textit{annual} GPP across all 131 sites (\rsq : 0.69) can be compared to results from MODIS GPP (MOD17A2, \rsq = 0.73, 12 sites, \citet{heinsch06}, and for the updated version MOD17A2H: \rsq = 0.62, 18 sites, \citet{wang17rs}), or BEPS (\rsq : 0.81, RMSE: 347 g C m$^{-2}$ d$^{-1}$, 124 sites \citep{he18grl}). Unfortunately, we cannot present a direct comparison between these models, based on data from identical dates and sites. A targeted model intercomparison may address this. While seasonal and spatial variations in GPP are reliably simulated by the P-model, the model's performance in simulating interannual GPP variations is weaker. Similar results regarding relatively poor model performance in explaining interannual variations have been found from previous studies in both empirical \citep{richardson07gcb, urbanski07jgr} and process model-based \citep{keenan12gcb, biederman16gcb} analyses. This is likely due to lagged effects of climate anomalies expressed through biotic responses \citep{richardson07gcb, keenan12gcb}.

The P-model-based estimates of global GPP (105 GtC yr$^{-1}$ when using MODIS FPAR and 121 GtC yr$^{-1}$ when using fAPAR3g forcing data, mean over 2000-2011, FULL setup) are within the range of other estimates of global GPP (also means over 2000-2011): 133 GtC yr$^{-1}$ for MTE \citep{jung11jgr}, 130 GtC yr$^{-1}$ for FLUXCOM \citep{tramontana16bg}, 113 GtC yr$^{-1}$ for MODIS-55 GPP and 105 GtC yr$^{-1}$ for MODIS-6 GPP \citep{running04, zhao05}, 133 GtC yr$^{-1}$ for BESS \citep{jiang16rse, ryu11gbc}, 121 GtC yr$^{-1}$ for BEPS \citep{he18grl, chen16agrformet}, and 135 GtC yr$^{-1}$ for VPM \citep{zhang17scidat}. The P-model results presented here are based on simulations that embody relatively strong simplifying assumptions. In particular, we assumed all vegetation to follow the C$_3$ photosynthetic pathway and we made no distinction between croplands and other vegetation, although crops are often more productive \citep{guanter14pnas}. Due to the short period for which forcing data and outputs from comparable models are available, we did not analyse temporal trends in global GPP here. Analyses that are not shown here suggest the introduction of the \jmax\ limitation (not included, e.g., in \citet{keenan17natcomm}) to substantially increase the sensitivity of GPP to \coo\ but we have not further evaluated respective model behaviour against other observations, e.g., \coo\ manipulation experiments. This will have to be addressed before this P-model version (including the \jmax limitation) can reliably be used for the simulation of \coo -related trends. \citet{stocker19natgeo} found that soil moisture effects did not significantly affect global GPP trend over the past decades.

The large spread of tropical GPP estimates is striking. The highest estimate among the other GPP models we used for evaluation here - coming from BESS - is more than 50\% higher than MODIS GPP from Collection 6. The fAPAR3g-based P-model tropical GPP estimate falls within the range of other GPP models, while the MODIS FPAR-based estimate is lower than all other models. However, the latter's comparably low tropical GPP agrees well with the latitudinal distribution of SiF. However, large changes in leaf area index across latitudes, combined with a dependency of the SiF signal on vegetation structure \citep{zeng19rse} may undermine the validity of SiF as a benchmark for GPP models. A lack of evaluation data from eddy-covariance measurements in dense tropical forests precludes us from drawing conclusions on the accurateness of these diverging tropical GPP estimates. 

With a particular focus on soil moisture effects, \citet{stocker19natgeo} presented global GPP based on the P-model, corresponding to a setup with the soil moisture stress function but without the temperature-dependence of the quantum yield efficiency. They also used a different parametrisation with $\varphi_0 = 0.0579$, $a_\theta = 0.107$, and $b_\theta = 0.478$ for their intermediate model version. Their estimate for global GPP is around 130 PgC yr$^{-1}$ for recent years, having increased from less than 120 PgC yr$^{-1}$ in the early 1980s. This is within the range of other models. In contrast to MODIS GPP, the P-model shows a clear increasing trend in these simulations.


The coefficients of determination (\rsq ) of simulated versus observed values are lower for LUE (0.37 for the spatial correlation in the FULL setup, Fig. \ref{fig:lue}b) than for GPP (0.70 for the spatial correlation in the FULL setup). This is because GPP variations are strongly driven by variations in absorbed light (PPFD$\cdot$fAPAR), which are observed and used for modelling. In contrast, variations in LUE cannot be observed directly. Using remotely-sensed information for estimating LUE variations, e.g., based on sun-induced fluorescence \citep{frankenberg18, li18gcb, ryu19rse} or alternative reflectance indices \citep{gamon92, gamon16pnas, Badgley2017-tw}, is an active field of research and the separation of remotely sensed signals into contributions by LUE and absorbed light remains challenging \citep{porcarcastell14, ryu19rse}. Other remote sensing-based GPP models rely on vegetation type-specific model parameters for LUE \citep{Zhang2017-yr, running04, jiang16rse}. The P-model in its FULL setup explains 53\% of the variations in LUE across sites aggregated to vegetation types without relying on vegetation or biome-type specific parametrisations. In its ORG setup, it explains 21\% of the variations (not shown), and 51\% of the variations when excluding sites classified as `open shrublands', which tend have a substantially lower LUE than simulated by the P-model (not shown). In spite of this substantial portion of explained variability, the NULL model with its temporally constant and spatially uniform LUE achieves higher \rsq\ values for GPP than the ORG P-model setup at the spatial, annual, and seasonal scales (Tab. \ref{tab:rsq}). This indicates that the spatial and temporal variations in absorbed light are the main drivers of GPP in LUE-type models and underlines the importance of evaluation against a NULL model benchmark. Taken together, these findings demonstrate that the P-model offers a simple but powerful method for simulating terrestrial GPP using readily available input datasets and a very small number of free (calibratable) parameters. Here, three parameters are calibrated (for the FULL setup). Other model parameters are derived from independent field and laboratory measurements.

% It appears surprising, however, that the NULL model achieves only a slightly lower \rsq\ (0.64, spatial) compared to the FULL model (0.70, spatial), while using a constant and uniform LUE across all sites and time. 

Accounting for the temperature-dependence of the quantum yield efficiency ($\varphi_0$) clearly improves model predictions. The parameter $\varphi_0$ is commonly treated as a constant in global vegetation models \citep{rogers17}. Our results indicate potential for improving DVM photosynthesis routines by accounting for the temperature-dependence of $\varphi_0$. 

$\varphi_0$ appears as a linear scalar in the LUE model. However, the magnitude of this scalar is uncertain and depends on whether incomplete light absorption by the leaf is included in the definition of $\varphi_0$ or in fAPAR data. We have used MODIS FPAR and MODIS EVI data to define fAPAR in different model setups.  While the two are well correlated, their absolute values differ. Hence, we have calibrated an \textit{apparent} quantum yield efficiency ($\widehat{\varphi_0}$) to GPP data separately for different fAPAR datasets, thereby implicitly distinguishing what components of light absorption factors are contained in the fAPAR data. The leaf absorbtance $a_L$, which is typically taken to be around 0.8 in global vegetation models \citep{rogers17} is similar to the ratio of fitted $\widehat{\varphi_0}$ values for simulation FULL and FULL\textunderscore EVI, here calculated as 0.67 (Tab. \ref{tab:setups}).

An improvement in model performance is obtained by accounting for soil moisture stress using an empirical function.  However, the use of an empirical function masks underlying processes. Furthermore, the use of an empirical function is not consistent with the optimality approach that underlies the P-model. The bias reduction associated with using an empirical soil moisture stress function hints at missing factors in the theoretical approach which rests on an assumed constancy of the unit costs of transpiration ($a$ in Eq. \ref{eq:optimality_chi}). \citet{prentice14ecollett} provide a definition of $a$ that is explicit in terms of plant hydraulic traits and physical properties that determine water transport along the plant-soil-atmosphere continuum. In particular, $a \propto ( \Delta \Psi k_s )^{-1}$, where $\Delta \Psi$ is the maximum daytime difference in leaf-to-soil water potential and $k_s$ is the sapwood area-specific permeability. However, large variations in stomatal conductance are known to occur in response to relatively fast soil dry-downs (time scale of days) \citep{keenan10agrformet, egea11, stocker18newphyt}. This suggests a potential to improve the P-model by allowing the unit cost of transpiration to be a function of rooting-zone moisture availability, and by coupling stomatal conductance with the soil water balance.

Observational uncertainty could affect both parameter calibration and model evaluation. \citet{keenan19natee} found a systematic bias in GPP estimates based on the nighttime partitioning method due to inhibition of leaf respiration in the light \citep{kok49, wehr16}, which affects fluxes unevenly throughout the season and across vegetation types. However, we found no clear difference in model-data agreement, nor in fitted parameters, in comparisons of three alternative GPP datasets that use different approaches to decompose net \coo\ exchange fluxes from eddy covariance measurements into ecosystem respiration and GPP terms.

We have found a consistent early-season high-bias in simulated GPP for numerous sites in regions with deciduous broadleaved vegetation in temperate and cold climates (in particular US-MMS, IT-Col, US-WCr, US-UMd, US-UMB, and US-Ha1), and also in mixed and needle-leaved stands (in particular US-Syv, US-NR1, FI-Hyy, CA-Qfo, and CA-Man). The temperature-dependence of the intrinsic quantum yield, as introduced in setups BRC and FULL, did not resolve this bias. Additional analyses (not shown) suggested that this bias is not related to soil temperatures. The P-model, as applied here, uses daily air temperature for simulating temperature stress on the intrinsic quantum yield in setups BRC and FULL. A reduction in the quantum yield efficiency arises from several mechanisms, including increased non-photochemical quenching, a reduction in chlorophyll and absorption by screening pigments \citep{huner93, oquist03, ensminger04gcb, adams04, verhoeven14}. These adaptations serve to limit oxidative damage under high light and low temperature conditions, where an imbalance between electron supply and demand exists, arising from an imbalance between temperature-insensitive photochemical rates and temperature-sensitive biochemical rates. The reversion of these adaptations and resumption of the intrinsic quantum yield efficiency and photosynthesis requires sustained temperatures above a certain critical threshold \citep{tanja03} and exhibits a delay with respect to instantaneous air temperatures \citep{pelkonen80, makela04}. Approaches accounting for a delayed resumption of photosynthesis after cold periods offer scope for further improvement of the P-model and may be included in global vegetation and Earth system models where this effect is currently not accounted for \citep{tanja03, rogers17}.

There is a positive bias in simulated GPP during the dry season at a number of sites where the vegetation phenology is influenced by drought. The positive bias is related to the combination of using prescribed fAPAR data, which shows substantial absorption by non-green vegetation, and insufficient sensitivity of simulated LUE to soil drying. However, GPP is accurately simulated at other sites affected by seasonally recurring water stress. The modelled sensitivity to dry soils is determined by the soil moisture stress function, which depends on the mean aridity of the site as estimated using a fixed depth soil moisture "bucket". Accounting for variability in rooting zone depth, which may also be influenced by local topographical factors and access to groundwater \citep{fan13sci, fan17pnas} may help to minimise model biases in drought-prone areas.

The current implementation of the P-model involves some simplifications in terms of climate drivers by using average daily meteorological conditions, measured above the canopy, as input. Optimality in balancing carbon and water costs for average daily conditions is not necessarily equivalent to optimality in balancing integrated water and carbon costs over the diurnal cycle. Large variations in ambient conditions over a diurnal cycle, combined with a non-linear dependence of costs on these conditions suggest that the approach of taking average daily conditions may be an over-simplification. Nevertheless, prior evaluations have shown robust and accurate predictions of optimal $\chi$ across a range conditions \citep{wang17natpl}. Using above-canopy VPD values instead of VPD at the leaf surface for scaling water losses implicitly assumes a perfectly coupled atmospheric boundary layer. Using above-canopy air temperature instead of leaf temperatures introduces a bias when the two become decoupled \citep{michaletz15tee}. The impact of these simplifications may be minor but should be evaluated. %Furthermore, the light use efficiency model, using prescribed fAPAR, implies that no distinction is made between light absorptance by the canopy of diffuse and direct radiation. 

A further simplification is that investment in electron transport capacity (expressed by \jmax ) and investments in the carboxylation capacity (expressed by \vcmax ) are coordinated so that for conditions with which the model is forced (here, monthly means of daily averages), photosynthesis operates at the co-limitation point of the light- and Rubisco-limited assimilation rates and an effective linear relationship between absorbed light and mean assimilation emerges. This assumption follows from the \textit{coordination hypothesis} \citep{chen93, haxeltine96}, which itself can be understood as an optimality principle \citep{haxeltine96}, and is well supported by observations \citep{maire12po}. However, this coordination is contingent on the time scale at which photosynthetic acclimation occurs, which is not known precisely  \citep{smithdukes13gcb, way14}. By simulating $\chi$ usingh monthly mean meteorological variables, we assume a monthly time scale of acclimation. This is probably a conservative estimate \citep{smithdukes17, veres84}. Considering the concave relationship of assimilation rates and absorbed light that follows from the FvCB model for a given \jmax , linearly scaling a given monthly LUE term with daily varying absorbed light levels should lead to an overestimation of assimilation rates at high light levels. This overestimation should disappear as the time scale over which light levels are averaged is increased. However, our results do not confirm these expectations. The fact that the model did not exhibit a systematic error in simulating GPP variations when applied at the daily time scale suggests that the day-to-day variability in light levels is relatively small compared to the within-day variability.

\begin{figure}[t]
\includegraphics[width=8.3cm]{fig/hist_anomalies.pdf}
    \caption{Distribution of anomalies from the mean seasonal cycle, evaluated for daily values (a) and 8-day means (b).} 
    \label{fig:modobs_anomalies}
\end{figure}


\conclusions
The P-model provides a simple, parameter-sparse but powerful method to predict photosynthetic capacity and light use efficiency across a wide range of climatic conditions and vegetation types. It provides a basis for a terrestrial light use efficiency model driven by remotely sensed vegetation greenness. Using optimality principles for the formulation of the P-model reduces its dependence on uncertain or vegetation type-specific parameters and enables robust predictions of GPP and its variations through the seasons, between years, and across space. Further work is required to develop a distinct treatment of C$_4$ vegetation for global applications and additional evaluations are needed to examine the P-model's sensitivity to increasing \coo . We have shown that accounting for the effects of low soil moisture and the reduction in the quantum yield efficiency under low temperatures improves model performance. There is potential to include below-ground water limitation effects in the mechanistic optimality framework of the P-model. 

\codedataavailability{The P-model is implemented as an R package (\textit{rpmodel}) and available through \url{https://stineb.github.io/rpmodel/}. The R package will be made available through CRAN. Code for all evaluations presented here is available through \url{https://github.com/stineb/eval_pmodel}. Model outputs are available on \textit{Zenodo} \url{http://doi.org/10.5281/zenodo.3247930}.}


\appendix
\section{Site information}

Table \ref{tab:sites} provides meata information and references for each site from the FLUXNET2015 Tier 1 dataset, used for model calibration and evaluation in the present study.

% This table is created by 

\section{Temperature and pressure dependence of photosynthesis parameters}

\subsection{Photorespiratory Compensation Point $\Gamma^\ast$}
\label{sec:gammastar}
The temperature and pressure-dependent photorespiratory compensation point in absence of dark respiration $\Gamma^\ast(T,p)$ is calculated from its value at standard temperature ($T_0=$ 25${^\circ}$C) and atmospheric pressure ($p_0 = $101325 Pa), referred to as $\Gamma^\ast_{25, p_0}$. It is modified by temperature following an Arrhenius-type temperature response function $f_{\text{Arrh}}(T_K, \Delta H_{\Gamma\ast})$ with activation energy $\Delta H_{\Gamma\ast}$, and is corrected for atmospheric pressure $p(z)$ at elevation $z$. 
\begin{equation}
\label{eq:gammastar}
    \Gamma^\ast (T_K, z) = \Gamma^\ast_{25, p_0} \; f_{\text{Arrh}}(T_K, \Delta H_{\Gamma\ast}) \; \frac{p(z)}{p_0}
\end{equation}
Values of $\Delta H_{\Gamma\ast}$ and $\Gamma^\ast_{25, p_0}$ are taken from \citet{bernacchi01}. The latter is converted to Pa and standardised to $p_0$ simply by multiplication with $p_0$ ($\Gamma^\ast_{25, p_0} = 42.75\; \mu$mol mol$^{-1} \cdot 10^{-6} \cdot 101325$ Pa $ = 4.332$ Pa). $\Delta H_{\Gamma\ast}$ is 37830 J mol$^{-1}$. All parameter values are summarised in Tab. \ref{tab:params}. The function $p(z)$ is defined in Sec \ref{sec:press}. Note that $T_K$ indicates that the respective temperature value is given in Kelvin and $T_{K,0}=$ 298.15 K.

To correct for effects by temperature following the Arrhenius Equation with its form $x(T_K)=\exp(c-\Delta H_a/(T_K R))$, the temperature-correction function $f_{\text{Arrh}}(T_K, \Delta H_a)$, used in Eq. \ref{eq:gammastar} and further equations below, is given by:
\begin{equation}
    f_{\text{Arrh}}(T_K) = x(T_K)/x(T_{K,0}) = \exp \left( \frac{\Delta H (T_K - T_{K,0})}{T_{K,0}\: R\: T_K} \right) 
\end{equation}
where $\Delta H$ is the respective activation energy (e.g., $\Delta H_{\Gamma\ast}$ in Eq. \ref{eq:gammastar}), and $R$ is the universal gas constant (8.3145 J mol$^{-1}$ K$^{-1}$).

\subsection{Deriving $\Gamma^\ast$}
The temperature and pressure dependency of $\Gamma^\ast$ follows from the temperature dependencies of $K_c$, $K_o$, $V_\text{c,max}$, and $V_\text{o,max}$ and the pressure dependency of $pO_2(p)$:
\begin{equation}
\label{eq:gsbasic}
    \Gamma^\ast (T_K, p) = \frac{pO_2(p)\: K_c(T_K)\: V_\text{omax}(T_K)}
                        {2\: K_o(T_K)\: V_\text{cmax}(T_K)}
\end{equation}
$pO_2(p)$ is the partial pressure of atmospheric oxygen (Pa) and scales linearly with $p(z)$. $K_c$ is the Michaelis-Menten constant for carboxylation (Pa); $K_o$ is the Michaelis-Menten constant for oxygenation (Pa); $V_\text{cmax}$ is maximum rate of carboxylation ($\mu$mol~m$^{-2}$~s$^{-1}$); and $V_\text{omax}$ is the maximum rate of oxygenation ($\mu$mol~m$^{-2}$~s$^{-1}$). The temperature-dependency equations for these four terms are given in Table 1 of \citet{bernacchi01} with respective scaling constants $c$ and activation energies $\Delta H_a$ as :
\begin{subequations}
\begin{align}
    K_c(T_K) &= \exp(38.05-79.43/(T_K R)) \\
    K_o(T_K) &= 1000 \cdot \exp(20.30-36.38/(T_K R)) \\
    V_\text{o,max}(T_K) &= \exp(22.98-60.11/(T_K R)) \\
    V_\text{c,max}(T_K) &= \exp(26.35-65.33/(T_K R))
\end{align}
\end{subequations}
By substituting the temperature-dependency equations for each term in Eq. \ref{eq:gsbasic} and rearranging terms, $\Gamma^\ast$ can be written as
\begin{equation}
    \label{eq:gsto}
    \Gamma^\ast(T_K, z) = pO_2(z)\: \exp(6.779-37.83/(T_K R))\;.
\end{equation}
With $pO_2(p)$ at standard atmospheric pressure (101325 Pa) taken to be 21000 Pa, and assuming a constant mixing ratio across the troposphere, its pressure dependence can be expressed as 
\begin{equation}
    \label{eq:oxy}
    pO_2(p) = 0.2095 \cdot p(z)\;
\end{equation}
hence
\begin{equation}
    \label{eq:gstop}
    \Gamma^\ast(T_K, p) = p(z) \exp(5.205-37.83/(T_K R))  % 6.779+\log(0.207254)=5.20519 
\end{equation}
We can use this to calculate $\Gamma^\ast$ at standard temperature ($T_K=$ 298.15 K) and pressure ($p(z)=$ 101325 Pa) as $\Gamma^\ast_{25, p_0} = 4.332$ Pa. 

Note that to convert Eq. \ref{eq:gsto} to the form corresponding to the one given by \citet{bernacchi01}, the partial pressure of oxygen ($pO_2$) has to be assumed at standard conditions. $pO_2$ is approximately 21000 Pa and with the standard atmospheric pressure of 101325 Pa, $pO_2$ can be converted from Pascals to parts-per-million (ppm) as $21000/101325 \times 10^6 = 207254$ ppm = $\exp(12.24)$ ppm. This can be combined with the exponent in Eq. \ref{eq:gsto} to $\exp(12.24) \cdot \exp(6.779) = \exp(19.02)$. This corresponds to the parameter values determining the temperature dependence of $\Gamma^\ast$ given by \citet{bernacchi01} as  $\Gamma^\ast = \exp(19.02-37.83/(T_K R))$.



%% \\\\\\\\\\\\\\\\\\\\\\\\\\\\\\\\\\\\\\\\\\\\\\\\\\\\\\\
%% MICHAELIS-MENTON COEFFICIENT
%% ///////////////////////////////////////////////////////

\subsection{Michaelis-Menten Coefficient of Photosynthesis}
\label{sec:kmm}
The effective Michaelis-Menten coefficient $K$ (Pa) of Rubisco-limited photosynthesis (Eq. \ref{eq:ac}) is determined by the Michaelis-Menten constants for the carboxylation and oxygenation reactions \citep{farquhar80}:
%% ------------------------------------------------------------------------ %%
%% eq:michaelis | Michaelis Menten coefficient
%% ------------------------------------------------------------------------ %%
\begin{equation}
\label{eq:michaelis}
  K(T_K, p) = K_c(T_K)\: \left( 1 + \frac{pO_2(p)}{K_o(T_K)} \right) \;,
\end{equation}
where $K_c$ is the Michaelis-Menten constant for CO$_2$ (Pa), $K_o$ is the Michaelis-Menten constant for the carboxylation and oxygenation reaction, respectively, and $pO_2$ is the partial pressure of oxygen (Pa). $K_c$ and $K_o$ follow a temperature dependence, given by the Arrhenius equation analogously to the temperature dependence of $\Gamma^\ast$ (Eq. \ref{eq:gammastar}):
%% ------------------------------------------------------------------------ %%
%% eq:kcko | Michaelis Menten Kc & Ko coefficients
%% ------------------------------------------------------------------------ %%
\begin{subequations}
\label{eq:kcko}
\begin{align}
  K_c(T_K)& = K_{c25}\: f_{\text{Arrh}}(T_K, \Delta H_{Kc}) \label{eq:kc} \\
    K_o(T_K)& = K_{o25}\: f_{\text{Arrh}}(T_K, \Delta H_{Ko}) \label{eq:ko}
\end{align}
\end{subequations}
Values $\Delta H_{Kc} = 79430$ J mol$^{-1}$, $\Delta H_{Ko} = 36380$ J mol$^{-1}$, $K_{c25} = 39.97$ Pa, and $K_{o25} = 27480$ Pa are taken from \citet{bernacchi01} and (see also Tab. \ref{tab:params}). The latter two have been converted from $\mu$mol mol$^{-1}$ in \citet{bernacchi01} to units of Pa by multiplication with the standard atmosphere (101325 Pa). Note that $K_{c25}$ and $K_{o25}$ are rate constants and are independent of atmospheric pressure. Pressure-dependence of $K$ is solely in $pO_2(p)$ (see Eq. \ref{eq:oxy}).

% where $K_{c25}$ is the Michaelis-Menten constant for CO$_2$ at 25~$^{\circ}$C (Pa), $K_{o25}$ is the Michaelis-Menten constant for O$_2$ at 25~$^{\circ}$C (Pa), $\Delta H_{a,c}$ is the activation energy for carboxylation [79$\,$430 J mol$^{-1}$], $\Delta H_{a,o}$ is the activation energy for oxygenation [36$\,$380 J mol$^{-1}$], $R$ is the universal gas constant [8.31447 J mol$^{-1}$ K$^{-1}$], and $T_K$ is the leaf temperature [K].

% \noindent Once again, leaf temperature, as in Eqns. \ref{eq:michaelis} and \ref{eq:kcko}, may be substituted by the ambient air temperature, $T_{air}$, converted to units of Kelvin. 

% The partial pressure values of $K_{c25}$ and $K_{o25}$ are based on the empirical temperature dependencies given by \citet{bernacchi01}, in mole fractions, converted to partial pressures by Dalton's Law (see Eq. \ref{eq:pp}):
%% ------------------------------------------------------------------------ %%
%% eq:kcko25 | Michaelis Menten Kc & Ko coefficients
%% ------------------------------------------------------------------------ %%
% \begin{subequations}
% \label{eq:kcko25}
% \begin{align}
%   K_{c25}&=1\times 10^{-6} \exp \left[ 38.05 - 
%       \frac{\Delta H_{a,c}}{298.15\: R}
%     \right]\: P_{atm} \label{eq:kc25} \\
%     K_{o25}&=1\times 10^{-3} \exp \left[ 20.30 - 
%       \frac{\Delta H_{a,o}}{298.15\: R}
%     \right]\: P_{atm} \label{eq:ko25}
% \end{align}
% \end{subequations}

% \noindent The experiments to determine these values were conducted in a laboratory under ambient atmospheric conditions at the University of Illinois at Urbana-Champaign (Carl Bernacchi, personal communication, 24 March 2015) where the elevation is approximately 227 m above mean sea level. The constant partial pressure of $K_{c25}$ is 39.93 Pa and the partial pressure of $K_{o25}$ is 27$\,$460 Pa (based on $P_{atm}$~=~98627~Pa).

%% \\\\\\\\\\\\\\\\\\\\\\\\\\\\\\\\\\\\\\\\\\\\\\\\\\\\\\\\\\\\\\\\\\\\\\\\ %%
%% ATMOSPHERIC PRESSURE
%% //////////////////////////////////////////////////////////////////////// %%

\subsection{Atmospheric pressure}
\label{sec:press}
The elevation-dependence of atmospheric pressure is computed by assuming a linear decrease in temperature with elevation and a mean adiabatic lapse rate \citep{berberan97}:
%% ---------------------------------------------------------------%%
%% eq:pz | Atmospheric pressure as a function of elevation
%% ---------------------------------------------------------------%%
\begin{equation}
\label{eq:pz}
    p(z) = p_0 \left( 
      1 - \frac{L z}{T_{K,0}} 
    \right)^{g M_a (R L)^{-1}} \;,
\end{equation} 
where $z$ is the elevation above mean sea level (m), $g$ is the gravitational constant (9.80665 m s$^{-2}$), $p_0$ is the standard atmospheric pressure at 0 m a.s.l. (101325 Pa), $L$ is the mean adiabatic lapse rate (0.0065 K m$^{-2}$), $M_a$ is the molecular weight for dry air (0.028963 kg mol$^{-1}$), and $R$ is the universal gas constant (8.3145 J mol$^{-1}$ K$^{-1}$). All parameter values that are held fixed in the model (not calibrated) are summarised in Tab. \ref{tab:params}.



\section{Corollary of the $\chi$ prediction}
\label{sec:corollary}

\subsection{Stomatal conductance}
\label{sec:gs}
Stomatal conductance $g_s$ (mol C Pa$^{-1}$) follows from the prediction of $\chi$ given by Eq. \ref{eq:chiopt} and $g_s = A / ( c_a\;(1-\chi) )$ (from Eq. \ref{eq:ags}). Stomatal contuctance can thus be written as
\begin{equation}
\label{eq:gs}
    g_s = \left( 1 + \frac{\xi}{\sqrt{D}} \right) \frac{A}{c_a - \Gamma^\ast}\;.
\end{equation}
This has a similar form as the solution for $g_s$ derived from a different optimality principle by \citet{medlyn11gcb} (their Eq. 11). Differences are that an additional term $g_0$ is missing here and that $\Gamma^\ast$ does not appear in \citet{medlyn11gcb}. The theory presented by \citet{prentice14ecollett} provides a theoretical interpretation for the parameter $g_1$ in \citet{medlyn11gcb}: It is given by $\xi$ (Eq. \ref{eq:xi}) and can thus be predicted from the environment. However, it is notable that the underlying optimality criterion used by \citet{medlyn11gcb}, as proposed by \citet{Cowan1977-ud}, is one that maintains a constant marginal water cost of carbon gain $\lambda = \partial E / \partial A$. It thus describes an instantaneous $g_s$ adjustment, e.g., to diurnal variations in $D$ and has been adopted into DVMs and ESMs for respective predictions (with a given \vcmax ). In contrast, the theory presented here and underlying the P-model predicts $\chi$ which is jointly controlled by $g_s$ and \vcmax . In other words, it predicts a $g_s$ that is coordinated with \vcmax\ and thus acclimates at a similar time scale (which is on the order of days to weeks). This $\chi$ can be understood as a ``set-point'' for an average $\chi$ with actual $\chi$ varying around it at a daily to sub-daily time scale.

% $g_s$ also follows from the predictions of $A$ and $ci$, using Eq. \ref{eq:fick}. The stomatal conductance to water vapour (not CO$_2$) is:
% \begin{equation}
% g_s^W = \frac{1.6 \; p\; A}{c_a - c_i}
% \end{equation}
% With $g_s^W$ commonly expressed in units of mol H$_2$O m$^{-2}$ s$^{-1}$, $g_s$ in P-model being the stomatal conductance to CO$_2$, and $c_i$ (and $c_a$) defined as CO$_2$ partial pressure in units of Pa, multiplication with atmospheric pressure $p$ (Pa) and the factor 1.6 to convert stomatal conductance to CO$_2$ into stomatal conductance to H$_2$O are required.

% Note that in the P-model output, $c_i$ is given in ppm.

\subsection{Intrinsic water use efficiency}
The intrinsic water use efficiency (iWUE, in Pa) has been defined as the ratio of assimilation over stomatal conductance (to water) \citep{beer09gbc} as $\text{iWUE} = A / (1.6 g_s)$. The factor 1.6 accounts for the difference in diffusivity between CO$_2$ and H$_2$O. Using Fick's Law (Eq. \ref{eq:ags}), this is simply
\begin{equation}
\label{eq:iwue}
    \mathrm{iWUE} = \frac{c_a (1-\chi)}{1.6} \;,
\end{equation}
or, using the prediction of optimal $\chi$ given by Eq. \ref{eq:chiopt}, this can be expressed as
\begin{equation}
    \text{iWUE} = \frac{1}{1.6 \left( 1+ \frac{\xi}{\sqrt{D}} \right) }\; (c_a - \Gamma^\ast)
\end{equation}

\subsection{Maximum carboxylation capacity}
\label{sec:vcmax}
$V_{\mathrm{cmax}}$}
With $A_J=A_C$, \vcmax\ can directly be derived as 
\begin{equation}
    \label{eq:vcmax}
    V_{\mathrm{cmax}} = \varphi_0\;I_{\mathrm{abs}}\;\frac{c_i + K}{c_i + 2\Gamma^\ast} = \varphi_0\;I_{\mathrm{abs}}\; \frac{m}{m_C}\;,
\end{equation}
$c_i$ is given by $c_a \chi$. The second part of the equation follows from the definitions of $m$ (Eq. \ref{eq:m_co2limitation}) and $m_C$ (Eq. \ref{eq:mc}). Normalising \vcmax\ to standard temperature (25$^{\circ}$C) following a modified Arrhenius function based on \citet{kattge07} gives $V_{\mathrm{cmax25}}$ as
\begin{align}
    \label{eq:vcmax25}
    V_{\mathrm{cmax25}} &= V_{\mathrm{cmax}} / f_V (T_K, T_{K,0}) \\ 
    \label{eq:vcmaxsens}
    f_V (T_K, T_{K,0}) &= f_{\text{Arrh}}(T_K, \Delta H_V) \cdot \frac{1+\exp( (T_{K,0}\Delta S-H_d) / (T_{K,0} R) )}{1+\exp( (T_K\Delta S - H_d)/(T_K R) )}
\end{align}
with $H_V$ being the activation energy (71513 J mol$^{-1}$), $H_d$ is the deactivation energy (200000 J mol$^{-1}$), and $\Delta S$ is an entropy term (J mol$^{-1}$ K$^{-1}$) calculated using a linear relationship with $T$ from Kattge and Knorr (2007), with a slope of $b_S =$ 1.07 J mol$^{-1}$ K$^{-2}$ and intercept of $a_S = $ 668.39 J mol$^{-1}$ K$^{-1}$:
\begin{equation}
\label{eq:entropy}
    \Delta S = a_S - b_S T
\end{equation}
Note that $T$ is in units of $^{\circ}$C in above equation. Equation \ref{eq:vcmaxsens} describes the \textit{instantaneous} response to temperature and is not the same as the optimality-driven \textit{acclimation} to temperature predicted by the P-model.


\subsection{Dark respiration $R_{\mathrm{d}}$}
\label{sec:rd}
Dark respiration at standard temperature $R_{\mathrm{d25}}$ is calculated as being proportional to $V_{\mathrm{cmax25}}$:
\begin{equation}
\label{eq:rd25}
    R_{\mathrm{d25}} = b_0 \; V_{\mathrm{cmax25}}
\end{equation}
where $b_0 = 0.015$ \citep{atkin15}. Dark respiration follows a slightly different instantaneous temperature sensitivity than \vcmax\ following \citet{heskel16}:
\begin{align}
\label{eq:rdsens}
    R_{\mathrm{d}} &=  R_{\mathrm{d25}}\; f_R  \\
    f_R &= \exp \left(  0.1012(T_{K,0}-T_K) - 0.0005(T_{K,0}^2-T_K^2) \right) 
\end{align}
By combining Eqs. \ref{eq:vcmaxsens}, \ref{eq:rd25}, and \ref{eq:rdsens}, $R_d$ at growth temperature $T$ can directly be calculated from $V_{\mathrm{cmax}}$ as
\begin{equation}
\label{eq:rd}
    R_d = b_0 \frac{f_R}{f_V}\;V_{\mathrm{cmax}}
\end{equation}

\section{Soil water holding capacity}
\label{sec:whc}
The soil water balance is solved following the SPLASH model but with the total soil water holding capacity per unit ground area ($\theta_\text{WHC}$, in mm) calculated as a function of the soil texture. Precipitation in the form of rain ($P_{\text{rain}}$) and snow ($P_{\text{snow}}$) are taken from WATCH-WFDEI \citep{Weedon2014-nv} and are summed and converted from kg m$^{-2}$ s$^{-1}$ to mm d$^{-1}$ by multiplication with $(60 \cdot 60 \cdot 24)$ s d$^{-1}$. To obtain $\theta_\text{WHC}$, we use soil depth-to-bedrock and texture data from SoilGrids \citep{Hengl2014-jm}, extracted around the FLUXNET sites. We assumed that the plant-available WHC is determined by the WHC down to a maximum depth of 2 m and is limited by the depth to bedrock. The water holding capacity ($w_\text{WHC}$, in mm) was defined as the difference in volumetric soil water storage at field capacity ($W_{\text{FC}}$, in m$^3$ m$^{-3}$) and the permanent wilting point ($W_{\text{PWP}}$, in m$^3$ m$^{-3}$):
\begin{equation}
\theta_\text{WHC} = (W_{\text{FC}} - W_{\text{PWP}}) \; (1-f_\text{gravel})\cdot \min(z_\text{bedrock}, z_\text{max})
\end{equation}
$f_\text{gravel}$ is the gravel fraction, $z_\text{bedrock}$ is the depth to bedrock (in mm), and $z_\text{max}$ is 2000 mm. The volumetric soil water storage at field capacity and wilting point were derived from texture and organic matter content data through pedotransfer functions, as described by \citet{saxton06}. $W_{\text{FC}}$ is calculated as:
\begin{equation}
W_{\text{FC}}= k_\text{FC}+(1.283\cdot k_\text{FC}^{2}-0.374\cdot k_\text{FC}-0.015)\;, 
\end{equation}
where
\begin{align}
k_\text{FC} &=-0.251\cdot f_{\text{sand}} + 0.195\cdot f_{\text{clay}} + 0.011\cdot f_{\text{OM}}\\                            
&+ 0.006\cdot (f_{\text{sand}} f_{\text{OM}})\\
&- 0.027\cdot (f_{\text{clay}} f_{\text{OM}})\\
&+ 0.452\cdot (f_{\text{sand}} f_{\text{clay}})\\
&+ 0.299
\end{align}
$f_{\text{sand}}$, $f_{\text{clay}}$, $f_{\text{OM}}$ are the sand, clay and organic matter contents in percent by weight. $W_{\text{PWP}}$ is calculated as:
\begin{equation}
W_{\text{PWP}} = k_\text{PWP}+(0.14\cdot k_\text{PWP}-0.02) \;,
\end{equation}
where
\begin{align}
k_\text{PWP} & = -0.024 \cdot f_{\text{sand}} + 0.487 \cdot f_{\text{clay}} + 0.006 \cdot f_{\text{OM}} \\
                  &+0.005 \cdot ( f_{\text{sand}} f_{\text{OM}} )\\
                  &-0.013 \cdot ( f_{\text{clay}} f_{\text{OM}} )\\
                  &+0.068 \cdot ( f_{\text{sand}} f_{\text{clay}} )\\
                  &+0.031
\end{align}
%Contents of sand, clay, organic matter and soil depth data were acquired from the ISRIC-SoilGrids web portal. % (ftp://ftp.soilgrids.org/data/aggregated/10km/)


\section{Vapour pressure Deficit}
\label{sec:vpd}
Vapour pressure deficit ($D$) is calculated from specific humidity
($q_\text{air}$) as:
\begin{equation}
    D = e_\text{sat} - e_\text{act} \;,
\end{equation}
with 
\begin{equation}
    e_\text{sat} = 611.0 \cdot \; \exp \left( \frac{17.27 \; T}{T + 237.3} \right)
\end{equation}
and
\begin{equation}
    e_\text{act} = \frac{ p(z) \; w_\text{air} \; R_v }{ R_d + w_\text{air} \; R_v} \;.
\end{equation}
$p(z)$ is atmospheric pressure, taken here as a constant function of elevation $z$ (Sec. \ref{sec:press}), $w_\text{air}$ is the mass mixing ratio of water vapor to dry air (dimensionless) and derived from specific humidity as $w_\text{air} = q_\text{air} / ( 1 - q_\text{air} )$. $R_d$ and $R_v$ are the specific gas constants of dry air and water vapour, respectively and are given by $R/M_d$ and $R/M_v$, respectively; where $R$ is the universal gas constant (8.314 J mol$^{-1}$ K$^{-1}$) and $M_d$ (28.963 g mol$^{-1}$) and $M_v$ (18.02 g mol$^{-1}$) are the molecular mass of dry air of water vapour, respectively. $T$ is air temperature in $^{\circ}$C.


\section{Extended theory}
\subsection{Deriving $\chi$}
\label{sec:steps_chi}

Using Eqs. \ref{eq:egs} and \ref{eq:ags}, the term on the left-hand side of Eq. \ref{eq:optimality_chi} can thus be written as
\begin{equation}
\label{eq:partial1}
    \frac{\partial (E/A)}{\partial \chi} = \frac{1.6\;D}{c_a\;(1-\chi)^2}\;.
\end{equation}
Using Equation \ref{eq:ac} and the simplification $\Gamma^{\ast}=0$, the derivative term on the right-hand-side of Eq.\ref{eq:optimality_chi} can be written as
\begin{equation}
\label{eq:partial2}
    \frac{\partial (V_{\mathrm{cmax}}/A)}{\partial \chi} = - \frac{K}{c_a\;\chi^2}\;.
\end{equation}
Eq. \ref{eq:optimality_chi} can thus be written as
\begin{equation}
    a\;\frac{1.6\;D}{c_a\;(1-\chi)^2} = b\;\frac{K}{c_a\;\chi^2}
\end{equation}
and solved for $\chi$:
\begin{align}
    \chi &= \frac{\xi}{\xi + \sqrt{D}} \\ 
    \xi &= \sqrt{\frac{\beta K}{1.6 \eta^\ast}}
\end{align}
Where $b/a=\beta/\eta^\ast$. The exact solution, without the simplification $\Gamma^{\ast}=0$, and following analogous steps, is 
\begin{align}
\label{eq:chi_exact}
    \chi &= \frac{\Gamma^{\ast}}{c_a} + \left(1- \frac{\Gamma^{\ast}}{c_a}\right)\frac{\xi}{\xi + \sqrt{D}}\\
    \xi &= \sqrt{\frac{b(K+\Gamma^{\ast})}{1.6\;a}}
\end{align}
This can also be written as
\begin{equation}
\label{eq:ci}
    c_i = \frac{\Gamma^{\ast}\sqrt{D}+ \xi\;c_a}{\xi + \sqrt{D}} \;. 
\end{equation}

\subsection{Deriving the \jmax\ limitation factor}
\label{sec:steps_jmaxlim}

By taking the derivative of $A_J$ with respect to \jmax , Eq. \ref{eq:jmaxpartial} can be expressed as
\begin{equation}
    c = \frac{ m (\varphi_0 I_\text{abs})^3}{ 4 \sqrt{ \left[ (\varphi_0 I_\text{abs})^2 + (\frac{J_\text{max}}{4})^2 \right]^3 }}
\end{equation}
This can be re-arranged to
\begin{equation}
    \left(\frac{4c}{m}\right)^{2/3} = \frac{1}{1 + \left( \frac{J_\text{max}}{4\varphi_0 I_\text{abs}}\right)^2}
\end{equation}
For simplification, we can substitute 
\begin{equation}
    k = \frac{4 \varphi_0 I_\text{abs}}{J_\text{max}}
\end{equation}
and 
\begin{equation}
    u = \left(\frac{4c}{m}\right)^{2/3}
\end{equation}
With this, we can write
\begin{equation}
    \frac{1}{1+k^{-2}} = u \;.
\end{equation}
This can be re-arranged to 
\begin{equation}
    (1-u)^{1/2} = \frac{1}{\sqrt{1+k^2}} 
\end{equation}
The right-hand term now corresponds to the \jmax\ limitation factor $L$ in Eq. \ref{eq:ajlim}, and we get Eq. \ref{eq:factor_jmaxlim}.

To sum up, the P-model calculates GPP as 
\begin{equation}
  \text{GPP} = I_\text{abs} \; \varphi_0(T) \; \beta(\theta) \; m' \; M_C \;,
\end{equation}
where
\begin{equation}
    m' = m \; \sqrt{1 - \left( \frac{c^\ast}{m} \right)^{2/3} }
\end{equation}
and 
\begin{equation}
    m = \frac{c_a - \Gamma^{\ast}}{c_a + 2 \Gamma^{\ast} + 3 \Gamma^{\ast} \sqrt{\frac{1.6 \eta^{\ast} D }{\beta\;(K+\Gamma^{\ast})}}} \;.
\end{equation}
$I_\text{abs}$ is the absorbed light (taken as fAPAR$\cdot$PPFD, mol m$^{-2}$), $\varphi_0(T)$ is the temperature-dependent intrinsic quantum yield, $\beta(\theta)$ is the soil moisture stress factor, and $M_C$ is the molar mass of carbon (g mol$^{-1}$).

\subsection{An alternative method for introducing the \jmax\ limitation}
\label{sec:jmaxlim_smith}
Sect. \ref{sec:jmax} introduced the effect of a finite \jmax\, leading to a saturating relationship between absorbed light and the light-limited assimilation rate $A_J$. An alternative method was presented by \citet{smith19ecollett} and is implemented in \textit{rpmodel} as an optional method (argument \texttt{method\textunderscore jmaxlim = "smith19"}). Following their approach, the light-limited assimilation rate is described as
\begin{equation}
\label{eq:aj_smith}
    A_J = \left(\frac{J}{4} \right) \; m \;.
\end{equation}
$m$ is the \coo\ limitation factor (Eq. \ref{eq:m_co2limitation}), and $J$ is a saturating function of absorbed light, approaching \jmax\ for high light levels, following \citet{farquhar80}:
\begin{equation}
\label{eq:j_smith}
   \theta  J^2 - 
    \left(
    \varphi_0  I_{\mathrm{abs}} \; + J_{\mathrm{max}}
    \right)  J +
     \varphi_0 I_{\mathrm{abs}}  J_{\mathrm{max}} = 0 \;.  
\end{equation}
$\theta$ is a unitless parameter determining the curvature of the response of $J$ to $I_{\mathrm{abs}}$, here taken as 0.85, based on \citet{smith19ecollett} and references therein. Eq. \ref{eq:j_smith} can be substituted into Eq. \ref{eq:aj_smith} to yield
\begin{equation}
\label{eq:aj_smith_long}
    A_J = \left( \frac{m}{4} \right)
    \frac{\varphi_0 I_{\mathrm{abs}} + J_{\mathrm{max}} \pm 
    \sqrt{
    \left(\varphi_0 I_{\mathrm{abs}} + J_{\mathrm{max}} \right)^2 -
    4  \theta \varphi_0 I_{\mathrm{abs}} J_{\mathrm{max}}}}
    {2 \theta} \;,
\end{equation}
from which the smaller root is used to derive $A_J$. Similar as in the method used by \citet{wang17natpl} and outlined in Sect. \ref{sec:jmax}, a proportionality between $A_J$ and \jmax\ is assumed ($\partial A / \partial J_{\mathrm{max}} = c$; Eq. \ref{eq:jmaxpartial}). Taking the derivative of Eq. \ref{eq:aj_smith_long} with respect to \jmax\ and setting equal to $c$ leads to 
\begin{equation}
    J_{\mathrm{max}} = \varphi_0 \; I_{\mathrm{abs}} \; \omega
\end{equation}
with
\begin{equation}
    \omega = - \left(1 - 2 \theta \; \right) +
    \sqrt{\left(1 - \theta \right)
    \left(
    \frac{1}{
    \frac{4  c}{m}
    \left(1 - \theta 
    \frac{4  c}{m}\right)
    } - 4  \theta \right) }\;.
\end{equation}
Using this, $A_J$ can be written analogously to Eq. \ref{eq:ajlim4}, but with 
\begin{equation}
\label{eq:mprime_smith}
    m' = m \; \frac{\omega^{\ast}}{8 \theta} \;,
\end{equation}
and 
\begin{equation}
    \omega^{\ast} = 1 + \omega - \sqrt{\left(1 + \omega \right)^2 -
    4  \theta \omega} \;.
\end{equation}
The cost parameter $c$ was assumed to be non-varying. Under
standard conditions of 25 $^{\circ}$C, 101325 Pa atmospheric pressure, 1000 Pa vapor pressure deficit, and 360 ppm \coo , at which the ratio of \jmax\ to \vcmax\ was assumed to be 2.07  \citep{smithdukes17}, $c$ was derived as 0.053 \citep{smith19ecollett}.

Using the definition of \vcmax\ from Eq. \ref{eq:vcmax}, $m$ can be replaced by $m'$ from Eq. \ref{eq:mprime_smith} to calculate an ``intermediate rate of \vcmax'' \citep{smith19ecollett} as
\begin{equation}
    V_\text{cmax} = \varphi_0 \; I_{\mathrm{abs}} \; \frac{m'}{m_C}
\end{equation}

\section{The \texttt{rpmodel()} function of the \textit{rpmodel} R package}

The \textit{rpmodel} R package provides an implementation of the P-model as described here. The main function is \texttt{rpmodel()} which returns a list of variables that are mutually consistent within the theory of the P-model (Sect. \ref{sec:theory}) and based on calculations defined in this paper. References for the returned list of variables are given in Tab. \ref{tab:out_rpmodel}

\noappendix       %% use this to mark the end of the appendix section

%% Regarding figures and tables in appendices, the following two options are possible depending on your general handling of figures and tables in the manuscript environment:

%% Option 1: If you sorted all figures and tables into the sections of the text, please also sort the appendix figures and appendix tables into the respective appendix sections.
%% They will be correctly named automatically.

%% Option 2: If you put all figures after the reference list, please insert appendix tables and figures after the normal tables and figures.
%% To rename them correctly to A1, A2, etc., please add the following commands in front of them:

\appendixfigures  %% needs to be added in front of appendix figures

\appendixtables   %% needs to be added in front of appendix tables

\begin{table*}[t]
\caption{Sites used for evaluation. Lon. is longitude, negative values indicate west longitude; Lat. is latitude, positive values indicate north latitude; Veg. is vegetation type: deciduous broadleaf forest (DBF); evergreen broadleaf forest (EBF); evergreen needleleaf forest (ENF); grassland (GRA); mixed deciduous and evergreen needleleaf forest (MF); savanna ecosystem (SAV); shrub ecosystem (SHR); wetland (WET).} 
\begin{tabular}{lllllllll}
  \tophline
  Site & Lon. & Lat. & Period & Veg. & Clim. & N & Calib. & Reference \\ 
  \middlehline
  AR-SLu & -66.46 & -33.46 & 2009-2011 & MF & Bwk & 446 &  & \citet{AR-SLu} \\ 
  AR-Vir & -56.19 & -28.24 & 2009-2012 & ENF & Csb & 749 & Y & \citet{AR-Vir} \\ 
  AT-Neu & 11.32 & 47.12 & 2002-2012 & GRA & Dfc & 3243 &  & \citet{AT-Neu} \\ 
  AU-Ade & 131.12 & -13.08 & 2007-2009 & WSA & Aw & 532 & Y & \citet{AU-Ade} \\ 
  AU-ASM & 133.25 & -22.28 & 2010-2013 & ENF & BSh & 1045 & Y & \citet{AU-ASM} \\ 
  AU-Cpr & 140.59 & -34.00 & 2010-2014 & SAV & BSk & 1370 &  & \citet{AU-Cpr} \\ 
  AU-Cum & 150.72 & -33.61 & 2012-2014 & EBF & Cfa & 744 &  & \citet{AU-XXX} \\ 
  AU-DaP & 131.32 & -14.06 & 2007-2013 & GRA & Aw & 1402 & Y & \citet{AU-DaP} \\ 
  AU-DaS & 131.39 & -14.16 & 2008-2014 & SAV & Aw & 2265 & Y & \citet{AU-DaS} \\ 
  AU-Dry & 132.37 & -15.26 & 2008-2014 & SAV & Aw & 1598 & Y & \citet{AU-Dry} \\ 
  AU-Emr & 148.47 & -23.86 & 2011-2013 & GRA & Bwk & 755 &  & \citet{AU-Emr} \\ 
  AU-Fog & 131.31 & -12.55 & 2006-2008 & WET & Aw & 878 & Y & \citet{AU-Fog} \\ 
  AU-Gin & 115.71 & -31.38 & 2011-2014 & WSA & Csa & 942 & Y & \citet{AU-XXX} \\ 
  AU-GWW & 120.65 & -30.19 & 2013-2014 & SAV & Bwk & 663 &  & \citet{AU-GWW} \\ 
  AU-Lox & 140.66 & -34.47 & 2008-2009 & DBF & Bsh & 273 &  & \citet{AU-Lox} \\ 
  AU-RDF & 132.48 & -14.56 & 2011-2013 & WSA & Bwh & 431 &  & \citet{AU-RDF} \\ 
  AU-Rig & 145.58 & -36.65 & 2011-2014 & GRA & Cfb & 1130 &  & \citet{AU-XXX} \\ 
  AU-Rob & 145.63 & -17.12 & 2014-2014 & EBF & Csb & 337 &  & \citet{AU-XXX} \\ 
  AU-Stp & 133.35 & -17.15 & 2008-2014 & GRA & BSh & 1318 & Y & \citet{AU-Stp} \\ 
  AU-TTE & 133.64 & -22.29 & 2012-2013 & OSH & BWh &  94 &  & \citet{AU-TTE} \\ 
  AU-Tum & 148.15 & -35.66 & 2001-2014 & EBF & Cfb & 4335 &  & \citet{AU-Tum} \\ 
  AU-Wac & 145.19 & -37.43 & 2005-2008 & EBF & Cfb & 979 &  & \citet{AU-Wac} \\ 
  AU-Whr & 145.03 & -36.67 & 2011-2014 & EBF & Cfb & 1065 & Y & \citet{AU-Whr} \\ 
  AU-Wom & 144.09 & -37.42 & 2010-2012 & EBF & Cfb & 934 & Y & \citet{AU-Wom} \\ 
  AU-Ync & 146.29 & -34.99 & 2012-2014 & GRA & BSk & 392 &  & \citet{AU-Ync} \\ 
  BE-Bra & 4.52 & 51.31 & 1996-2014 & MF & Cfb & 4208 & Y & \citet{BE-Bra} \\ 
  BE-Vie & 6.00 & 50.31 & 1996-2014 & MF & Cfb & 4733 & Y & \citet{BE-Vie} \\ 
  BR-Sa3 & -54.97 & -3.02 & 2000-2004 & EBF & Am & 1206 &  & \citet{BR-Sa3} \\ 
  CA-Man & -98.48 & 55.88 & 1994-2008 & ENF & Dfc & 1411 &  & \citet{CA-Man} \\ 
  CA-NS1 & -98.48 & 55.88 & 2001-2005 & ENF & Dfc & 771 &  & \citet{CA-NS} \\ 
  CA-NS2 & -98.52 & 55.91 & 2001-2005 & ENF & Dfc & 873 &  & \citet{CA-NS} \\ 
    \bottomhline
\end{tabular}
\label{tab:sites}
\end{table*}
\clearpage

\begin{table*}[t]
\caption{Continued from Table \ref{tab:sites}} 
\begin{tabular}{lllllllll}
  \tophline
  Site & Lon. & Lat. & Period & Veg. & Clim. & N & Calib. & Reference \\ 
  \middlehline  
  CA-NS3 & -98.38 & 55.91 & 2001-2005 & ENF & Dfc & 1069 &  & \citet{CA-NS} \\ 
  CA-NS4 & -98.38 & 55.91 & 2002-2005 & ENF & Dfc & 610 &  & \citet{CA-NS} \\ 
  CA-NS5 & -98.48 & 55.86 & 2001-2005 & ENF & Dfc & 912 &  & \citet{CA-NS} \\ 
  CA-NS6 & -98.96 & 55.92 & 2001-2005 & OSH & Dfc & 913 &  & \citet{CA-NS} \\ 
  CA-NS7 & -99.95 & 56.64 & 2002-2005 & OSH & Dfc & 709 &  & \citet{CA-NS} \\ 
  CA-Qfo & -74.34 & 49.69 & 2003-2010 & ENF & Dfc & 1812 &  & \citet{CA-Qfo} \\ 
  CA-SF1 & -105.82 & 54.48 & 2003-2006 & ENF & Dfc & 525 &  & \citet{CA-SF} \\ 
  CA-SF2 & -105.88 & 54.25 & 2001-2005 & ENF & Dfc & 675 &  & \citet{CA-SF} \\ 
  CA-SF3 & -106.01 & 54.09 & 2001-2006 & OSH & Dfc & 651 &  & \citet{CA-SF} \\ 
  CH-Cha & 8.41 & 47.21 & 2005-2014 & GRA & Cfb & 2885 &  & \citet{CH-Cha} \\ 
  CH-Dav & 9.86 & 46.82 & 1997-2014 & ENF & ET & 4444 &  & \citet{CH-Dav} \\ 
  CH-Fru & 8.54 & 47.12 & 2005-2014 & GRA & Cfb & 2566 & Y & \citet{CH-Fru} \\ 
  CH-Lae & 8.37 & 47.48 & 2004-2014 & MF & Cfb & 3204 & Y & \citet{CH-Lae} \\ 
  CH-Oe1 & 7.73 & 47.29 & 2002-2008 & GRA & Cfb & 2104 & Y & \citet{CH-Oe1} \\ 
  CN-Cha & 128.10 & 42.40 & 2003-2005 & MF & Dwb & 982 &  & \citet{CN-Cha} \\ 
  CN-Cng & 123.51 & 44.59 & 2007-2010 & GRA & Bsh & 1113 & Y & \citet{CN-Cng} \\ 
  CN-Dan & 91.07 & 30.50 & 2004-2005 & GRA & ET & 647 &  & \citet{CN-Dan} \\ 
  CN-Din & 112.54 & 23.17 & 2003-2005 & EBF & Cfa & 917 &  & \citet{CN-Din} \\ 
  CN-Du2 & 116.28 & 42.05 & 2006-2008 & GRA & Dwb & 616 &  & \citet{CN-Du2} \\ 
  CN-Ha2 & 101.33 & 37.61 & 2003-2005 & WET & ET & 1030 &  & \citet{CN-Ha2} \\ 
  CN-HaM & 101.18 & 37.37 & 2002-2004 & GRA &  & 688 &  & \citet{CN-HaM} \\ 
  CN-Qia & 115.06 & 26.74 & 2003-2005 & ENF & Cfa & 992 & Y & \citet{CN-Qia} \\ 
  CN-Sw2 & 111.90 & 41.79 & 2010-2012 & GRA & Bsh & 237 &  & \citet{CN-Sw2} \\ 
  CZ-BK1 & 18.54 & 49.50 & 2004-2008 & ENF & Dfb & 1100 &  & \citet{CZ-BK1} \\ 
  CZ-BK2 & 18.54 & 49.49 & 2004-2006 & GRA & Dfb & 161 &  & \citet{CZ-BK2} \\ 
  CZ-wet & 14.77 & 49.02 & 2006-2014 & WET & Cfb & 2605 & Y & \citet{CZ-wet} \\ 
  DE-Gri & 13.51 & 50.95 & 2004-2014 & GRA & Cfb & 3387 & Y & \citet{DE-Gri} \\ 
  DE-Hai & 10.45 & 51.08 & 2000-2012 & DBF & Cfb & 3435 & Y & \citet{DE-Hai} \\ 
  DE-Lkb & 13.30 & 49.10 & 2009-2013 & ENF & Cfb & 1001 &  & \citet{DE-Lkb} \\ 
  DE-Obe & 13.72 & 50.78 & 2008-2014 & ENF & Cfb & 2043 & Y & \citet{DE-Obe} \\ 
  DE-RuR & 6.30 & 50.62 & 2011-2014 & GRA & Cfb & 1195 & Y & \citet{DE-RuR} \\ 
    \bottomhline
\end{tabular}
\label{tab:sites1}
\end{table*}
\clearpage

\begin{table*}[t]
\caption{Continued from Table \ref{tab:sites}} 
\begin{tabular}{lllllllll}
  \tophline
  Site & Lon. & Lat. & Period & Veg. & Clim. & N & Calib. & Reference \\ 
  \middlehline  
  DE-SfN & 11.33 & 47.81 & 2012-2014 & WET & Cfb & 750 &  & \citet{DE-SfN} \\ 
  DE-Spw & 14.03 & 51.89 & 2010-2014 & WET & Cfb & 1339 & Y & \citet{DE-Spw} \\ 
  DE-Tha & 13.57 & 50.96 & 1996-2014 & ENF & Cfb & 4887 & Y & \citet{DE-Tha} \\ 
  DK-NuF & -51.39 & 64.13 & 2008-2014 & WET & ET & 882 & Y & \citet{DK-NuF} \\ 
  DK-Sor & 11.64 & 55.49 & 1996-2014 & DBF & Cfb & 4483 & Y & \citet{DK-Sor} \\ 
  DK-ZaF & -20.55 & 74.48 & 2008-2011 & WET & ET & 381 &  & \citet{DK-ZaF} \\ 
  DK-ZaH & -20.55 & 74.47 & 2000-2014 & GRA & ET & 1696 &  & \citet{DK-ZaH} \\ 
  ES-LgS & -2.97 & 37.10 & 2007-2009 & OSH & Csa & 794 &  & \citet{ES-LgS} \\ 
  ES-Ln2 & -3.48 & 36.97 & 2009-2009 & OSH & Csa &  69 &  & \citet{ES-Ln2} \\ 
  FI-Hyy & 24.30 & 61.85 & 1996-2014 & ENF & Dfc & 4222 & Y & \citet{FI-Hyy} \\ 
  FI-Lom & 24.21 & 68.00 & 2007-2009 & WET & Dfc & 575 &  & \citet{FI-Lom} \\ 
  FI-Sod & 26.64 & 67.36 & 2001-2014 & ENF & Dfc & 2816 & Y & \citet{FI-Sod} \\ 
  FR-Fon & 2.78 & 48.48 & 2005-2014 & DBF & Cfb & 2827 & Y & \citet{FR-Fon} \\ 
  FR-LBr & -0.77 & 44.72 & 1996-2008 & ENF & Cfb & 2800 & Y & \citet{FR-LBr} \\ 
  FR-Pue & 3.60 & 43.74 & 2000-2014 & EBF & Csa & 4723 & Y & \citet{FR-Pue} \\ 
  GF-Guy & -52.92 & 5.28 & 2004-2014 & EBF & Af & 3719 &  & \citet{GF-Guy} \\ 
  IT-CA1 & 12.03 & 42.38 & 2011-2014 & DBF & Csa & 1036 &  & \citet{IT-CA} \\ 
  IT-CA3 & 12.02 & 42.38 & 2011-2014 & DBF & Csa & 913 &  & \citet{IT-CA} \\ 
  IT-Col & 13.59 & 41.85 & 1996-2014 & DBF & Cfa & 2822 & Y & \citet{IT-Col} \\ 
  IT-Cp2 & 12.36 & 41.70 & 2012-2014 & EBF & Csa & 764 & Y & \citet{IT-Cp2} \\ 
  IT-Cpz & 12.38 & 41.71 & 1997-2009 & EBF & Csa & 2601 & Y & \citet{IT-Cpz} \\ 
  IT-Isp & 8.63 & 45.81 & 2013-2014 & DBF & Cfb & 588 & Y & \citet{IT-Isp} \\ 
  IT-Lav & 11.28 & 45.96 & 2003-2014 & ENF & Cfb & 3919 & Y & \citet{IT-Lav} \\ 
  IT-MBo & 11.05 & 46.01 & 2003-2013 & GRA & Dfb & 3236 & Y & \citet{IT-MBo} \\ 
  IT-Noe & 8.15 & 40.61 & 2004-2014 & CSH & Cwb & 3083 & Y & \citet{IT-Noe} \\ 
  IT-PT1 & 9.06 & 45.20 & 2002-2004 & DBF & Cfa & 828 & Y & \citet{IT-PT1} \\ 
  IT-Ren & 11.43 & 46.59 & 1998-2013 & ENF & Dfc & 3043 & Y & \citet{IT-Ren} \\ 
  IT-Ro2 & 11.92 & 42.39 & 2002-2012 & DBF & Csa & 2671 &  & \citet{IT-Ro2} \\ 
  IT-SR2 & 10.29 & 43.73 & 2013-2014 & ENF & Csa & 668 & Y & \citet{IT-SR2} \\ 
  IT-SRo & 10.28 & 43.73 & 1999-2012 & ENF & Csa & 3791 & Y & \citet{IT-SRo} \\ 
  IT-Tor & 7.58 & 45.84 & 2008-2014 & GRA & Dfc & 1487 & Y & \citet{IT-Tor} \\ 
    \bottomhline
\end{tabular}
\label{tab:sites2}
\end{table*}
\clearpage

\begin{table*}[t]
\caption{Continued from Table \ref{tab:sites}} 
\begin{tabular}{lllllllll}
  \tophline
  Site & Lon. & Lat. & Period & Veg. & Clim. & N & Calib. & Reference \\ 
  \middlehline  
  JP-MBF & 142.32 & 44.39 & 2003-2005 & DBF & Dfb & 471 &  & \citet{JP-XXX} \\ 
  JP-SMF & 137.08 & 35.26 & 2002-2006 & MF & Cfa & 1288 & Y & \citet{JP-XXX} \\ 
  NL-Hor & 5.07 & 52.24 & 2004-2011 & GRA & Cfb & 2131 & Y & \citet{NL-Hor} \\ 
  NL-Loo & 5.74 & 52.17 & 1996-2013 & ENF & Cfb & 4507 & Y & \citet{NL-Loo} \\ 
  NO-Adv & 15.92 & 78.19 & 2011-2014 & WET & ET & 151 &  & \citet{NO-Adv} \\ 
  NO-Blv & 11.83 & 78.92 & 2008-2009 & SNO & ET & 112 &  & \citet{NO-Blv} \\ 
  RU-Che & 161.34 & 68.61 & 2002-2005 & WET & Dfc & 313 &  & \citet{RU-Che} \\ 
  RU-Cok & 147.49 & 70.83 & 2003-2014 & OSH & Dfc & 985 &  & \citet{RU-Cok} \\ 
  RU-Fyo & 32.92 & 56.46 & 1998-2014 & ENF & Dfb & 4042 & Y & \citet{RU-Fyo} \\ 
  RU-Ha1 & 90.00 & 54.73 & 2002-2004 & GRA & Dfc & 519 &  & \citet{RU-Ha1} \\ 
  SD-Dem & 30.48 & 13.28 & 2005-2009 & SAV & BWh & 762 & Y & \citet{SD-Dem} \\ 
  SN-Dhr & -15.43 & 15.40 & 2010-2013 & SAV & BWh & 686 & Y & \citet{SN-Dhr} \\ 
  US-AR1 & -99.42 & 36.43 & 2009-2012 & GRA & Cfa & 1011 &  & \citet{US-AR} \\ 
  US-AR2 & -99.60 & 36.64 & 2009-2012 & GRA & Cfa & 882 &  & \citet{US-AR} \\ 
  US-ARb & -98.04 & 35.55 & 2005-2006 & GRA & Cfa & 414 &  & \citet{US-AR} \\ 
  US-ARc & -98.04 & 35.55 & 2005-2006 & GRA & Cfa & 488 &  & \citet{US-AR} \\ 
  US-Blo & -120.63 & 38.90 & 1997-2007 & ENF & Csb & 1827 &  & \citet{US-Blo} \\ 
  US-Cop & -109.39 & 38.09 & 2001-2007 & GRA & BSk & 1067 &  & \citet{US-Cop} \\ 
  US-GBT & -106.24 & 41.37 & 1999-2006 & ENF & Dfc & 541 &  & \citet{US-GBT} \\ 
  US-GLE & -106.24 & 41.37 & 2004-2014 & ENF & Dfb & 2254 & Y & \citet{US-GLE} \\ 
  US-Ha1 & -72.17 & 42.54 & 1991-2012 & DBF & Dfb & 3259 & Y & \citet{US-Ha1} \\ 
  US-KS2 & -80.67 & 28.61 & 2003-2006 & CSH & Cfa & 1263 &  & \citet{US-KS2} \\ 
  US-Los & -89.98 & 46.08 & 2000-2014 & WET & Dfb & 2071 & Y & \citet{US-Los} \\ 
  US-Me1 & -121.50 & 44.58 & 2004-2005 & ENF & Csb & 287 &  & \citet{US-Me1} \\ 
  US-Me2 & -121.56 & 44.45 & 2002-2014 & ENF & Csb & 3525 & Y & \citet{US-Me2} \\ 
  US-Me6 & -121.61 & 44.32 & 2010-2014 & ENF & Csb & 1283 &  & \citet{US-Me6} \\ 
  US-MMS & -86.41 & 39.32 & 1999-2014 & DBF & Cfa & 3524 & Y & \citet{US-MMS} \\ 
  US-Myb & -121.77 & 38.05 & 2010-2014 & WET & Csb & 1153 &  & \citet{US-Myb} \\ 
  US-NR1 & -105.55 & 40.03 & 1998-2014 & ENF & Dfc & 4084 &  & \citet{US-NR1} \\ 
  US-PFa & -90.27 & 45.95 & 1995-2014 & MF & Dfb & 3679 &  & \citet{US-PFa} \\ 
  US-Prr & -147.49 & 65.12 & 2010-2013 & ENF & Dfc & 546 &  & \citet{US-Prr} \\ 
    \bottomhline
\end{tabular}
\label{tab:sites3}
\end{table*}
\clearpage

\begin{table*}[t]
\caption{Continued from Table \ref{tab:sites}} 
\begin{tabular}{lllllllll}
  \tophline
  Site & Lon. & Lat. & Period & Veg. & Clim. & N & Calib. & Reference \\ 
  \middlehline  
  US-SRG & -110.83 & 31.79 & 2008-2014 & GRA & BSk & 2146 & Y & \citet{US-SRG} \\ 
  US-SRM & -110.87 & 31.82 & 2004-2014 & WSA & BSk & 3093 & Y & \citet{US-SRM} \\ 
  US-Syv & -89.35 & 46.24 & 2001-2014 & MF & Dfb & 2045 & Y & \citet{US-Syv} \\ 
  US-Ton & -120.97 & 38.43 & 2001-2014 & WSA & Csa & 4321 & Y & \citet{US-Ton} \\ 
  US-Tw1 & -121.65 & 38.11 & 2012-2014 & WET & Csa & 688 &  & \citet{US-Tw1} \\ 
  US-Tw4 & -121.64 & 38.10 & 2013-2014 & WET & Csa & 325 &  & \citet{US-Tw4} \\ 
  US-UMB & -84.71 & 45.56 & 2000-2014 & DBF & Dfb & 4015 & Y & \citet{US-UM} \\ 
  US-UMd & -84.70 & 45.56 & 2007-2014 & DBF & Dfb & 2050 & Y & \citet{US-UM} \\ 
  US-Var & -120.95 & 38.41 & 2000-2014 & GRA & Csa & 2981 & Y & \citet{US-Var} \\ 
  US-WCr & -90.08 & 45.81 & 1999-2014 & DBF & Dfb & 2333 & Y & \citet{US-WCr} \\ 
  US-Whs & -110.05 & 31.74 & 2007-2014 & OSH & BSk & 1561 &  & \citet{US-Whs} \\ 
  US-Wi0 & -91.08 & 46.62 & 2002-2002 & ENF & Dfb & 228 &  & \citet{US-Wi} \\ 
  US-Wi3 & -91.10 & 46.63 & 2002-2004 & DBF & Dfb & 415 &  & \citet{US-Wi} \\ 
  US-Wi4 & -91.17 & 46.74 & 2002-2005 & ENF & Dfb & 712 & Y & \citet{US-Wi} \\ 
  US-Wi6 & -91.30 & 46.62 & 2002-2003 & OSH & Dfb & 351 &  & \citet{US-Wi} \\ 
  US-Wi9 & -91.08 & 46.62 & 2004-2005 & ENF & Dfb & 302 &  & \citet{US-Wi} \\ 
  US-Wkg & -109.94 & 31.74 & 2004-2014 & GRA & BSk & 2676 &  & \citet{US-Wkg} \\ 
  ZA-Kru & 31.50 & -25.02 & 2000-2010 & SAV & BSh & 2124 &  & \citet{ZA-Kru} \\ 
  ZM-Mon & 23.25 & -15.44 & 2000-2009 & DBF & Aw & 641 & Y & \citet{ZM-Mon} \\ 
  \bottomhline
\end{tabular}
\label{tab:sites4}
\end{table*}
\clearpage

\begin{table*}[t]
\caption{Fixed parameters. 'SC' stands for 'at standard conditions' (25 $^{\circ}$C, 101325 Pa). 'MM coef.' refers to 'Michaelis Menten coefficient'.}
\begin{tabular}{lllll}
  \tophline
    Symbol     & Value   & Units         & Description           &  Reference   \\
  \middlehline
    $\beta$      & 146.0     & 1             & Unit cost ratio, Eq. \ref{eq:optimality_chi} & This study \\
  $\Gamma^\ast_{25, p_0}$ & 4.332 & Pa & Photorespiratory compensation point, SC & \citet{bernacchi01} \\
  $K_{c25}$    & 39.97   & Pa            & MM coef. for CO$_2$, SC&  \citet{bernacchi01} \\
  $K_{o25}$    & 27480   & Pa            & MM coef. for O$_2$, SC&  \citet{bernacchi01} \\
  $\Delta H_{\Gamma\ast}$ & 37830 & J mol$^{-1}$ & Activation energy for $\Gamma^\ast$  & \citet{bernacchi01} \\
  $\Delta H_{Kc}$ & 79430  & J mol$^{-1}$  & Activation energy for $K_c$&  \citet{bernacchi01} \\
  $\Delta H_{Ko}$ & 36380  & J mol$^{-1}$  & Activation energy for $K_o$&  \citet{bernacchi01} \\
  $H_V$        & 71513   & J mol$^{-1}$  & Activation energy for \vcmax\ & \citet{kattge07} \\
  $H_d$        & 200000   & J mol$^{-1}$  & Deactivation energy for \vcmax\ & \citet{kattge07} \\
  $p_0$        & 101325  & Pa            & Standard atmosphere   & -- \\
  $g$          & 9.80665 & m s$^{-2}$    & Gravitation constant  & -- \\
  $L$          & 0.0065  & K m$^{-2}$    & Adiabatic lapse rate  & -- \\
  $R$          & 8.3145  & J mol$^{-1}$ K$^{-1}$ & Universal gas constant & -- \\
  $M_a$        & 28.963  & g mol$^{-1}$  & Molecular mass of dry air & -- \\
    $M_C$        & 12.0107 & g mol$^{-1} $ & Molecular mass of carbon & -- \\ 
  $a_S$        & 668.39  & J mol$^{-1}$ K$^{-1}$ & Intercept for entropy term in Eq. \ref{eq:vcmaxsens} & \citet{kattge07} \\
  $b_S$        & 1.07  & J mol$^{-1}$ K$^{-2}$ & Slope for entropy term in Eq. \ref{eq:vcmaxsens} & \citet{kattge07} \\
  \bottomhline
\end{tabular}
\caption{Fixed parameters. 'SC' stands for 'at standard conditions' (25 $^{\circ}$C, 101325 Pa). 'MM coef.' refers to 'Michaelis Menten coefficient'.}
\label{tab:params}
\end{table*}
\clearpage

\begin{table*}[t]
\caption{Variables returned by the function \texttt{rpmodel()}. Variable names correspond to the named elements of the list returned by the \texttt{rpmodel()} function call. Symbols correspond to their use in this paper.} 
\begin{tabular}{lllll}
  \tophline
  Variable name       & Symbol        & Description                         & Units & Reference \\ 
  \middlehline
  \texttt{ca}         & $c_a$         & Ambient \coo\ partial pressure      & Pa & Sect. \ref{sec:watercarbon} \\
  \texttt{gammastar}  & $\Gamma^\ast$ & Photorespiratory compensation point & Pa & Sect. \ref{sec:gammastar} \\
  \texttt{kmm}        & $K$           & Michaelis-Menten coefficient for photosynthesis & Pa & Sect. \ref{sec:kmm} \\
  \texttt{ns\textunderscore star} & $\eta^\ast$   & Change in the viscosity of water, relative to its value at 25 $^{\circ}$C & unitless & \citet{huber09} \\
  \texttt{chi}        & $\chi$        & Ratio of leaf internal-to-ambient \coo & unitless &  Sect. \ref{sec:watercarbon} \\
  \texttt{ci}         & $c_i$         & Leaf internal \coo partial pressure & Pa & Eq. \ref{eq:ci} \\
  \texttt{lue}        & LUE           & Light use efficiency                & g C mol$^{-1}$ & Eq. \ref{eq:lue_identification} \\
  \texttt{mj}         & $m$           & \coo\ limitation factor for light-limited assimilation & unitless & Eq. \ref{eq:m_co2limitation}  \\
  \texttt{mc}         & $m_C$         & \coo\ limitation factor for Rubisco-limited assimilation & unitless & Eq. \ref{eq:mc}  \\
  \texttt{gpp}        & GPP           & Gross primary production & g C m$^{-2}$ d$^{-1}$ & Eqs. \ref{eq:luemodel} and \ref{eq:lue_identification}  \\
  \texttt{iwue}       & iWUE          & Intrinsic water use efficiency & Pa & Eq. \ref{eq:iwue}  \\
  \texttt{gs}         & $g_s$         & Stomatal conductance & mol C m$^{-2}$ d$^{-1}$ Pa$^{-1}$ & Sect. \ref{sec:gs} \\
  \texttt{vcmax}      & \vcmax        & Maximum rate of carboxylation &  mol C m$^{-2}$ d$^{-1}$ & Eq. \ref{eq:vcmax} \\
  \texttt{vcmax25}    & $V_\text{cmax25}$ & Maximum rate of carboxylation, normalised to 25 $^{\circ}$C &  mol C m$^{-2}$ d$^{-1}$ & Eq. \ref{eq:vcmax25} \\
  \texttt{rd}         & $R_d$         & Dark respiration & mol C m$^{-2}$ d$^{-1}$ & Eq. \ref{eq:rd} \\
   \bottomhline
  \end{tabular}
  \label{tab:out_rpmodel}
\end{table*}
\clearpage

%% Please add \clearpage between each table and/or figure. Further guidelines on figures and tables can be found below.

\authorcontribution{B.D.S. designed the study, wrote the model code, conducted the analysis, and wrote the paper. H. W. developed the model and wrote the an initial version of the model description. N.G.S. developed the model and implemented model code. S.P.H. contributed to designing the study and writing the manuscript. T.K. contributed to study design, model implementation and manuscript writing. D.S. implemented the water holding capacity model. T.D. wrote an initial version of the model code and model documentation. I.C.P. developed the model and contributed to designing the study.} %% this section is mandatory for the journals ACP and GMD. For all other journals it is strongly recommended to make use of this section

\competinginterests{The authors have no competing interests.} %% this section is mandatory even if you declare that no competing interests are present

%\disclaimer{TEXT} %% optional section

\begin{acknowledgements}
B.D.S. was funded by ERC H2020-MSCA-IF-2015, grant number 701329. N.G.S. acknowledges support from Texas Tech University. T.F.K. acknowledges support from the Laboratory Directed Research and Development (LDRD) fund under the auspices of DOE, BER Office of Science at Lawrence Berkeley National Laboratory, and the NASA Terrestrial Ecology Program IDS Award NNH17AE86I. S.P.H. acknowledges support from the ERC-funded project GC 2.0 (Global Change 2.0: Unlocking the past for a clearer future, grant number 694481). I.C.P. acknowledges support from the ERC under the European Union’s Horizon 2020 research and innovation programme (grant agreement no: 787203 REALM). 
\end{acknowledgements}




%% REFERENCES

%% The reference list is compiled as follows:

% \begin{thebibliography}{}

% \bibitem[AUTHOR(YEAR)]{LABEL1}
% REFERENCE 1

% \bibitem[AUTHOR(YEAR)]{LABEL2}
% REFERENCE 2

% \end{thebibliography}




%% Since the Copernicus LaTeX package includes the BibTeX style file copernicus.bst,
%% authors experienced with BibTeX only have to include the following two lines:

\bibliographystyle{copernicus}
\bibliography{beni.bib}

%% URLs and DOIs can be entered in your BibTeX file as:
%%
%% URL = {http://www.xyz.org/~jones/idx_g.htm}
%% DOI = {10.5194/xyz}


%% LITERATURE CITATIONS
%%
%% command                        & example result
%% \citet{jones90}|               & Jones et al. (1990)
%% \citep{jones90}|               & (Jones et al., 1990)
%% \citep{jones90,jones93}|       & (Jones et al., 1990, 1993)
%% \citep[p.~32]{jones90}|        & (Jones et al., 1990, p.~32)
%% \citep[e.g.,][]{jones90}|      & (e.g., Jones et al., 1990)
%% \citep[e.g.,][p.~32]{jones90}| & (e.g., Jones et al., 1990, p.~32)
%% \citeauthor{jones90}|          & Jones et al.
%% \citeyear{jones90}|            & 1990



%% FIGURES

%% When figures and tables are placed at the end of the MS (article in one-column style), please add \clearpage
%% between bibliography and first table and/or figure as well as between each table and/or figure.


%% ONE-COLUMN FIGURES

%%f
%\begin{figure}[t]
%\includegraphics[width=8.3cm]{FILE NAME}
%\caption{TEXT}
%\end{figure}
%
%%% TWO-COLUMN FIGURES
%
%%f
%\begin{figure*}[t]
%\includegraphics[width=12cm]{FILE NAME}
%\caption{TEXT}
%\end{figure*}
%
%
%%% TABLES
%%%
%%% The different columns must be seperated with a & command and should
%%% end with \\ to identify the column brake.
%
%%% ONE-COLUMN TABLE
%
%%t
%\begin{table}[t]
%\caption{TEXT}
%\begin{tabular}{column = lcr}
%\tophline
%
%\middlehline
%
%\bottomhline
%\end{tabular}
%\belowtable{} % Table Footnotes
%\end{table}
%
%%% TWO-COLUMN TABLE
%
%%t
%\begin{table*}[t]
%\caption{TEXT}
%\begin{tabular}{column = lcr}
%\tophline
%
%\middlehline
%
%\bottomhline
%\end{tabular}
%\belowtable{} % Table Footnotes
%\end{table*}
%
%%% LANDSCAPE TABLE
%
%%t
%\begin{sidewaystable*}[t]
%\caption{TEXT}
%\begin{tabular}{column = lcr}
%\tophline
%
%\middlehline
%
%\bottomhline
%\end{tabular}
%\belowtable{} % Table Footnotes
%\end{sidewaystable*}
%
%
%%% MATHEMATICAL EXPRESSIONS
%
%%% All papers typeset by Copernicus Publications follow the math typesetting regulations
%%% given by the IUPAC Green Book (IUPAC: Quantities, Units and Symbols in Physical Chemistry,
%%% 2nd Edn., Blackwell Science, available at: http://old.iupac.org/publications/books/gbook/green_book_2ed.pdf, 1993).
%%%
%%% Physical quantities/variables are typeset in italic font (t for time, T for Temperature)
%%% Indices which are not defined are typeset in italic font (x, y, z, a, b, c)
%%% Items/objects which are defined are typeset in roman font (Car A, Car B)
%%% Descriptions/specifications which are defined by itself are typeset in roman font (abs, rel, ref, tot, net, ice)
%%% Abbreviations from 2 letters are typeset in roman font (RH, LAI)
%%% Vectors are identified in bold italic font using \vec{x}
%%% Matrices are identified in bold roman font
%%% Multiplication signs are typeset using the LaTeX commands \times (for vector products, grids, and exponential notations) or \cdot
%%% The character * should not be applied as mutliplication sign
%
%
%%% EQUATIONS
%
%%% Single-row equation
%
%\begin{equation}
%
%\end{equation}
%
%%% Multiline equation
%
%\begin{align}
%& 3 + 5 = 8\\
%& 3 + 5 = 8\\
%& 3 + 5 = 8
%\end{align}
%
%
%%% MATRICES
%
%\begin{matrix}
%x & y & z\\
%x & y & z\\
%x & y & z\\
%\end{matrix}
%
%
%%% ALGORITHM
%
%\begin{algorithm}
%\caption{...}
%\label{a1}
%\begin{algorithmic}
%...
%\end{algorithmic}
%\end{algorithm}
%
%
%%% CHEMICAL FORMULAS AND REACTIONS
%
%%% For formulas embedded in the text, please use \chem{}
%
%%% The reaction environment creates labels including the letter R, i.e. (R1), (R2), etc.
%
%\begin{reaction}
%%% \rightarrow should be used for normal (one-way) chemical reactions
%%% \rightleftharpoons should be used for equilibria
%%% \leftrightarrow should be used for resonance structures
%\end{reaction}
%
%
%%% PHYSICAL UNITS
%%%
%%% Please use \unit{} and apply the exponential notation


\end{document}

